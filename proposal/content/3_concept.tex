\chapter{Concept}

The concept for the master thesis is to take the idea of the QCM, specifically the core concept of quantum control flow, and reduce it to its most basic elements and make it realistic for and applicable to NISQ era quantum computers. Concretely, we want to go from jump instructions to basic if-else clause to reduce the complexity of the code and make it easier to read and write. These if-else clauses can easily be implemented as the application of controlled gates. Moreover, because of the synchronization principle and the fact that current quantum computer technology does not support loops depending on measurement, any other loop can be reduced to a for-loop that is unrolled at compile time.

To achieve this goal, we want to define a language "Luie" (short for loop-unrolled if-else) which is partially based on the quantum while language used by Ying~\cite{Ying11}. The language is extended by a quantum if clause which takes a quantum register and executes the statements in the clause based on the value of the register. Furthermore, the clause could even be extended to include the evaluation of boolean expression. While the language cannot include while statements based on measurements of registers, as it is the case in the language proposed by Ying, it can include bounded loops which are unrolled at compile time. The language will then be compiled to QASM. A basic grammar for the language can be seen in Appendix~\ref{appendix:grammar}.


% \begin{itemize}
%     \item Idea \info{Explain in concept} 
%     \begin{itemize}
%         \item reduce QCM to basics
%         \item lead to concept section
%     \end{itemize}
%     \item Language features: qif-else, bounded loops, (boolean eval)
%     \item Translation to quasm
%     \item overall (more) realistic for NISQ 
%     \item Further (compiler optimizations)
%     \item Example grammar
% \end{itemize}

