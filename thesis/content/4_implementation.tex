\chapter{Implementation}
\label{ch:implementation}

\section{Semantic analysis}
\begin{itemize}
    \item What is semantic analysis used for?
    \item How is it implemented in Luie?
    \item Different types of semantic analysis
    \item Errors
    \begin{itemize}
        \item Types of errors: Critical, warning
        \item Different critical errors (Type, undefined, \dots)
        \item Different warnings (invalid range, \dots)
    \end{itemize}
\end{itemize}

\section{Code Generation}
\begin{itemize}
    \item How is code generated?
    \item Important classes and abstractions
\end{itemize}

\section{Language Overview}

\subsection{Blocks and Scopes}
\begin{itemize}
    \item Basic structure of Luie
    \item Consists of blocks and statements
    \item One main block
    \item Symbol table that handles scopes
    \item All blocks have scope 
\end{itemize}
\subsubsection{Grammar}
\begin{itemize}
    \item A block consists of arbitrarily many definitions and statements
\end{itemize}

\begin{lstlisting}[style=ANTLR]
    grammar Luie;
    
    parse
     : block EOF
     ;
    
    block
     : (definition | statement)*
     ;
\end{lstlisting}

\subsection{Data Types}
\begin{itemize}
    \item Different data types
    \begin{itemize}
        \item Register
        \item Qubits (Registers with size 1)
        \item Iterators, in more detail in Sec.~\ref{sec:control_flow}
    \end{itemize}

\end{itemize}

\subsubsection{Grammar}
\begin{itemize}
    \item Registers and qubits defined in declaration
    \item Optional size, otherwise qubit (size 1).
    \item Identifier is the name of the register/qubit
    \item Arbitrary string starting with a character or underscore
    \item Later 
\end{itemize}
\begin{lstlisting}[style=ANTLR]
    declaration
     : 'qubit' ('[' size=expression ']')? IDENTIFIER ';'
     ;

    IDENTIFIER 
      : [a-zA-Z_] [a-zA-Z_0-9]*
      ;
\end{lstlisting}
\subsubsection{Semantic analysis}
\begin{itemize}
    \item Semantic analysis can be differentiated between checking the use of an identifier and checking the type of an identifier.
    \item \emph{Definedness}: symbol table to check if an identifier is defined
    \item \emph{Type checking}: symbol table to check if the type of an identifier is correct
    \begin{itemize}
        \item What are different type 
        \item Object oriented approach, i.e. because qubit inherits from register it is also a register
        \item -> less checks required
    \end{itemize}
\end{itemize}

\subsubsection{Code Generation}
\begin{itemize}
    \item \texttt{Definition}s used for code generation, symbols can be transformed into definitions
    \item Needs unique identifier (language has scopes -> multiple variables with same name possible, while QASM does not)
    \item Unique identifiers given at generation time because of for loops
    \begin{itemize}
        \item Explain why \dots
    \end{itemize}
\end{itemize}

\subsection{Gate Application}
\subsubsection{Grammar}
\subsubsection{Semantic Analysis}
\subsubsection{Code Generation}

\subsection{Control Flow}
\label{sec:control_flow}
\subsubsection{Grammar}
\subsubsection{Code Generation}

\subsection{Expressions}
\subsubsection{Grammar}
\begin{itemize}
    \item Consists of expressions, terms and factors
    \begin{itemize}
        \item Expressions consist of expression, operator, and term or just a term
        \item Term consists of term, operator, and factor or just a factor
        \item Factor consists of expression in parentheses, a negated factor, number, identifier or function call
    \end{itemize}
    \item Inherent order of operations
\end{itemize}
\subsubsection{Evaluation}
\begin{itemize}
    \item All expressions evaluated at compile time
    \item All expressions inherit from abstract \texttt{Expression} class, which has an \texttt{Evaluate} method
    \item Generic return type \texttt{T}
    \item \dots
\end{itemize}

\section{Optimization}
\subsection{Constant folding}
\begin{itemize}
    \item Inherent in the language
    \item All variables known at compile time
    \item Any expression is evaluated at compile time
\end{itemize}
\subsection{Peephole optimization}
\begin{itemize}
    \item Not implemented, but planed
    \item Replace sequences of gates with more efficient ones
\end{itemize}

\section{Test Cases}
\begin{itemize}
    \item Different test categories
    \item How are they implemented?
    \item What do they test?
    \item (Continuous integration) 
\end{itemize}