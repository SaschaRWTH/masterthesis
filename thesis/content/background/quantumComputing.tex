\section{Quantum Computing}
\label{sec:background_quantumComputing}
While computers are prevalent and important in today's society, there are many relevant problems that classical computers cannot currently and perhaps will never realistically be able to solve. Quantum computing (QC) is gaining more momentum as the technology that could solve at least some of these problems. For example, quantum algorithms like Shor's algorithm~\cite{Shor97} could provide a significant improvement for prime factorization given sufficient technology. Therefore, it is estimated to be a valuable market with many of the largest technology companies as well as governments investing billions in the research and development of quantum technology~\cite{RDB*22, Pres18}. 
In the following section, we take a look at the basic concepts of a quantum computer and the core principles it relies on.

Classical computers are based on simple operations executed on bits, like \texttt{and}, \texttt{or}, and \texttt{not}. These bits can either have a value of $0$ or $1$. Similarly, at their core, quantum computers apply simple operations, like \texttt{controlled-not} and \texttt{Hadamard}, on quantum bits (qubits). 
On a higher level, a classical computer executes operations on a register, consisting of multiple bits, while a quantum computer operates on quantum registers, consisting of multiple qubits.
In contrast to classical bits, quantum computers use the unique properties of quantum mechanics to enable qubits to have not just one value of either $0$ or $1$ but a combination of both. The phenomenon, where a particle or qubit exists in a combination of both states, is called \emph{superposition}. Additionally, quantum computers also use the idea of \emph{entanglement} to their advantage. Two qubits are entangled when the value of one is dependent on the value of the other. The combination of superposition and entanglement enables quantum computers to solve specific problems more efficiently than classical computers~\cite{RDB*22}.

Models for quantum computers can be divided into three main categories: the \emph{analog model}, the \emph{measurement-based model}, and the \emph{gate-based model}. The analog model uses smooth operations to evolve a quantum system over time such that the resulting system encodes the desired result with high probability. It is not clear whether this model allows for universal quantum computation or quantum speedup~\cite{DiCh20b}. Instead of smoothly evolving a system, the measurement-based model starts with a fixed quantum state, the cluster state. The computation is accomplished by measuring qubits of the system, possibly depending on the results of previous measurements. While there are different measurement-based models, one technique to apply gates is to leverage quantum teleportation, so-called gate teleportation~\cite{Jozs05}. The result is a bit string of the measurement results~\cite{DiCh20b, Niel06}. Lastly, the gate-based model uses a digitized, discrete set of qubits that are manipulated by a sequence of operations represented by quantum gates. The result is obtained by measuring the qubits at the end of the computation. Often, intermediate measurements are also supported.
Although digital quantum computation is more sensitive to noise than analog computations, the digitization can also be used for quantum error correction~\cite{DMN13} and to mitigate the increased noise~\cite{DiCh20b}. Furthermore, because qubits are actively manipulated and not passively evolved, digital quantum computers are more flexible than analog ones~\cite{RDB*22}. Therefore, the gate-based model is the most common model, and this thesis will mainly focus on it. 

\subsection{Superposition}
\label{sec:background_superposition}
The first important property of quantum mechanics used by quantum computers is the idea of superposition. Qubits in superposition are often informally described as simultaneously having a value of $0$ and $1$ until their state is measured. However, a qubit in superposition is more formally a linear combination of its basis states. The basis states are the states where the qubit has a value of $0$, written $\ket{0} = \begin{pmatrix} 1 \\ 0 \end{pmatrix}$, and $1$, written $\ket{1} = \begin{pmatrix} 0 \\ 1 \end{pmatrix}$~\cite{DiCh20a}.
Furthermore, the state can be reduced to a simple vector. Therefore, a state $\psi$ in superposition can be written as:
\begin{equation*}
    \ket{\psi} = \alpha \ket{0} + \beta \ket{1} = \alpha \begin{pmatrix} 1 \\ 0 \end{pmatrix} + \beta \begin{pmatrix} 0 \\ 1 \end{pmatrix} = \begin{pmatrix} \alpha \\ \beta \end{pmatrix}.
\end{equation*}

The factors $\alpha$ and $\beta$ are the amplitudes of the basis states and are complex numbers. The factors must also satisfy the condition $\abs{\alpha}^2 + \abs{\beta}^2 = 1$. This is a result of the relation between the amplitudes and the probability to which basis state the state will collapse when measured, described in Sec.~\ref{sec:background_measurement}.

Besides $\ket{0}$ and $\ket{1}$, there exist more relevant short-hands for quantum states. For example, $\ket{+}$ and $\ket{-}$ are states in uniform superposition, \ie, both basis states are equally likely, and they are often used when discussing quantum states and transformations. They are defined as follows:
\begin{equation*}
    \ket{+} = \frac{1}{\sqrt{2}} (\ket{0} + \ket{1}) = \frac{1}{\sqrt{2}} \begin{pmatrix} 1 \\ 1 \end{pmatrix}
    \quad  \text{and}  \quad
     \ket{-} = \frac{1}{\sqrt{2}} (\ket{0} - \ket{1})  = \frac{1}{\sqrt{2}} \begin{pmatrix} 1 \\ -1 \end{pmatrix}.
\end{equation*}

\subsection{Entanglement}
\label{sec:background_entanglement}
Another important quantum mechanical concept is entanglement. Simply said, two qubits are entangled when their values depend on each other. An example would be a quantum system where two qubits are in superposition and equally likely to collapse to either $0$ or $1$; whichever value one qubit collapses to when measured, the second one will also collapse to the same value.
Additionally, changes to one of the qubits can also affect the other one. This happens independent of the locations of the two qubits~\cite{RDB*22, HHHH09}.

A more formal definition for an entangled state uses the definition of a composite system. Two separate quantum systems can be represented as a single system with the tensor product of both systems. For example, the combined state $\ket{\psi}$ of the separate states $\ket{0}$ and $\ket{1}$ can be represented as:
\begin{equation*}
    \ket{\psi} 
    = \ket{0} \otimes \ket{1} 
    = \ket{01} 
    = \begin{pmatrix}
        0 \\    
        1 \\    
        0 \\    
        0     
    \end{pmatrix}
    = \begin{pmatrix}
        1 \\ 0
    \end{pmatrix} \otimes 
    \begin{pmatrix}
        0 \\ 1
    \end{pmatrix}.
\end{equation*} 
When a quantum state cannot be expressed as a tensor product of two states, the state is entangled. The previous example is a case of a maximally entangled Bell state~\cite{DiCh20a, MHH19}, often denoted $\beta_{00}$, and can be expressed as the following:
\begin{equation*}
    \beta_{00} = \frac{1}{\sqrt{2}} (\ket{00} + \ket{11}) = \frac{1}{\sqrt{2}} 
    \begin{pmatrix}
        1 \\
        0 \\
        0 \\
        1    
    \end{pmatrix}.
\end{equation*}


The entanglement of states is used by leveraging the effect of the qubits on each other to collaborate to calculate the result. Although this can be simulated on classical computers, it cannot be achieved ``natively'' because all classical bits are independent of each other. Moreover, quantum algorithms not using entangled states can often be simulated efficiently on classical computers~\cite{MHH19}. Therefore, entanglement is at the core of quantum computing, but it can also have unintended consequences one needs to be aware of when designing quantum algorithms.

To calculate specific functions or intermediate values, quantum algorithms may need to use additional qubits or registers whose state can, in turn, be entangled with the main data of the algorithm. If this entanglement is not resolved in time, the changes to the qubit or register can interfere with future calculations or measurements and cause the results to be invalid. This effect is called \emph{disruptive entanglement}~\cite{YVC24}.
To prevent disruptive entanglement, the concept of \emph{uncomputation}, formalized by Bennett~\cite{Benn73}, can be used. The idea is to apply computations such that the entanglement of the main data and auxiliary qubits is undone. One way to do this is to apply the inverse of applied computations in reverse order~\cite{DiCh20}.

\subsection{Quantum Gates}
\label{sec:background_quantumGates}
In gate-based quantum computers, the transformations applied to the quantum data are represented by \emph{quantum gates}. Similar to quantum states, which can be represented by linear combinations of basis states, or vectors, quantum gates can be formulated as linear transformations of these combinations, or a matrix. Because the result of such a transformation also needs to be a valid quantum state, the transformation needs to be norm-preserving, or \emph{unitary}~\cite{DiCh20a}. The most relevant and often used unitary gates are depicted in Tab.~\ref{tab:gates}.

\begin{table}[htp]
    \centering
    \begin{tabular}{c|c|cc}
                                 & Gates    & Matrix & Ket-notation \\ \hline
    \multirow{5}{*}{Pauli gates} & X        
                                    & $\begin{pmatrix} 0 & 1 \\ 1 & 0 \end{pmatrix}$       
                                    & $\begin{matrix} \ket{0} \mapsto \ket{1} \\ \ket{1} \mapsto \ket{0} \end{matrix}$            \\ \cline{2-4} 
                                 & Y        
                                    & $\begin{pmatrix} 0 & -i \\ i & 0 \end{pmatrix}$       
                                    & $\begin{matrix} \ket{0} \mapsto \ket{i} \\ \ket{1} \mapsto -\ket{i} \end{matrix}$             \\ \cline{2-4} 
                                 & Z        
                                    & $\begin{pmatrix} 1 & 0 \\ 0 & -1 \end{pmatrix}$       
                                    & $\begin{matrix} \ket{0} \mapsto \ket{0} \\ \ket{1} \mapsto -\ket{1} \end{matrix}$             \\ \hline
    Hadamard gate                & H 
                                    & $\frac{1}{\sqrt{2}}\begin{pmatrix} 1 & 1 \\ 1 & -1 \end{pmatrix}$       
                                    & $\begin{matrix} \ket{0} \mapsto \ket{+} \\ \ket{1} \mapsto \ket{-} \end{matrix}$             \\ \hline
    Phase gate                   & $P(\lambda)$ 
                                    & $\begin{pmatrix} 1 & 0 \\ 0 & e^{i\lambda} \end{pmatrix}$       
                                    & $\begin{matrix} \ket{0} \mapsto \ket{0} \\ \ket{1} \mapsto e^{i\lambda} \cdot \ket{1} \end{matrix}$             \\ \hline
    Controlled-NOT gate          & CX     
                                    & $\begin{pmatrix} 1 & 0 & 0 & 0\\ 0 & 1 & 0 & 0\\ 0 & 0 & 0 & 1\\ 0 & 0 & 1 & 0 \end{pmatrix}$       
                                    & $\begin{matrix} \ket{00} \mapsto \ket{00} \\ \ket{01} \mapsto \ket{01} \\ \ket{10} \mapsto \ket{11} \\ \ket{11} \mapsto \ket{10} \end{matrix}$\\ \hline
    Toffoli  gate                & CCX     
                                    & $\begin{pmatrix} 1 & 0 & 0 & 0 & 0 & 0 & 0 & 0 \\ 0 & 1 & 0 & 0 & 0 & 0 & 0 & 0 \\ 0 & 0 & 1 & 0 & 0 & 0 & 0 & 0 \\ 0 & 0 & 0 & 1 & 0 & 0 & 0 & 0 \\ 0 & 0 & 0 & 0 & 1 & 0 & 0 & 0 \\ 0 & 0 & 0 & 0 & 0 & 1 & 0 & 0 \\ 0 & 0 & 0 & 0 & 0 & 0 & 0 & 1 \\ 0 & 0 & 0 & 0 & 0 & 0 & 1 & 0 \end{pmatrix}$       
                                    & $\begin{matrix} \ket{000} \mapsto \ket{000} \\ \ket{001} \mapsto \ket{001} \\ \ket{010} \mapsto \ket{010} \\ \ket{011} \mapsto \ket{011} \\  \ket{100} \mapsto \ket{100} \\ \ket{101} \mapsto \ket{101} \\ \ket{110} \mapsto \ket{111} \\ \ket{111} \mapsto \ket{110} \end{matrix}$            
    \end{tabular}
    \caption{List of relevant quantum gates in matrix representation as as functions in ket-notation.}
    \label{tab:gates}
\end{table}

A matrix $U$ is unitary if it has an inverse matrix that is equal to its conjugate transpose $U^\dagger$, \ie, the following must hold:
\begin{equation*}
    U U^\dagger = I.
\end{equation*}
Therefore, all transformations applied to quantum states in a gate-based quantum computer must be reversible by definition. This limitation does not apply to classical computers, where non-reversible transformations, \eg, mapping an arbitrary bit to a specific value, are easily implementable.

To design a useful quantum computer or language, the set of gates should be \emph{universal}. A set of gates is universal if any gate can be simulated by a combination of the gates from the set with arbitrary accuracy~\cite{BrBr02}. An example of a universal set of gates is the combination of the Toffoli gate together with the Hadamard gate~\cite{DiCh20a}.

\subsection{Measurement}
\label{sec:background_measurement}
For a quantum computer to be of any use, we need a way to read out information about its state. However, the information we can obtain from a quantum system is limited by the quantum measurement postulate. The postulate states that the only way to gain any information from a quantum system is to measure it. When measuring a quantum state, the state irreversibly collapses to one of its basis states.
Furthermore, this is a probabilistic transformation, and the original state in superposition cannot be recovered from the result. Therefore, in contrast to all other transformations, measurements are neither unitary nor reversible. 
For a state $\ket{\psi} = \alpha \ket{0} + \beta \ket{1}$, the measurement collapses the state to $\ket{0}$ with a probability of $\abs{\alpha}^2$. Correspondingly, the state will collapse to $\ket{1}$ with a probability of $\abs{\beta}^2$ when measured~\cite{DiCh20a}.

Measurement can be represented as a measurement basis set $\{M_i\}_i$ which requires the following condition:
\begin{equation*}
    \sum_i M_i^\dagger M_i = I.
\end{equation*}
The probability that outcome $i$ is obtained when measuring a state $\ket{\psi}$ is equivalent to $\abs{M_i \ket{\psi}}^2$. After the measurement of outcome $i$, the state $\ket{\psi'}$ will be equivalent to
\begin{equation*}
    \ket{\psi'} = \frac{M_i \ket{\psi}}{|M_i \ket{\psi}|} = \frac{M_i \ket{\psi}}{\sqrt{\Probability{[\text{observe }i]}}}.
\end{equation*} 

Additionally, measurements in quantum programming languages can be differentiated between implicit and explicit measurements. Implicit measurements rely on two concepts. Any discarded qubit is implicitly measured, and any measurement can be delayed until the end of a circuit. In turn, instead of explicitly measuring a qubit in a quantum circuit or program, all qubits can be implicitly measured. However, some quantum communication protocols may rely on the explicit measurements of qubits. Therefore, implicit measurements are not always the best solutions, and many quantum languages include explicit measurements~\cite{Inou24}.

\subsection{Relevant Algorithms}
\label{sec:background_algorithms}
Since quantum computers differ greatly from classical computers not only in their technology but also in the concepts they use for calculation, they cannot function without specially designed algorithms. These algorithms need to exploit the special quantum properties of qubits to achieve \emph{quantum advantage}, \ie, a better complexity than any classical algorithm. One of the first algorithms to show its quantum advantage was the Deutsch–Josza algorithm~\cite{DeJo92}. Deutsch et al. define a problem that can be solved in exponential time on classical computers and present a quantum algorithm that can solve the problem in polynomial time. The Bernstein-Vazirani algorithm~\cite{BeVa93} is another example with shown quantum advantage, resulting in a polynomial speed-up. However, currently, there does not exist a use case for either of the algorithms, and, therefore, they are only of limited theoretical interest~\cite{DiCh20c}.

An algorithm with more potential for practical use is Shor's algorithm~\cite{Shor97}. It presents a more efficient, polynomial-time quantum implementation for the discrete logarithm, \ie, find $r$ for a given $a$, $x$, and $p$ such that $a^r = x \mod p$. The algorithm is of special interest because Shor also provides a reduction of prime factorization to order finding; order finding is a special case of the discrete logarithm where $x = 1$. Modern cryptography is often based on the complexity of factoring large prime numbers, \eg, the commonly used RSA cryptosystem~\cite{RSA78}. Therefore, an advanced quantum computer could break these systems with Shor's algorithm~\cite{MVZJ18}. Not only does this prospect provide a practical use case for QC, but it also results in the research field of \emph{post-quantum cryptography}~\cite{BeLa17}.

Another relevant algorithm or transformation is the quantum Fourier transform (QFT)~\cite{Copp02}. Besides being used as a subroutine in Shor's algorithm, it is also relevant for other algorithms, \eg, an addition algorithm for quantum registers~\cite{Drap00}. Similar to the discrete Fourier transform~\cite{Wino78}, which operates on vectors, the QFT$_{2^n}$ operates on the quantum equivalent of vectors, quantum registers, of size $n$. Registers of size $n$ consist of $n$ qubits. From the register, the QFT extracts the present periodic features. Then, other algorithms can use these features for their calculations.

\subsection{Circuit optimization}
\label{sec:background_circuitOptimization}
Despite the expansive theoretical foundations for QC, the current state of the art for its technology is limited. However, the technology is nearing its first milestone towards usable quantum computers with the advent of prototypes with noisy intermediate-scale quantum (NISQ) technology~\cite{BFA22}. Nevertheless, the technology is still far away from fault-tolerant quantum computers and, by definition, limited in the number of available qubits. Furthermore, the gate count of NISQ-era quantum computers is limited by the inherent noise, which is increased with each additional transformation~\cite{Pres18}. Therefore, attributes such as the gate count of a quantum algorithm are an important metric for its utility. To improve the utility of an algorithm, its quantum circuit can be optimized with different techniques and rules.

There exist many kinds of optimization techniques for quantum circuits. They are mostly concerned with optimizing the gate count of quantum circuits with the use of peephole optimizations, as described in Sec.~\ref{sec:background_codeOptimization}. These techniques can range from general rules~\cite{GaCh11, LBZ21}, which can be applied to all quantum circuits, to hardware-specific optimizations~\cite{KMO*23}. Furthermore, machine learning-based optimization frameworks for quantum circuits are also gaining popularity~\cite{FNML21,LPM*24, RLB*24}.

The simplest general optimizations are so-called \emph{null gates}~\cite{GaCh11}. They are gate combinations or gates under specific conditions that are equivalent to the identity gate $I$. Therefore, any occurrence of such a null gate can be removed from the circuit. The most basic example of null gates is the double application of a self-inverse gate, \ie, a gate that is its own inverse. These include the $H, X, Y,$ and $Z$ gates. The identities are also depicted in Fig.~\ref{fig:nullgates_selfInverse}.
Furthermore, the same holds for any controlled version of a self-inverse gate such that the rule can also be applied to $CNOT$ and similar gates. 

\begin{figure}[htp!]
    %\centering
    \[
        \Qcircuit @C=.5em @R=0em {
            & \gate{H} & \gate{H} & \qw & \push{\rule{.3em}{0em}=\rule{.3em}{0em}} & & \gate{X} & \gate{X} & \qw & \push{\rule{.3em}{0em}=\rule{.3em}{0em}} & & \gate{Y} & \gate{Y} & \qw & \push{\rule{.3em}{0em}=\rule{.3em}{0em}} & & \gate{Z} & \gate{Z} & \qw & \push{\rule{.3em}{0em}=\rule{.3em}{0em}} & & \gate{I} & \qw
            }
    \]
    \caption{Null gates of self-inverse gates.}
    \label{fig:nullgates_selfInverse}
\end{figure}

The second kind of null gates are gates that do not have an effect under specific conditions. For example, a controlled gate $U$ has no effect on its qubits if we know that the control is $\ket{0}$. Similarly, the $X$ gate does not have an effect on a qubit in the $\ket{+}$ state. Hence, the two circuits depicted in Fig.~\ref{fig:nullgates_control} are null gates and semantically equivalent to an identity gate.
\begin{figure}[htp!]
    \[
        \Qcircuit @C=1em @R=1em {
            \lstick{\ket{0}} & \ctrl{1} &  \qw &&&&& \lstick{\ket{\psi}} & \ctrl{1} & \qw \\
            \lstick{\ket{\psi}} & \gate{U} & \qw &&&&& \lstick{\ket{+}} & \targ & \qw
            }
    \]
    \caption{Null gates for gate in specific conditions.}
    \label{fig:nullgates_control}
\end{figure}

Another class of optimizations is called \emph{control reversal}. Control reversal describes gate combination equalities based on the symmetry of the controlled $Z$ gate. For the controlled $Z$ gate, it is semantically equivalent to apply the $Z$ gate to the second wire with a control on the first and to apply the $Z$ gate on the first wire with a control on the second one. Based on this and together with the equalities $HZH = X$ and $HXH = Z$, a controlled $X$ gate surrounded by $H$ gates on both wires can be represented as the reversed $X$ gate. Both equalities are depicted in Fig.~\ref{fig:control_reversal_cz} and Fig.~\ref{fig:control_reversal_hcx}. 

\begin{figure}[htp!]
    \centering
    \begin{minipage}{.45\textwidth}
        \[
            \Qcircuit @C=1em @R=1em {
                & \ctrl{1} & \qw & \raisebox{-2.2em}{=} & & \gate{Z} & \qw \\
                & \gate{Z} & \qw &           & & \ctrl{-1} & \qw
                }
        \]
        \caption{Control reversal of the controlled Z gate.}
        \label{fig:control_reversal_cz}
    \end{minipage}
    \hfill
    \begin{minipage}{.45\textwidth}
        \[
            \Qcircuit @C=1em @R=1em {
               & \gate{H} & \ctrl{1} & \gate{H} & \qw & \raisebox{-2.2em}{=} & & \targ & \qw \\
               & \gate{H} & \targ & \gate{H} & \qw &           & & \ctrl{-1} & \qw
                }
        \]
        \caption{Control reversal of CX.}
        \label{fig:control_reversal_hcx}
    \end{minipage}
\end{figure}

In contrast to general optimization rules, hardware-specific optimizations are mostly not concerned with the reduction of gates based on mathematically equal gate combinations; however, they exploit either the specific properties of the hardware for optimizations or replace gates with cheaper equivalents on the specific hardware. For example, a shuttling-based trapped-ion quantum computer operates by physically moving ions to segments in the hardware where operations can be applied. Since ions can and must be freely moved, swapping qubits can easily be accomplished by physically changing the position of their hardware equivalent. On the software side, this can be achieved by removing the swap gate and swapping all following instances of both qubits~\cite{KMO*23}.