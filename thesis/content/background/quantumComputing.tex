\section{Quantum Computing}
% \begin{itemize}
%     \item Introduction into quantum computing
%     \item From bits to qubits
%     \item Entanglement
% \end{itemize}
While computers are prevalent and important in todays society, there are many relevant problem which classical computers can currently and perherps will never realistically be able to solve. Quantum Computing is gaining more momentum as the technology that could solve at least some of these problems. For example, Quantum algorithms like Shor's algorithm~\cite{Shor97} could provide a significant improvement for prime factorization given sufficient technology. Therefore, it is estimated to be a valuable market with many of the largest technology companies as well as governments investing billions in the research and development of quantum technology~\cite{RDB*22}. While there already exist detailed theoretical foundations~\cite{van20, Ying11,YYF12} and advanced algorithms for QC~\cite{ACR*10,BGB*18,LoCh19,Shor97}, the technology of quantum computers is said to be on the level of classical computer in the 1950s~\cite{CFM17}. In the following section, we take a look at the basic concepts of a quantum computer and the core principles it relies on.

Classical Computers are based on simple operations, like \texttt{and}, \texttt{or}, and \texttt{not}, on bits. These bits can either have a value of $0$ or $1$. Similarly, at their core, quantum computers apply simple operations, like \texttt{controlled not}, and \texttt{hadamar}, on quantum bits (qubits). In contrast to classical bits, quantum computers use the unique properties of quantum mechanics to enable qubits to have not just one value of either $0$ or $1$ but a combination of both\unsure{citation needed?}. Additionally, quantum computers also use the idea of entanglement to their advantage where the value of a qubit is dependent on another qubit. The combination of superposition and entanglement enable quantum computers to solve specific problems more efficiently than classical computers~\cite{RDB*22}, e.g. prime factorization~\cite{Shor97}\unsure{Repeating info from paragraph above?}.

Different QC models: analog/digital -> focus on digital, use~\cite{DiCh20}

\subsection{Superposition}
The concept of superposition is most known for its role in the ``Schrödinger's cat'' thought experiment~\cite{Wine13} where the life of a cat is dependent on a particle in superposition. Similar to the cat being refered to as alive and dead at the same time, qubits in superposition are often informally described as simultaniously having a value of $0$ and $1$. However, a qubit in superposition is more formally a linear combination of its basis states. The basis states are the states where the qubit has a value of $0$, written $\ket{0} = \begin{pmatrix} 1 \\ 0 \end{pmatrix}$, and $1$, written $\ket{1} = \begin{pmatrix} 0 \\ 1 \end{pmatrix}$\unsure{also describe the definition of $\ket{+}, \ket{-}$?}. Therefore, a state $\psi$ in superposition can be written as:
\begin{equation*}
    \ket{\psi} = \alpha \ket{0} + \beta \ket{1} = \alpha \begin{pmatrix} 1 \\ 0 \end{pmatrix} + \beta \begin{pmatrix} 0 \\ 1 \end{pmatrix}
\end{equation*}
The factors $\alpha$ and $\beta$ are the amplitudes of the basis states and are complex numbers. The probability that the state collapses to a basis state when measured is related to be corresponding amplitude of that basis state and can be computed by squaring it. Because a state will always collapse to a basis state when measured, the sum of the squares of the amplitudes must always be equal to one for the state to be a valid quantum state\unsure{formulate at math formula to be easier to read?}.

\subsection{Entanglement}
\begin{itemize}
    \item Explain entanglement
    \item How is entanglement relevant for quantum computing?
    \item Explain disruptive entanglement
\end{itemize}


\subsection{Operators and Gates}
\begin{itemize}
    \item General theoretical bases of operators and gates
    \item Most important gates and their functions
\end{itemize}

\subsection{Measurement}
\begin{itemize}
    \item How are qubits measured?
    \item What is the effect of measurement on qubits?
\end{itemize}

\subsection{Relevant Algorithms}
\begin{itemize}
    \item Shortly describe algorithms referenced later
    \item QFT (Quantum Fourier Transform) 
    \item Shor's algorithm 
\end{itemize}