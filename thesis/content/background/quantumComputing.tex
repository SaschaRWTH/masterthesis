\newpage
\section{Quantum Computing}
% \begin{itemize}
%     \item Introduction into quantum computing
%     \item From bits to qubits
%     \item Entanglement
% \end{itemize}
While computers are prevalent and important in today's society, there are many relevant problems which classical computers can currently and perhaps will never realistically be able to solve. Quantum Computing is gaining more momentum as the technology that could solve at least some of these problems. For example, Quantum algorithms like Shor's algorithm~\cite{Shor97} could provide a significant improvement for prime factorization given sufficient technology. Therefore, it is estimated to be a valuable market with many of the largest technology companies as well as governments investing billions in the research and development of quantum technology~\cite{RDB*22}. While there already exist detailed theoretical foundations~\cite{van20, Ying11,YYF12} and advanced algorithms for QC~\cite{ACR*10,BGB*18,LoCh19,Shor97}, the technology of quantum computers is said to be on the level of classical computers in the 1950s~\cite{CFM17}. In the following section, we take a look at the basic concepts of a quantum computer and the core principles it relies on.

Classical Computers are based on simple operations, like \texttt{and}, \texttt{or}, and \texttt{not}, on bits. These bits can either have a value of $0$ or $1$. Similarly, at their core, quantum computers apply simple operations, like \texttt{controlled not}, and \texttt{Hadamard}, on quantum bits (qubits). In contrast to classical bits, quantum computers use the unique properties of quantum mechanics to enable qubits to have not just one value of either $0$ or $1$ but a combination of both. The phenomenon, where a particle or qubit exists in multiple states at the same time\unsure{This is just a coloquially explaination and not technically correct}, is called \emph{superposition}. Additionally, quantum computers also use the idea of \emph{entanglement} to their advantage where the value of a qubit is dependent on another qubit. The combination of superposition and entanglement enable quantum computers to solve specific problems more efficiently than classical computers~\cite{RDB*22}, \eg prime factorization~\cite{Shor97}\unsure{Repeating info from paragraph above?}.

Models for Quantum Computers can be divided into three main categories, the \emph{analog model}, the \emph{measurement-based model}, and the \emph{gate-based model}. The analog model uses smooth operations to evolve a quantum system over time such that the resulting system encodes the desired result with high probability. It is not clear whether this model allows for universal quantum computation or quantum speedup~\cite{DiCh20}. Instead of smoothly evolving a system, the measurement-based model starts with a fixed quantum state, the cluster-state. The computation is accomplished by measuring qubits of the system, possibly depending on the results of previous measurements. The concept of gate teleportation~\unsure{explain?} is used such that the measurements realize quantum gates. The result is a bit-string of the measurement results~\cite{DiCh20, Niel06}. Lastly, the gate-based model uses a digitized, discrete set of qubits that are manipulated by a sequence of operations represented by quantum gates. The result is obtained by measuring the qubits at the end of the computation. Although digital quantum computation is more sensitive to noise than analog computations, the digitization can also be used for quantum error correction~\cite{DMN13} and mitigate the increased noise~\cite{DiCh20}. Furthermore, because qubits are actively manipulated and not passively evolved, digital quantum computers are more flexible than analog ones~\cite{RDB*22}. Therefore, the gate-based model is the most common model and this thesis will mainly focus on it. \improvement{Add section on: no cloning/deleting, implicit measurement theorems?\cite{WoZu82, KuBr00}}
% In contrast to the analog mode, both the other models were shown to be universal quantum computation models. (not specifically stated for gate in DiCh20)!

\subsection{Superposition}\unsure{Is a citation needed for this definition? (if yes use \cite{DiCh20}) } 
The first important property of quantum mechanics used by quantum computers is the idea of superposition. The concept of superposition is most known for its role in the ``Schrödinger's cat'' thought experiment~\cite{Wine13} where the life of a cat in a box is dependent on a particle in superposition, only when ``measuring'' the state of the cat, \ie looking into the box, we can know if it is still alive. 

Similar to the cat being referred to as alive and dead at the same time, qubits in superposition are often informally described as simultaneously having a value of $0$ and $1$ until their state is measured. However, a qubit in superposition is more formally a linear combination of its basis states. The basis states are the states where the qubit has a value of $0$, written $\ket{0} = \begin{pmatrix} 1 \\ 0 \end{pmatrix}$, and $1$, written $\ket{1} = \begin{pmatrix} 0 \\ 1 \end{pmatrix}$.
Furthermore, the state can be reduced to a simple vector. Therefore, a state $\psi$ in superposition can be written as:
\begin{equation*}
    \ket{\psi} = \alpha \ket{0} + \beta \ket{1} = \alpha \begin{pmatrix} 1 \\ 0 \end{pmatrix} + \beta \begin{pmatrix} 0 \\ 1 \end{pmatrix} = \begin{pmatrix} \alpha \\ \beta \end{pmatrix}.
\end{equation*}

The factors $\alpha$ and $\beta$ are the amplitudes of the basis states and are complex numbers. The factors must also satisfy the condition $\abs{\alpha}^2 + \abs{\beta}^2 = 1$. This is because of the relation of the amplitudes to the probability to which basis state the state will collapse when measure, described in Sec.~\ref{sec:background_measurement} about measurement.

Beside $\ket{0}$ and $\ket{1}$, there exist more relevant short hands for quantum state. For example, $\ket{+}$ and $\ket{-}$ are states in uniform superposition, \ie both basis state are equally likely, and often used when discussing quantum state und transformations. They are defined as follows:
\begin{equation*}
    \ket{+} = \frac{1}{\sqrt{2}} (\ket{0} + \ket{1}) \quad  \text{and}  \quad \ket{-} = \frac{1}{\sqrt{2}} (\ket{0} - \ket{1}).
\end{equation*}

\subsection{Entanglement}
Another important quantum mechanical concept is entanglement. Simply said, two qubits are entangled when their values depend on each other. An example would be a quantum system where two qubits are in superposition and equally likely to collapse to either $0$ or $1$; whichever value one qubit collapses to when measured, the second one will also collapse to the same values.\improvement{Define Bell $\beta_{00}$ state}
Additionally, changes to one of the qubits can also affect the other one. This happens independent of the locations of the two qubits~\cite{RDB*22, HHHH09}.

A more formal definition for an entangled state uses the definition of a composite system. Two separate quantum system can be represented as a single system with the tensor product of both systems. For example, the combined state $\ket{\psi}$ of the separate states $\ket{0}$ and $\ket{1}$ can be represented as:
\begin{equation*}
    \ket{\psi} 
    = \ket{0} \otimes \ket{1} 
    = \ket{01} 
    = \begin{pmatrix}
        0 \\    
        1 \\    
        0 \\    
        0     
    \end{pmatrix}
    = \begin{pmatrix}
        1 \\ 0
    \end{pmatrix} \otimes 
    \begin{pmatrix}
        0 \\ 1
    \end{pmatrix}.
\end{equation*} 
When a quantum state cannot be expressed as a tensor product of two states, the state is entangled. The previous example is a case of a maximally entanglement Bell state~\cite{DiCh20, MHH19}, often denoted $\beta_{00}$, and can be expressed as the following:
\begin{equation*}
    \beta_{00} = \frac{1}{\sqrt{2}} (\ket{00} + \ket{11}) = \frac{1}{\sqrt{2}} 
    \begin{pmatrix}
        1 \\
        0 \\
        0 \\
        1    
    \end{pmatrix}.
\end{equation*}


The entanglement of states is used by leveraging the effect of the qubits on each other to collaborate to calculate the result. Although this can be simulated on classical computers, it cannot be achieved ``natively'' because all classical bits are independent of each other. Moreover, quantum algorithms not using entangled states can often be simulated efficiently on classical computers~\cite{MHH19}. Therefore, entanglement is at the core of quantum computing but it can also have unintended consequences one needs to be aware of when designing quantum algorithms.

To calculate specific functions or intermediate values, quantum algorithms may need to use additional qubits or registers\change{register were not previously mentioned, add small reference} which state can, in turn, be entangled with the main data of the algorithm. If this entanglement is not resolved in time by, \eg, uncomputing\change{Uncomputing as a concept was not introduced before} the changes to the qubit or register, it can interfere with future calculations or measurements and cause the results to be invalid. This effect is called \emph{disruptive entanglement}~\cite{YVC24}\unsure{Cannot find literature besides \cite{YVC24} which calls this effect disruptive entanglement, use anyway?}.  


\subsection{Measurement}
\label{sec:background_measurement}
For quantum computer to be of any use, we need a way to read out information about its state. However, the information we can obtain from a quantum system is limited by the quantum measurement postulate. The postulate states that the only way, to gain any information from a quantum system, is to measure it. When measuring a quantum state, the state irreversibly collapses to one of its basis states.
Furthermore, this is a probabilistic transformation and the original state in superposition cannot be recovered from the result. 
For a state $\ket{\psi} = \alpha \ket{0} + \beta \ket{1}$, the measurement collapses the state to $\ket{0}$ with a probability of $\abs{\alpha}^2$. Correspondingly, the state will collapse to $\ket{1}$ with a probability of $\abs{\beta}^2$ when measured~\cite{DiCh20}.

Measurement can be represented a a measurement basis set $\{M_i\}_i$ which requires the following condition:
\begin{equation*}
    \sum_i M_i^\dagger M_i = I.
\end{equation*}
The probability that outcome $i$ is obtained when measuring a state $\ket{\psi}$ is equivalent to $\abs{M_i \ket{\psi}}^2$. After the measurement of outcome $i$, the state $\ket{\psi'}$ will be equivalent to
\begin{equation*}
    \ket{\psi'} = \frac{M_i \ket{\psi}}{|M_i \ket{\psi}|} = \frac{M_i \ket{\psi}}{\sqrt{\Probability{[\text{observe }i]}}}.
\end{equation*} 
In contrast to all other transformations, measurements are neither unitary nor reversible and, therefore, are able to ``destroy'' information on the quantum state before the measurement~\cite{DiCh20}.\improvement{already mentioned above, only mention once}

\subsection{Quantum Gates}
\label{sec:background_quantumGates}
In gate-based quantum computer, the transformations applied to the quantum data are represented by \emph{quantum gates}. Similar to quantum states, which can be represented by linear combinations of basis states, or vectors, quantum gates can be formulated as linear transformations of these combinations, or a matrix. Because the result of such a transformation also needs to be a valid quantum state, the transformation needs to be norm-preserving, or \emph{unitary}~\cite{DiCh20}. The most relevant and often used unitary gates are depicted in Tab.~\ref{tab:gates}\improvement{Add depications for gates in circuits?}\improvement{Add troffoli gate, large but important for universality?}.

\begin{table}[htp]
    \centering
    \begin{tabular}{c|c|cc}
                                 & Gates    & Matrix & Ket-notation \\ \hline
    \multirow{5}{*}{Pauli gates} & X        
                                    & $\begin{pmatrix} 0 & 1 \\ 1 & 0 \end{pmatrix}$       
                                    & $\begin{matrix} \ket{0} \mapsto \ket{1} \\ \ket{1} \mapsto \ket{0} \end{matrix}$            \\ \cline{2-4} 
                                 & Y        
                                    & $\begin{pmatrix} 0 & -i \\ i & 0 \end{pmatrix}$       
                                    & $\begin{matrix} \ket{0} \mapsto \ket{i} \\ \ket{1} \mapsto -\ket{i} \end{matrix}$             \\ \cline{2-4} 
                                 & Z        
                                    & $\begin{pmatrix} 1 & 0 \\ 0 & -1 \end{pmatrix}$       
                                    & $\begin{matrix} \ket{0} \mapsto \ket{0} \\ \ket{1} \mapsto -\ket{1} \end{matrix}$             \\ \hline
    Hadamard gate                & H 
                                    & $\frac{1}{\sqrt{2}}\begin{pmatrix} 1 & 1 \\ 1 & -1 \end{pmatrix}$       
                                    & $\begin{matrix} \ket{0} \mapsto \ket{+} \\ \ket{1} \mapsto \ket{-} \end{matrix}$             \\ \hline
    Phase gate                   & $P(\lambda)$ 
                                    & $\begin{pmatrix} 1 & 0 \\ 0 & e^{i\lambda} \end{pmatrix}$       
                                    & $\begin{matrix} \ket{0} \mapsto \ket{0} \\ \ket{1} \mapsto e^{i\lambda} \cdot \ket{1} \end{matrix}$             \\ \hline
    Controlled-NOT gate          & CX     
                                    & $\begin{pmatrix} 1 & 0 & 0 & 0\\ 0 & 1 & 0 & 0\\ 0 & 0 & 0 & 1\\ 0 & 0 & 1 & 0 \end{pmatrix}$       
                                    & $\begin{matrix} \ket{00} \mapsto \ket{00} \\ \ket{01} \mapsto \ket{01} \\ \ket{10} \mapsto \ket{11} \\ \ket{11} \mapsto \ket{10} \end{matrix}$\\ \hline
    Traffoli gate                & CCX     
                                    & $\begin{pmatrix} 1 & 0 & 0 & 0 & 0 & 0 & 0 & 0 \\ 0 & 1 & 0 & 0 & 0 & 0 & 0 & 0 \\ 0 & 0 & 1 & 0 & 0 & 0 & 0 & 0 \\ 0 & 0 & 0 & 1 & 0 & 0 & 0 & 0 \\ 0 & 0 & 0 & 0 & 1 & 0 & 0 & 0 \\ 0 & 0 & 0 & 0 & 0 & 1 & 0 & 0 \\ 0 & 0 & 0 & 0 & 0 & 0 & 0 & 1 \\ 0 & 0 & 0 & 0 & 0 & 0 & 1 & 0 \end{pmatrix}$       
                                    & $\begin{matrix} \ket{000} \mapsto \ket{000} \\ \ket{001} \mapsto \ket{001} \\ \ket{010} \mapsto \ket{010} \\ \ket{011} \mapsto \ket{011} \\  \ket{100} \mapsto \ket{100} \\ \ket{101} \mapsto \ket{101} \\ \ket{110} \mapsto \ket{111} \\ \ket{111} \mapsto \ket{110} \end{matrix}$            
    \end{tabular}
    \caption{List of relevant quantum gates in matrix representation as as functions in ket-notation.}
    \label{tab:gates}
\end{table}

A matrix $U$ is unitary if it has an inverse matrix which is equal to its conjugate transpose $U^\dagger$, \ie the following must hold:
\begin{equation*}
    U U^\dagger = I.
\end{equation*}
Therefore, all transformations applied to quantum states in a gate-based quantum computer must be reversible by definition. This limitation does not apply to classical computers where non-reversible transformations, \eg mapping an arbitrary bit to a specific value, are easily implementable. 

To design a useful quantum computer or language, the set of gates should be \emph{universal}. A set of gates is universal if any gate can be simulated by a combination of the gates from the set with arbitrary accuracy~\cite{BrBr01}. An example for a universal set of gates is the combination of the Traffoli gate together with the Hadamard gate~\cite{DiCh20}\improvement{very short section, expand on universality?}.

\subsection{Relevant Algorithms}
Although quantum computers have impressive technical abilities\change{write better introduction}, they cannot function without a specially designed algorithm. This algorithm needs to exploit the special quantum properties of qubits to achieve \emph{quantum advantage}, \ie a better complexity than any classical algorithm. One of the first algorithms to show its quantum advantage was the Deutsch–Josza algorithm~\cite{DeJo92}. Deutsch et al.\ define a problem that can be solved in exponential time on classical computer and present a quantum algorithm which can solve the problem in polynomial time. The Bernstein-Vazirani algorithm~\cite{BeVa93} is another example with shown quantum advantage, resulting in a polynomial speed up. However, currently, there does not exist a use case for either of the algorithms and, therefore, they are only of limited theoretical interest~\cite{DiCh20}.  

An algorithm with more potential for practical use is Shor's algorithm~\cite{Shor97}. It presents an efficient\improvement{poly time} quantum implementation for the discrete logarithm, \ie find $r$ for a given $a$, $x$, $p$ such that $a^r = x \mod p$. The algorithm is of special interest because Shor also provides a reduction of prime factorization to order finding; order finding is a special case of the discrete logarithm where $x = 1$. Modern cryptography is often based on the complexity of factoring large prime numbers~\info{discrete log is also used in modern cryptography}, \eg the commonly used RSA cryptosystem~\cite{RSA78}. Therefore, an advanced quantum computer could brake these systems with Shor's algorithm~\cite{MVZJ18}. Not only does this prospect provide a practical use-case for Quantum Computer but it also creates the research field of \emph{post-quantum cryptography}~\cite{BeLa17}.  

Another relevant algorithm or transformation is the quantum Fourier transform (QFT)~\cite{Copp02}. Beside being used as a subroutine in Shor's algorithm, it is also relevant for other algorithm, e.g. addition of quantum registers~\cite{Drap00}. Similar to the discrete Fourier transform~\cite{Wino78} which operates on vectors, the QFT$_{2^n}$ operates on the quantum equivalent of vector, quantum registers, of size $n$. Registers of size $n$ consist of $n$ qubits. From the register, the QFT extracts periodic features which are then used by the algorithms using the QFT\change{bad formulation}.

% possible subsection on Grovers algorithm