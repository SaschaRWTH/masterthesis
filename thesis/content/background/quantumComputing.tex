\section{Quantum Computing}
% \begin{itemize}
%     \item Introduction into quantum computing
%     \item From bits to qubits
%     \item Entanglement
% \end{itemize}
While computers are prevalent and important in today's society, there are many relevant problems which classical computers can currently and perhaps will never realistically be able to solve. Quantum Computing is gaining more momentum as the technology that could solve at least some of these problems. For example, Quantum algorithms like Shor's algorithm~\cite{Shor97} could provide a significant improvement for prime factorization given sufficient technology. Therefore, it is estimated to be a valuable market with many of the largest technology companies as well as governments investing billions in the research and development of quantum technology~\cite{RDB*22}. While there already exist detailed theoretical foundations~\cite{van20, Ying11,YYF12} and advanced algorithms for QC~\cite{ACR*10,BGB*18,LoCh19,Shor97}, the technology of quantum computers is said to be on the level of classical computers in the 1950s~\cite{CFM17}. In the following section, we take a look at the basic concepts of a quantum computer and the core principles it relies on.

Classical Computers are based on simple operations, like \texttt{and}, \texttt{or}, and \texttt{not}, on bits. These bits can either have a value of $0$ or $1$. Similarly, at their core, quantum computers apply simple operations, like \texttt{controlled not}, and \texttt{hadamar}, on quantum bits (qubits). In contrast to classical bits, quantum computers use the unique properties of quantum mechanics to enable qubits to have not just one value of either $0$ or $1$ but a combination of both\unsure{citation needed?}. Additionally, quantum computers also use the idea of entanglement to their advantage where the value of a qubit is dependent on another qubit. The combination of superposition and entanglement enable quantum computers to solve specific problems more efficiently than classical computers~\cite{RDB*22}, e.g. prime factorization~\cite{Shor97}\unsure{Repeating info from paragraph above?}.

Models for Quantum Computers can be divided into three main categories, the \emph{analog model}, the \emph{measurement-based model}, and the \emph{gate-based model}. The analog model uses smooth operations to evolve a quantum system over time such that the resulting system encodes the desired result with high probability. It is not clear whether this model allows for universal quantum computation or quantum speedup~\cite{DiCh20}. Instead of smoothly evolving a system, the measurement-based model starts with a fixed quantum state, the cluster-state. The computation is accomplished by measuring qubits of the system, possibly depending on the results of previous measurements. The concept of gate teleportation~\unsure{explain?} is used such that the measurements realize quantum gates. The result is a bit-string of the measurement results~\cite{DiCh20, Niel06}. Lastly, the gate-based model uses a discrete set of qubits that are manipulated by a sequence of operations represented by quantum gates. The result is obtained by measuring the qubits at the end of the computation. Although digital quantum computation is more sensitive to noise than analog computations, the digitization can also be used for quantum error correction and mitigate the increased noise~\cite{DiCh20}. The gate-based model is the most used model and this thesis will mainly focus on it. 
% In contrast to the analog mode, both the other models were shown to be universal quantum computation models. (not specifically stated for gate in DiCh20)!

\subsection{Superposition}\unsure{Is a citation needed for this definition? (if yes use \cite{DiCh20}) } 
The first important property of quantum mechanics used by quantum computers is the idea of superposition. The concept of superposition is most known for its role in the ``Schrödinger's cat'' thought experiment~\cite{Wine13} where the life of a cat in a box is dependent on a particle in superposition, only when ``measuring'' the state of the cat, i.e. looking into the box, we can know if it is still alive. 

Similar to the cat being referred to as alive and dead at the same time, qubits in superposition are often informally described as simultaneously  having a value of $0$ and $1$ until their state is measured. However, a qubit in superposition is more formally a linear combination of its basis states. The basis states are the states where the qubit has a value of $0$, written $\ket{0} = \begin{pmatrix} 1 \\ 0 \end{pmatrix}$, and $1$, written $\ket{1} = \begin{pmatrix} 0 \\ 1 \end{pmatrix}$\unsure{also describe the definition of $\ket{+}, \ket{-}$?}. Therefore, a state $\psi$ in superposition can be written as:
\begin{equation*}
    \ket{\psi} = \alpha \ket{0} + \beta \ket{1} = \alpha \begin{pmatrix} 1 \\ 0 \end{pmatrix} + \beta \begin{pmatrix} 0 \\ 1 \end{pmatrix}.
\end{equation*}

The factors $\alpha$ and $\beta$ are the amplitudes of the basis states and are complex numbers. The probability that the state will either be $0$ or $1$ when measured is related to be amplitude of the corresponding basis state and can be computed by squaring the absolute values of the amplitude. Because a state will always collapse to a basis state when measured, the following must hold for a state to be a valid quantum state:
\begin{equation*}
    \abs{\alpha}^2 + \abs{\beta}^2 = 1.
\end{equation*}

\subsection{Entanglement}
The second important quantum mechanical concept is entanglement. Simply said, two qubits are entangled when their values depend on each other. An example would be a quantum system where two qubits are in superposition and equally likely to collapse to either $0$ or $1$; whichever value one qubit collapses to when measured, the second one will also collapse to the same values. Additionally, changes to one of the qubits can also affect the other one. This happens independent of the locations of the two qubits~\cite{RDB*22, HHHH09}. 
\unsure{Also introduce notation for composition of systems?}

The entanglement of states is used by leveraging the effect of the qubits on each other to collaborate to calculate the result. Although this can be simulated on classical computers, it cannot be achieved ``natively'' because all qubits are independent of each other. Moreover, quantum algorithms not using entangled states can often be simulated efficiently on classical computers~\cite{MHH19}. Therefore, entanglement is at the core of quantum computing but it can also have unintended consequences one needs to be aware of when designing quantum algorithms.

... Disruptive entanglement

\subsection{Operators and Gates}
\begin{itemize}
    \item General theoretical bases of operators and gates
    \item Most important gates and their functions
\end{itemize}

\subsection{Measurement}
\begin{itemize}
    \item How are qubits measured?
    \item What is the effect of measurement on qubits?
\end{itemize}

\subsection{Relevant Algorithms}
\begin{itemize}
    \item Shortly describe algorithms referenced later
    \item QFT (Quantum Fourier Transform) 
    \item Shor's algorithm 
\end{itemize}