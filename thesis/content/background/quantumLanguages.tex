\section{Quantum Languages}
\label{sec:background_quantumLanguages}
With the emergence of quantum computing, many quantum languages were introduced. Most languages focus on a lower-level representation of quantum circuits. An example is the popular Open Quantum Assembly Language (OpenQASM)~\cite{CBSG17}. OpenQASM consists mainly of quantum and classical registers that can be manipulated by predefined and composite gates. Additionally, some classical control flow is possible with if-statements depending on classical bits or measurements. As its name suggests, the language is designed for low-level interactions with quantum computers and is mostly used to directly describe a quantum circuit. In Sec.~\ref{sec:background_qasm}, OpenQASM is discussed in more detail. 

In contrast to the low-level circuit descriptions of OpenQASM, there are also languages with a focus on high-level interactions. One such language is Tower~\cite{YuCa22}. It does not only allow for basic qubits and registers in superposition but also abstract data structures such as lists. Another example is the language Silq~\cite{BBGV20}, which allows for the automatic and safe uncomputation of registers after they have been used for, \eg, intermediate calculations. What both languages have in common is the restriction to quantum data while using only classical control flow. 

Although quantum control flow was formally defined by Ying et al.~\cite{YYF12}, as described in Sec.~\ref{sec:background_quantumControlFlow}, over ten years ago, only very few languages have incorporated the principle. One example is the functional programming language proposed by Altenkirch et al.~\cite{AlGr05}, where quantum branching is used to define, \eg, the Hadamard gate. Only recently was the Quantum Control Machine with quantum control flow at its core proposed by Yuan et al.~\cite{YVC24}. It presents an instruction set similar to classical assembly languages but for quantum computers and discusses the resulting limitations for the language. In the following section, we will discuss the quantum control machine in more detail.

\subsection{Quantum Control Machine}
\label{sec:background_quantumControlMachine}
The Quantum Control Machine (QCM), proposed by Yuan et al.~\cite{YVC24}, is an instruction set architecture that does not only allow for data in superposition but also quantum control flow. The architecture is designed around the limitations of control flow in superposition. 

The syntax and logic of the QCM are both heavily influenced by classical assembly languages. Similar to classical computers, the language provides a finite set of quantum registers that are all initialized to a value of zero. The instruction set of the architecture does not only provide limited gate transformations and swap operations but also more classical operations on registers, such as get-bit operations and simple arithmetical operations like addition and multiplication. However, what makes the QCM stand out are the jump instructions that enabled quantum control flow.

The gates of the architecture are limited to the $X$ and Hadamard gate $H$. However, since the QCM enables quantum branching, any gate can become a controlled gate such that the $X$ gate can easily be used in combination with quantum branching to create a Toffoli gate. Together with the Hadamard gate, the gate set is therefore universal, as described in Sec.~\ref{sec:background_quantumGates}.

There are three kinds of jump instructions. The first is a simple jump based on a given offset, the second is a conditional jump that performs a basic jump when a given register is $0$, and, lastly, an indirect jump, which is based on the value of a given register. Although the jump instructions are based on jumps in classical computers, they are limited by the restriction of unitary gates and must adhere to \emph{reversibility} and \emph{synchronization}~\cite{YVC24}, as described in Sec.~\ref{sec:background_quantumControlFlow}. An overview of some QCM instructions is depicted in Tab.~\ref{tab:qcm_instructionset}.

\begin{table}[htp]
    \centering
    \begin{tabular}{llp{.5\textwidth}}
        \multicolumn{1}{l|}{Operation}                          & \multicolumn{1}{l|}{Syntax}                                      & Semantics\footnote{After all operations, the instruction pointer is increased by the value of the branch control register.}                                                              \\ \hline
        
        \multicolumn{1}{l|}{No-op}                              & \multicolumn{1}{l|}{\texttt{nop}}               & Only increases instruction pointer by the branch control register.     \\ \hline
        
        \multicolumn{1}{l|}{Addition}                           & \multicolumn{1}{l|}{\texttt{add} $ra$ $rb$}     & Adds register $rb$ to $ra$.                                            \\
        \multicolumn{1}{l|}{Multiplication}                     & \multicolumn{1}{l|}{\texttt{mul} $ra$ $rb$}     & Multiplies register $ra$ by $rb$.                                      \\ \hline
        
        \multicolumn{1}{l|}{Jump}                               & \multicolumn{1}{l|}{\texttt{jmp} $p$}           & Increases branch control register by $p$.                              \\
        \multicolumn{1}{l|}{\multirow{2}{*}{Conditional Jumps}} & \multicolumn{1}{l|}{\texttt{jz} $p$ $ra$}       & Increases branch control register by $p$ if $ra$ is $0$.               \\
        \multicolumn{1}{l|}{}                                   & \multicolumn{1}{l|}{\texttt{jne} $p$ $ra$ $rb$} & Increases branch control register by $p$ if $ra$ is not equal to $rb$. 
        % \\ \hline \multicolumn{3}{p{.9\textwidth}}{\small * }                                                                  
    \end{tabular}
    \caption{An excerpt of the QCM instruction set with instructions used in later examples.}
    \label{tab:qcm_instructionset}
\end{table}

When quantum computers are based on unitary gates, all their operations need to be unitary and, therefore, reversible as well. This limits quantum jump instructions and prohibits them from working like their classical equivalent. However, the problem of a reversible architecture and instruction set is not unique to quantum computers but was also taken into consideration for classical architecture to, \eg, increase the energy efficiency of classical computers~\cite{AGY07, TAG12}. 
To enable reversible jumps, the QCM adapts the \emph{branch control register} from the reversible Bob architecture~\cite{TAG12}. Instead of directly changing the instruction pointer of the machine, the branch control register specifies how much the instruction pointer advances after each instruction.

The branch control register can then be manipulated reversibly by, \eg, adding or subtracting from it. To jump by a given $distance$, the branch control register needs to be increased to $distance$. However, after the instruction pointer has reached the desired location, the register needs to be decreased to its original value. Otherwise, the pointer would continue to jump in larger increments, and any further jumps, \ie modification to the branch control register, would not jump to the correct location. Since the jump instructions are defined to be reversible, the instruction set also includes a reverse jump instruction, which instead decreases the branch control register by a given offset. Therefore, a jump instruction always requires a reverse jump instruction to reset the program counter. Similarly, other operations can also be represented as the reverse operation of an existing one. For example, subtraction can be implemented as reverse addition. Further, to make the code easier to read and write, the QCM also allows for named labels, which can be used for jump instructions instead of offsets. The offset to the given label can then be computed at compile time. 

An example of a classical program and the reversible equivalent can be seen in Fig.~\ref{fig:qcm_not_reverse} and Fig.~\ref{fig:qcm_reverse}, respectively. Both programs calculate $x^y$ for two registers $x$ and $y$. While the first example has classical jumps that are not reversible, the second example uses reversible jump instructions and their reverse counterparts to create a reversible algorithm.

\begin{figure}[htp]
    \centering     
    \begin{minipage}{.40\textwidth}
        \lstinputlisting[style=QCM]{../figures/code/background/exp_non_reversible.qcm}
        \caption{A non-reversible exponentiation algorithm.}
        \label{fig:qcm_not_reverse}
    \end{minipage}
    \hfill
    \begin{minipage}{.50\textwidth}
        \lstinputlisting[style=QCM]{../figures/code/background/exp_non_sync.qcm}
        \caption{Reversible exponentiation algorithm.}    
        \label{fig:qcm_reverse}
    \end{minipage}
\end{figure}

Although such a program counter addresses the issue of reversibility, it can become entangled with data registers when in superposition. This can lead to disruptive entanglement where the output of the program becomes invalid~\cite{YVC24}. To prevent any disruptive entanglement of the data and control registers, QCM programs must adhere to the principle of synchronization, as described in Sec.~\ref{sec:background_quantumControlFlow}. 
It requires that the control flow is separated from the data at the end of execution. However, this is not the case for the reversible example program in Fig.~\ref{fig:qcm_reverse}, which, therefore, is not a valid QCM program.

The issue that occurs in the loop of the reversible example is the \emph{tortoise and hare} problem. Given a superposition of two different values $a$ and $b$ in the $y$ register, the loop will execute $a$ and $b$ times, respectively. Therefore, one of the two loops will finish before the other. Since we must adhere to synchronization, the instruction pointer needs to become independent of the two values again. However, because the branch with the faster execution of the loop cannot simply wait, the other branch cannot catch up, and the instruction pointer cannot become independent of the data values. Consequently, the program does not adhere to synchronization. 
To prevent this issue, the program must include padding operations that are executed instead of the main loop. Furthermore, the loop also needs to be bounded by a classical value, as described in Sec.~\ref{sec:background_controlflow_synchronization}. This results in the algorithm that calculates $x^{\min{(y, max)}}$, as depicted in Fig.~\ref{fig:qcm_sync}. Here, $max$ is a classical bound to the number of loop iterations, as required.

\begin{figure}[htp]
    \centering     
    \lstinputlisting[style=QCM]{../figures/code/background/exp_sync.qcm}
    \caption{A synchronized, reversible exponentiation algorithm.}
    \label{fig:qcm_sync}
\end{figure}

\subsection{OpenQASM Language}
\label{sec:background_qasm}
The Open Quantum Assembly Language (OpenQASM) 3~\cite{CJA*22} is the successor of the OpenQASM 2~\cite{CBSG17} language.
Both languages are imperative and machine-independent quantum languages. They are low-level quantum languages and, thereby, concretely describe a quantum algorithm in the form of a circuit. OpenQASM 2 developed into a de facto standard and is often used as an intermediate language for different quantum tools~\cite{CJA*22}. OpenQASM 3 was developed to fit the changing needs of current quantum research and hardware while being mostly backwards compatible except for some uncommon cases. For example, some keywords were added or changed for the successor such that identifiers of OpenQASM 2 circuits may be invalid in the successor language. Since OpenQASM 3 is the new and improved standard, we will focus on its features in the following section.

OpenQASM 3 requires the header to indicate the language in the circuit header for any top-level circuit. This is achieved by adding \texttt{"OpenQASM 3.0";} to the beginning. Additionally, the language supports the inclusion of other source files, which can be included with the \texttt{include} keyword.

Similar to other quantum languages, OpenQASM operates on two basic data types. The first is the classical bit, while the second is the qubit. Both primitives can also be used in registers with a fixed size. Additionally, OpenQASM 3 also supports further classical data types such as angles and signed and unsigned integers.
In contrast to its predecessor, where any identifiers have to start with a lowercase letter, in OpenQASM 3, identifiers can start with a range of Unicode characters with some exceptions. 

The basic operations of the language can be divided into unitary and non-unitary operations. In OpenQASM 3, all unitary operations are based on the unitary $U(a,b,c)$ where $a,b,c$ are angular parameters.
While OpenQASM 2 supported a controlled-NOT gate natively, the successor requires the gate to be defined with, \eg, the NOT gate and a control modifier. The control modifier can be used to turn any arbitrary unitary gate into a controlled gate with an arbitrary number of control qubits. Therefore, the formerly predefined gate $CX$ must now be defined by the programmer or represented by a NOT gate with a control modifier, \eg \texttt{ctrl @ x}. Lastly, the non-unitary operations are \texttt{measure} and \texttt{reset}. While the \texttt{measure} operation measures the state of a qubit and saves it to a classical bit, the \texttt{reset} operation discards the value of a qubit and replaces it with the $\ket{0}$ state.

The programmer can not only use the operations and modifiers provided by OpenQASM 3 but can also define custom gates. These user-defined gates are defined with an identifier for the gate and a fixed number of single-qubit arguments and angular parameters. In the body of the gate definition, the user can apply a sequence of gates to the qubit arguments with the given angular parameters. Additionally, the language also provides implicit iteration. This means that the application of a single-qubit gate to a quantum register will be interpreted as separate applications of the gate to all qubits in the register.

\begin{figure}[htp]
    \centering     
    \lstinputlisting[style=QASM]{../figures/code/background/example_circuit.qasm}
    \caption{Code for an OpenQASM 3 example circuit.}
    \label{fig:qasm_example}
\end{figure}

In Fig.~\ref{fig:qasm_example}, an example circuit, written in OpenQASM 3, is depicted. The circuit takes two qubits, brings them into an entangled superposition, measures their state, and saves the result to a classical register. In the beginning of the circuit definition, the circuit header indicates the language, and the $X$, $CX$, and $H$ gates are defined based on the predefined unitary $U$. Then, the quantum and classical registers, both with a size of 2, are defined. Next, the Hadamard gate $H$ is applied to the first qubit in the quantum register, followed with the application of a controlled-NOT gate to both qubits. Lastly, the state of both qubits is measured and the result is saved to the classical bits.