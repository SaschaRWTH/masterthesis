\section{Quantum Control Flow}
\label{sec:background_quantumControlFlow}
The idea of quantum control flow was first used by Altenkirch et al.~\cite{AlGr05} when defining a functional programming language with quantum control flow elements. The language uses an if-statement in superposition, $\texttt{if}^\circ$, which is used to, \eg, define the Hadamard gate as a function $had$ instead of a matrix. The $had$ function takes a qubit as an input. If the qubit is true, \ie, it has a value of $\ket{1}$, the function returns a uniform superposition of true and false, where true has a negative sign. Correspondingly, for a false input, a uniform superposition with both signs positive is returned.
\begin{align*}
    had : Q& \to Q\\
    had : x \mapsto& \texttt{ if}^\circ x\\
                   & \texttt{ then } \{false \mid -true\}\\
                   & \texttt{ else } \{false \mid true\}
\end{align*}
Quantum control flow can be divided into \emph{quantum branching} and \emph{iteration}~\cite{YVC24}. In the following, we will discuss both branching and iteration in superposition as well as the limitations of quantum control flow. 

\subsection{Branching}
\label{sec:background_branching}
Based on the work presented by Altenkirch et al.~\cite{AlGr05}, the concept of quantum control flow, more specifically quantum branching, was expanded on and formally defined by Ying et al.~\cite{YYF12}. They introduced two different types of quantum branching: quantum guarded commands and quantum choices as a special case of guarded commands. The definition of quantum guarded commands is based on Dijkstra's guarded commands~\cite{Dijk75}. Guarded commands concern the nondeterministic execution of functions based on Boolean expressions, where the nondeterminism derives from the possible overlapping of the guards. In contrast, quantum branching allows for the execution of functions based on a value in superposition. The functions are executed such that the result may be a superposition of the results of the individual functions~\cite{YVC24}.
Quantum branching is, \eg, used in simulation algorithms like \cite{BGB*18} and \cite{LoCh19}. Furthermore, many basic concepts, such as controlled gates, can be represented as quantum branching or even single-qubit gates, as seen in the previous example of the Hadamard implementation.

The formal definition for classical guarded commands is given by:
\begin{equation*}
    \square^n_{i=1} b_i \to C_i
\end{equation*}
where $C_i$ is a command guarded by a Boolean expression $b_i$. The command can only be executed if the expression is true. Similarly, quantum guarded commands map to a set of quantum programs $P_i$. Further, a set of qubits or quantum registers and a corresponding orthogonal basis $\ket{i}$ is given. However, the set of qubits guarding the program must be disjoint from the set of qubits used in the program. Without this condition, the resulting operation may not be unitary. For example, an X-gate that is executed if the wire it operates on is $1$ always results in a value of $0$; therefore, the operation can be neither reversible nor unitary. The resulting quantum guarded command is of the following form:
\begin{equation*}
    \square^n_{i=1} \bar{q}, \ket{i} \to P_i.
\end{equation*} 
The quantum programs are guarded by the basis states, and the control flow results from the superposition of these basis states~\cite{YYF12}. 

\subsection{Iteration}
\label{sec:background_iteration}
Quantum iteration can be implemented either as quantum recursion or quantum loops. While some languages implement loops based on the measurement of qubits or registers~\cite{Ying11}, the concept of quantum iteration requires the body of the loop to be executed in superposition based on a guard in superposition~\cite{YYF12}.

While classical iteration takes an operation and repeats it on a classical register for $k$ iterations, quantum iteration is dependent on a value $k'$ in superposition and, correspondingly, returns a quantum register in superposition. Moreover, it is a special case of quantum branching and heavily restricted by the limitations of quantum computers~\cite{YVC24}.

\subsection{Limitations}
\label{sec:background_limitations}	
While quantum control flow is often based on the corresponding control flow primitives on classical computers, it is restricted by multiple limitations imposed by quantum computers. Therefore, many control flow primitives that are used in classical programs can either not be used at all or be used in a limited capacity. There are two main limitations for quantum programs. Firstly, all gate-based quantum computers need to adhere to \emph{reversibility}. Secondly, programs need to follow the \emph{synchronization} principle for them to return any useful results~\cite{YVC24}.

\subsubsection{Reversibility}
\label{sec:background_controlflow_reversibility}
As introduced in Sec.~\ref{sec:background_quantumGates}, any sequence of instructions on gate-based quantum computers, excluding measurements, is required to be reversible by definition, as they are all unitary transformations. Therefore, any quantum control flow is also required to adhere to this principle. A resulting limitation, that is not present on classical computers, is that any guards for guarded commands need to be immutable in the commands themselves. For example, if a qubit's state is flipped when its value is $0$, the resulting command will always return a value of $1$. When a program returns the same result regardless of which statements were executed, the program cannot be reversible. This limitation is also inherent in the definition of quantum guarded commands, as described in Sec.~\ref{sec:background_branching}.
Moreover, control flow, as implemented in classical computers, is also not possible. At the most basic software level, modern computers use jump and conditional jump instructions to implement branching and loops. However, any classical jump instruction is inherently irreversible. Not only can a jump go to a section of code that is accessible without any jumps, but multiple jumps can also lead to the same line of code. Therefore, a reversed program cannot know which path was taken in the original program~\cite{YVC24}.

A simple solution seems to be offered by the \emph{Landauer Embedding}~\cite{Land61}. Fundamentally, the idea of the embedding is to turn a non-reversible function into a reversible one by not only returning the output but also the input of the function. For example, for a domain $D$ and a codomain $D'$, any non-reversible function $f : D \to D'$ can be given as a reversible function $g : D \to D' \times D$ with $g(x) = (f(x), x)$. In the case of a quantum program with, \eg, jump instructions based on the Landauer embedding, the output would be the result of the program and a complete history of which path was taken through the program. However, because the quantum data depends on the program history, they become entangled. This leads to disruptive entanglement, as described in Sec.~\ref{sec:background_entanglement}, causing invalid results~\cite{YVC24}.


\subsubsection{Synchronization}
\label{sec:background_controlflow_synchronization}
As previously discussed, reversibility alone is not the only limiting factor on quantum control flow. When handling control flow, similar to the classical implementation, with a program counter in superposition, the program counter can become entangled with the data and result in disruptive entanglement leading to an invalid result. To avoid this issue, the program must not only be reversible but also adhere to the principle of \emph{synchronization}. It states that control flow must become independent from the data. Further, because any quantum program needs to be synchronized to return any useful results, while loops dependent on a value in superposition need to be bounded by a classical value~\cite{YVC24}.