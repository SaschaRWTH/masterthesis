\chapter{Introduction}
\begin{itemize}
    \item Introduction with random citation to not cause error~\cite{ACR*10}
\end{itemize}

% Quantum algorithms like Shor's algorithm~\cite{Shor97} could provide a significant improvement to classical solutions given sufficient technology. Therefore, a lot of research is conducted in the area of Quantum Computing (QC). While there already exist detailed theoretical foundations~\cite{van20, Ying11,YYF12} and advanced algorithms for QC~\cite{ACR*10,BGB*18,LoCh19,Shor97}, the technology of quantum computers is said to be on the level of classical computer in the 1950s~\cite{CFM17}. Currently, there are many architectures and ideas for the future of QC. In this proposal and the following thesis, we want to build upon the Quantum Control Machine proposed by Yuan et al.~\cite{YVC24} and discuss the idea of quantum control flow in general.

% Quantum control flow can be divided into \emph{quantum branching} and \emph{iteration}.~\cite{YVC24} First defined by Ying et al.~\cite{YYF12}, quantum branching is based on Dijkstra's guarded clauses~\cite{Dijk75}. Guarded clauses concern the nondeterministic executing of functions based on boolean expressions. In contrast, quantum branching allows for execution of functions based on a value in superposition such that the result may be a superposition of the results of the individual functions. Quantum branching is, e.g., used in simulation algorithms like \cite{BGB*18}, and \cite{LoCh19}. Extending on the idea of branching, quantum iterations is the repetition of a function based on a value in superposition. 
