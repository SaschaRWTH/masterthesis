\chapter{Introduction}
\label{ch:introduction}
Decades ago, the field of quantum computing emerged as a theoretical combination of quantum physics and computer science. Over the years, quantum algorithms were conceived that exploit the special properties of quantum computers to gain a, in parts exponential~\cite{BeVa93}, speedup over classical implementations. Some of these algorithms have the potential to significantly impact the world of computer science if or when they are implemented on powerful enough quantum hardware. For example, Shor's algorithm~\cite{Shor97} provides a method for efficiently factoring prime numbers. Since many classical cryptographic encryptions rely on the complexity of factoring large primes, Shor's algorithm on powerful quantum computers may break these cryptosystem.~\cite{DiCh20}

However, while the theoretical background of quantum computing is extensive and many algorithms have been proposed, the physical technology of quantum computers is not advanced enough for practical use. In turn, most quantum programming languages are used in the context of simulations, instead of practical applications, and focus on the low-level representation of quantum circuits. One such language is the OpenQASM~\cite{CBSG17} language. OpenQASM programs consist mainly of quantum bits that can be manipulated by quantum gates. Additionally, their state can be measured and the results can be used in limited classical if-statements to control which quantum gates are applied. In contrast, there are also languages with a focus on high level interactions, e.g. Tower\cite{ChMi22} which contains data structures in superposition, and Silq~\cite{BBGV20} which allows for the automatic uncomputation of registers.

What all these languages have in common is the restriction to quantum data while using only classical control flow. The idea of control flow in superposition was first used in a functional quantum programming language proposed by Altenkirch et al.~\cite{AlGr05}. Later on, the concept of quantum control flow was defined by Ying et al.~\cite{YYF12}. Since then, only a few languages have incorporated the principle; only recently, the Quantum Control Machine, an instruction set architecture for quantum computers with quantum control flow at its core, was proposed by Yuan et al.~\cite{YVC24}. 
The instruction set is similar to the instructions provided by classical assembly languages, having jump and condition jump instructions. Additionally, it is specially designed for the limitations of control flow in superposition. However, the low-level design of the architecture and syntax similar to assembly languages combined with general restrictions on quantum programs with control flow in superposition results in complex programs which are both hard to read and write.     

In our project, we want to build upon the ideas of the quantum control machine and control flow in superposition and design a high-level language with control flow primitives in superposition. This abstraction should be similar to the abstraction from classical assembly languages to classical high-level languages where, \eg, classical jump instructions are abstracted to if-clauses or loops. Furthermore, the resulting language should have the same capabilities as the quantum control machine but should be more accessible, \ie easier to read and write programs.
Besides the language design, we also plan to implement a corresponding compiler which compiles our custom language to an existing quantum language. Moreover, the compiler should consist of all the basic elements of a compiler, including the lexical, syntactic, and semantic analysis as well as code generation and optimization.