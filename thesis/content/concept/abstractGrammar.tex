\section{Syntax}
\label{sec:concept_abstractGrammar}
The syntax of our programming language $Luie$ is defined by the context-free grammar $CFG_{Luie}$. It consists of a set of non-terminals $V$, terminals $\Sigma$, and grammar rules $R$ as well as the start symbol $prg$.

\begin{align*}
    CFG_{Luie} = \ & \{V, \Sigma, R, prg \}\\ 
    V = \ & \{ exp, rExp, gate, cGate, stm\\ 
            & \ \ dcl, gDcl, blk, t, prg\}\\ 
    \Sigma = \ & \{ z, id, \texttt{..}, \texttt{range}, \texttt{(}, \texttt{)}, \dots\} 
\end{align*}



\begin{align*}
    \mathbb{N}: \ & n \\
    Identifier: \ & id \\
    Expression: \ & exp ::= n \mid id \mid exp_1 + exp_2 \mid exp_1 - exp_2 \mid exp_1 * exp_2 \mid \dots\\
    RangeExpression: \ & rExp ::= n_1 .. n_2 \mid \texttt{range(} exp \texttt{)} \mid \texttt{range(} exp_1, exp_2 \texttt{)}\\
    Gate: \ & gate ::= id \mid cGate\\
    ConstGate: \ & cGate ::= \texttt{x} \mid \texttt{y} \mid \texttt{z} \mid \texttt{cx} \mid \texttt{ccx}\\
    QubitArgument: \ & qArg ::= id \mid id[exp]\\
    Statement: \ & stm ::= \texttt{qif } id \texttt{ do }  blk \texttt{ end} \mid \texttt{qif } id \texttt{ do }  blk_1 \texttt{ else } blk_2 \texttt{ end}  \\
               & \quad \quad \quad \quad \texttt{for } id \texttt{ in } rExp \texttt{ do } blk \texttt{ end} \mid \\
               & \quad \quad \quad \quad gate \ qArg_1, \dots, qArg_n \texttt{;} \mid \\
               & \quad \quad \quad \quad \texttt{skip;}\\
    Declaration: \ & dcl ::= \texttt{const } id \texttt{ = } exp \texttt{;} \mid \\
                 & \quad \quad \quad \quad \texttt{qubit } id \texttt{;} \mid \\
                 & \quad \quad \quad \quad \texttt{qubit[} exp \texttt{] } id \texttt{;}\\
    GateDeclaration: \ & gDcl::= \texttt{gate } id \texttt{ (}id_1, \dots, id_n\texttt{) do } blk \texttt{ end}\\
    Block: \ & blk::= t_1 \dots t_n \mid \epsilon\\
    Translatable : \ & t::= stm \mid dcl\\
    Program: \ & prg ::= gDcl_1 \dots gDcl_n \ blk \mid blk 
\end{align*}


\begin{align*}
    \mathbb{N}: \ & n \\
    Identifier: \ & id \\
    ConstGate: \ & cGate ::= \texttt{x} \mid \texttt{y} \mid \texttt{z} \mid \texttt{cx} \mid \texttt{ccx}\\
    QubitArgument: \ & a ::= id \mid id[exp]\\
    GateApplication: \ & gateApp::= cGate \ a_1, ..., a_n \texttt{;} \mid\\
    & \quad \quad \quad \texttt{ctrl(}n\texttt{)} \texttt{ @ } qGate \ a_1, ..., a_n \texttt{;} \mid\\
    & \quad \quad \quad \texttt{negctrl(}n\texttt{)} \texttt{ @ } qGate \ a_1, ..., a_n \texttt{;} \mid\\
    & \quad \quad \quad \texttt{ctrl(}n_1\texttt{)} \texttt{ @ } \texttt{negctrl(}n_2\texttt{)} \texttt{ @ } qGate \ a_1, ..., a_n \texttt{;}\\
    QubitDeclaration: \ & qubitDcl::= \texttt{qubit } id \texttt{;} \mid\\
    & \quad \quad \quad \texttt{qubit[}n\texttt{] } id \texttt{;}\\
    QASMStatement: \ & qStm::= gateApp \mid qubitDcl \mid \epsilon\\
    QASM \ & qasm ::= qStm_1 \dots qStm_n\\
    QASMProgam: \ & p ::= \texttt{OPENQASM 3.0;}\\
            & \quad \quad \texttt{include "stdgates.inc";}\\
            & \quad \quad qasm
\end{align*}