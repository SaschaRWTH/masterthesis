\section{Error Handling}
Generally, an important part of a program is error handling; useful and precise error messages are essential for comfortable interactions with the program. This is especially the case for compilers where the user should not only easily understand what the issue is but also where in the source code the error occurred.  

Our compiler has two types of errors with different severities. The first type is the \emph{warning}. A warning from the compiler can indicate issues in the source code that may cause unintended behavior. However, the issue itself does not prevent the compilation of the program and is simply an indication that there may be something wrong. In contrast, the \emph{critical error} is caused by a flaw in the source program that prevents the correct compilation and will result in the abortion of the compilation. In the following, we will discuss the different warnings and critical errors the compiler may raise and their corresponding causes. Furthermore, we discuss why they are either a warning or critical error. 

\subsection{Warnings}
The compiler can throw two different kinds of warnings. The first is the invalid range warning and the second is the unused symbol warning.

An invalid range warning can occur in the context of loop statements. They iterate over a range that is defined by the user. It can be given as either a size $n$ and iterate from $0$ to $n-1$ or a start and end index, $i_{Start}$ and $i_{End}$, respectively, and iterate from the start to the end. However, the range iterator is designed to only increase. Therefore, a range where $i_{Start} \geq i_{End}$ is invalid. Since the loop statement is unrolled at compile time, a range with a size less than or equal to zero can just be ignored. However, the user may not indent this behavior. Therefore, the compiler warns the user that the range is invalid.

The unused symbol warning is raised when a symbol, \eg, a register of composite gate, is defined in the source code but never used. The unused symbol does not have any negative effect on the compilation and the optimization step can easily remove, \eg, an unused register. Therefore, this is only a warning and the program can be compiled. However, an unused symbol may indicate that the wrong symbol was used somewhere else or part of the program is no longer used. Hence, the user is warned of the unused symbol and unintended behavior may be prevented.

\subsection{Critical Errors}
The list of critical errors, often just referred to as errors, is more extensive than the list of warnings. It includes the invalid access, number of arguments for gates and functions, and size errors. Furthermore, there are errors for the attempted declaration of a variable that is already declared, an undeclared and type error, and, lastly, an error for the invalid use of a qubit in a guarded code block.

The first error is the invalid access error. It occurs when a register is accessed at an invalid index $i$, \ie, $i$ is either smaller than zero or larger than $size - 1$, where $size$ is the size of the register. While this error could easily be ignored and would cause no issue when compiling the program, the resulting code would be an invalid circuit description. 

Secondly, the invalid number of arguments error for gates and functions is caused when the number of arguments given to either a gate or function does not correspond to the number of required arguments. For example, the Hadamard gate always expects one argument while the controlled-not gate requires two. Similarly, the \texttt{sizeof}-function operates on only one argument. The compiler cannot proceed when given too few arguments; for the opposite case, while dropping any leftover arguments is possible, it would result in unexpected behavior. % Additionally, too many arguments indicate an issue in the program. 
Therefore, the compiler reports an error for both cases and aborts the compilation.

Another error is the invalid size error. It occurs then a register is declared with an invalid size. A size is invalid if it is less or equal to $0$. A register with non entries cannot be used for anything and, likely, indicates an issue in the program while a register with a negative amount of entries is impossible. Therefore the compilation is aborted and the error is thrown.

The next two errors are concerned with the declaration of variables in a given context; they are the undeclared and already-declared errors. An undeclared error is raised when a variable is used in a context where it is not defined. In this case, the symbol table does not have a symbol stored for the given identifier and the compiler cannot continue. In contrast, the already-declared error occurs when a variable is declared in a context where the same variable identifier has already been assigned to a different symbol. While the compiler could overwrite the previous declaration, this can easily lead to unexpected behavior and, in turn, we do not allow a declaration in the same scope to be overwritten. 

The type error is thrown when a variable is used in a function or gate but is not the required type. Languages with loose typing may be able to convert some types to the required type by, for example, parsing the integer value of a string. However, this can not only result in unexpected behavior and hard to debug errors in the code but, in the case of a quantum language, it may also require the conversion between classical and quantum data which is not easily achievable. 

Finally, the last error is the use-of-guard error; it occurs when a qubit is referenced in a context that is guarded by itself. While the compiler can easily translate any such occurrence, they result in a invalid circuit description. As described in Sec.~\ref{sec:background_branching}, a gate that operates on and is controlled by the same qubit cannot be reversible. Therefore, the compiler prevents the generation of an invalid circuit and aborts the compilation. 