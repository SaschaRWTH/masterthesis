\section{Error Handling}
Generally, an important part of a program is error handling; useful and precise error messages are essential for comfortable interactions with the program. This is especially the case for compilers where the user should not only easily understand what the issue is but also where in the source code the error occurred.  

Our compiler has two types of errors with different severity. The first type is the warning. A warning from the compiler can indicate issues in the source code that may cause unintended behavior. However, the issues itself does not prevent the compilation of the program and is simply an indication that there my be something wrong. In contrast, the critical error is cause by a flaw in the source program that prevents the correct compilation. In the following, we will discuss the different warnings and critical errors, the compiler my raise and what their meaning is.

\subsection{Warnings}
The compiler can throw two different kinds of warnings. The first is the invalid range warning and the second is the unused symbol warning.

An invalid range warning can occur in the context of for loops. They iterate over a range that is defined by the user. It can be given as either a size $n$ and iterate from $0$ to $n-1$ or a start and end index, $i_{Start}$ and $i_{End}$ respectively, and iterate from the start to the end. However, the range iterator is designed to only increase. Therefore a range where $i_{Start} \geq i_{End}$ is invalid. Since the for loop is unrolled at compile time, a range with a size less than or equal to zero can just be ignored. This however may be unintended behavior. Therefore, the compiler warns the user that the range is invalid.

The unused symbol warning is raised when a symbol, \eg a register of composite gate, is defined in the source code but never used. The unused symbol does not have any negative effect on the compilation and the optimization step can easily remove, \eg, an unused register. Therefore, this is only a warning and the program an be compiled. However, an unused symbol may indicate, that the wrong symbol was used somewhere else or part of the program is no longer used, hence, warning the user of the unused symbol may prevent untended behavior.

\subsection{Critical Errors}
The list of critical errors, often just referred to as errors, is more extensive than the list of warnings. It includes the invalid access, number of arguments for gates and functions, and size errors. Furthermore, there are errors for the attempted declaration of a variable that is already declared, an undeclared and type error, and, lastly, an error for the invalid use of a qubits in a guarded code block.

The first error ist he invalid access error. It occurs when a register is accessed an index $i$, \ie $i$ is either smaller than $0$ or larger than $size - 1$, where $size$ is the size of the register. While this error could easily be ignored and would cause no issue when compiling the program, the resulting code would be an invalid circuit description. 

Secondly, the invalid number of arguments error for gates and functions is caused when the number of arguments given to either a gate or function is not as expected. For example, an the Hadamard gate, when not controlled, always expects one argument while the controlled-not gate requires two. Similarly, the \texttt{sizeof}-function operates one expertly one argument. The compiler cannot proceed when given to few arguments; for the opposite case, dropping any leftover arguments would result in an unexpected behavior. Additionally, too many arguments indicate an issue in the program. Therefore, the compiler reports an error for both cases and aborts the compilation.

invalid size

redeclaration error
undeclared error

type error

(invalid) use of guard error 