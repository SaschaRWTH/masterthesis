\section{Command Line Interface}
The command line interface of the compiler enables the programmer to interact with the compiler and specify specific behavior. There are currently four different parameters that can be specified, the input file, output file, optimizations to be applied, and the verbosity of the compilation. 

The first, and most important, parameter is the input file. Since without a specified input file the compiler has no program to compile, the parameter is mandatory. It can be given by either \texttt{-i} or \texttt{--input} followed by the path to the input file. Similarly, the output file is specified with another parameter; it can be specified with \texttt{-o} or \texttt{--output} followed by the path to the output file.  However, this parameter is not required and the default behavior is to create a file with the name ``output.qasm'' in the current directory.

Next, the optimization parameter can be used to specify which optimizations are applied to the code. Here, the possibilities are either none, the null gate, peeping control, Hadamard reduction, and control reversal optimization rules. Each optimization has a specific keyword that can be passed as a parameter to indicate the optimization. Additionally, multiple different optimizations can be applied by listing them separated by a vertical bar.
\info{currently, spaces are NOT allowed (should be implemented for ease of use)}
For example, to apply both the null gate and peeping control optimizations, the correct parameter would be ``nullgate|peepingcontrol''. The optimization parameter is specified by either passing \texttt{-O}, in this case an uppercase ``o'' to differentiate it from the output, or \texttt{--optimization}. When the optimization is not explicitly specified, the default behavior is no optimization. 

Another parameter that can be based to the compiler is the verbose parameter; it toggles the verbose mode of the compiler on. When this mode is active, the compiler prints info, warning, and error messages out to the used. In comparison, only error messages are displayed to the programmer when the compiler is executed without the verbose mode. In contrast to the other parameters, the verbose mode does not require any other arguments and can be activated by passing either \texttt{-v} or \texttt{--verbose} to the compiler. If the parameter is not passed, the compiler is executed normally. 

Besides the parameters that can be passed to the compiler to modify its behavior, it also provides a help test. The help text can be access by passing the help parameter to the compiler with either \texttt{-h} or \texttt{--help}. Additionally, it is displayed when an invalid or unknown parameter was given by the programmer together with a message that indicates that the given parameter is invalid. The help text itself gives a short description of the compiler and lists all parameters that can be passed. Furthermore, each parameter is listed with a short description of its behavior.

An example ...
\begin{figure}[htp]
    \centering
    % , label=fig:concept_cli_example, caption={A command line interface example.}]
    \begin{lstlisting}[language=bash, style=bashstyle]
./LUIECompiler --input "./program.luie" 
               --output "build/program.qasm" 
               --optimization nullgate|peepingcontrol
    \end{lstlisting}
    \caption{A command line interface example.}
    \label{fig:concept_cli_example}
\end{figure}
\todo{improve figure/lstlisting display}