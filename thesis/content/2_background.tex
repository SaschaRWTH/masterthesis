\chapter{Background}
... Background

\section{Quantum Computing}
\begin{itemize}
    \item Introduction into quantum computing
    \item From bits to qubits
    \item Entanglement
\end{itemize}

\subsection{Entanglement}
\begin{itemize}
    \item Explain entanglement
    \item How is entanglement relevant for quantum computing?
    \item Explain disruptive entanglement
\end{itemize}

\subsection{Operators and Gates}
\begin{itemize}
    \item General theoretical bases of operators and gates
    \item Most important gates and their functions
\end{itemize}

\subsection{Measurement}
\begin{itemize}
    \item How are qubits measured?
    \item What is the effect of measurement on qubits?
\end{itemize}

\subsection{Relevant Algorithms}


\section{Quantum Control Flow}
\begin{itemize}
    \item Introduction into quantum control flow
    \item Branching
    \item Iteration
    \item Limitations
\end{itemize}

\subsection{Branching}
\begin{itemize}
    \item Explain branching principle
    \item How is branching relevant for control flow?
    \begin{itemize}
        \item Example of branching in classical computing
        \item Example of branching in quantum computing
    \end{itemize}
\end{itemize}

\subsection{Iteration}

\subsection{Limitations}	

\subsubsection{Reversibility}
\begin{itemize}
    \item Explain reversibility principle
    \item How is reversibility relevant for control flow?
    \begin{itemize}
        \item Example of reversible and irreversible operations (JMP instructions)
    \end{itemize} 
    \item 
\end{itemize}
\subsubsection{Synchronization}
\begin{itemize}
    \item Explain synchronization principle
    \item Tortoise and hare example
\end{itemize}

\subsection{Quantum Control Machine}
\begin{itemize}
    \item Definition of Quantum Control Machine
    \item How does it solve/handle the limitations of quantum control flow?
    \item Example program
    \begin{itemize}
        \item Example for non-reversible program
        \item Example for reversible but not synchronous program
        \item Example for correct program
    \end{itemize}
\end{itemize}


\begin{figure}[htp]
    \centering     
    \begin{minipage}{.40\textwidth}
        \vspace{7.5em}
        \begin{lstlisting}[linewidth=17em,style=QCM]
    add   res $1
    add   r1  y
l1: rjne  l3  r1  y
l2: jz    l4  r1
    mul   res x
    radd  r1  $1
l3: jmp   l1  
l4: rjmp  l2      
        \end{lstlisting}
        \caption{QCM exponentiation without synchronization}
        \label{fig:qcm_not_sync}
    \end{minipage}
    \hfill
    \begin{minipage}{.55\textwidth}
        \begin{lstlisting}[linewidth=23em,style=QCM]
    add   res $1
    add   r1  max
l1: rjne  l3  r1  max
l2: jz    l4  r1
l5: jg    l7  r1  y   
    mul   res x
l6: jmp   l8  
l7: rjmp  l5
    nop           ; padding
l8: rjle  l6  r1  y
    radd  r1  $1
l3: jmp   l1
l4: rjmp  l2
        \end{lstlisting}*
        \caption{Synchronized QCM exponentiation}    
        \label{fig:qcm_sync}
    \end{minipage}
\end{figure}

\section{Quantum Languages}

\subsection{QASM Language}
\begin{itemize}
    \item Give overview of QASM language and concepts
\end{itemize}

\section{Compilation}

\subsection{Lexer}

\subsection{Parser}

\subsection{Semantic Analysis}

\subsection{Code Generation}

\subsection{Optimization}
\begin{itemize}
    \item Different optimization techniques
    \begin{itemize}
        \item Constant folding or constant propagation
        \item Peephole optimization
    \end{itemize}
\end{itemize}

\subsection{ANTLR}
\begin{itemize}
    \item Give overview of ANTLR and parsing in general
\end{itemize}