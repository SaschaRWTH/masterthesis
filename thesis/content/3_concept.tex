\chapter{Concept}
After the discussion of the background of quantum computing in general, quantum control flow and programming language, and the compilation of programs, in this chapter, we present a concept for the quantum programming language ``Luie'' with control flow primitives, based on the ideas of the Quantum Control Machine. After an informal description, we define both the syntax of the language and translation functions from the source language to the target language OpenQASM.  
% Next, we discuss some general concepts for the compiler, starting with the error handling. There, we introduce and explain all possible warnings and critical errors that can occur. The error handling is followed by the discussion of the optimizations that the compiler may apply to the code; this includes some inherent optimizations as well as optional peephole optimizations. 
Next, we present the different optimizations that are applied by the compiler. Additionally, we give a theoretical overview of the graphical abstraction that is used to optimize the program.
Lastly, we discuss the command line interface of the compiler, its options and their corresponding behavior.   

\section{Language Overview}
In the following section, we discuss the different concepts that the proposed language provides for writing quantum programs. Furthermore, we discuss their behavior and some use cases for them. Additionally, we explain why some features of other languages were not incorporated and the reasons for some special behaviors. 

The idea for our proposed language is to provide a high-level language with the capabilities of the Quantum Control Machine (QCM). For this, we want to remove low-level concepts, such as jump instructions, and add high-level ones, such as code blocks with variable contexts and additional data types. Additionally, since jump instructions in superposition are removed, we need to add other control flow statements so that the language is as expressive as the QCM. For this, we introduce two basic control flow statements: the loop statement, which is unrolled at compile time, and the quantum if- and else-statements. Based on these ideas, we propose the loop unrolled, if-else, programming language ``Luie''.

Firstly, we discuss code blocks and their corresponding variable contexts, or scopes. Then, we introduce the data types that are available in the language. Next, the basic operations as well as the more complex control flow statements are listed and explained. This is followed by an overview of the possible expressions and the available operations and built-in functions. Lastly, we discuss the composite gates of the language. 

%The following description is a high level overview of the concepts and functionalities of the language. While we discuss a high-level overview in the following, the concrete syntax and implementation of these features is discussed in Ch.~\ref{ch:implementation}.

\subsection{Blocks and Scopes}
\label{sec:concept_blocksAndScope}
Similar to many other languages, Luie uses code blocks and corresponding scopes. Both the code blocks and scopes are used to structure the code and enable the reuse of identifiers in different contexts. Code blocks and scopes are hierarchically structured; the parent of a code block is the block that directly contains this scope, while ancestor blocks are all blocks that contain the block.
In contrast, the descendants of a block are all blocks that are directly and indirectly contained in the block. Furthermore, two blocks are independent if one is neither an ancestor nor a descendant of the other. Since each scope corresponds to a code block, the same terms can be used for them.

Each Luie program is contained in the main code block. This main code block can contain arbitrarily many other code blocks. In turn, these code blocks can also contain any number of nested blocks. However, the main code block can only exist once and is the ancestor of all others. Therefore, it also cannot depend on any other block. The main block not only differs from the others in terms of hierarchy, but it is also the only block that can contain composite gate declarations. The composite gates are declared at the top of the main code block and are followed by other declarations or statements. 

Similar to the main code block, all other code blocks also consist of declarations and statements. A new code block is defined either in the declaration of a composite gate or in the body of a control flow primitive; this includes the if-statement, else-statement, and loop statement. Additionally, each code block has a unique scope.

Scopes represent the variable context of a code block. In a given scope, the program can access all variables previously defined in this scope and all its ancestors; however, it cannot access the variables of any descendants.
In contrast, two independent scopes do not have access to the variables defined in the other scope. Therefore, two independent if-statements can define a variable with the same name in their scope. 
% Additionally, a scope can also overwrite the definition of a variable of an ancestor with its own definition while the ancestor's definition is not affected.

\subsection{Data Types}
\label{sec:concept_dataTypes}
Luie is mainly focused on being a quantum language with quantum control flow; this is also reflected in its data types. The language operates mostly on two types of quantum data, registers and qubits. Both behave as described in Sec.~\ref{sec:background_quantumComputing}, where a register represents an array of qubits with a fixed length.
Both registers and qubits are declared similarly. First, the \texttt{qubit} keyword is used to indicate the declaration. In the case of a register, the size of the register is given in brackets. Lastly, the identifier for the register or qubit is specified. For data types in most classical languages, the variables need to be explicitly initialized. In quantum computers, however, qubits can only be initialized to $\ket{0}$. Therefore, Luie does not require or allow for an initialization value for qubits and registers but always defaults to $\ket{0}$ for all qubits and register entries.

The second class of data types in Luie are classical data types. They can be divided into constant values, named constants, and loop iterators. In contrast to most classical languages with similar data types, they are not referenced in the compiled program; they are only used at compile time and, therefore, must all have a constant value or a known behavior. For example, while the value of an integer constant is constant, the value of a loop iterator changes with each iteration but is constant in each.
Constant values are the most simple data types and represent any value given directly in the program code. 

Similar to constant values, named constants, often just called constants, also have a constant value that is saved as an expression when they are declared and evaluated when the code is generated. However, they can not only be integers but also unsigned integers and floating point values. Furthermore, they can be given by an expression when declared and referenced as often as needed in the context they were declared in. Constants are declared with the \texttt{const} keyword and need to be given an identifier as well as the type of the constant. The type is either the \texttt{int}, \texttt{uint}, or \texttt{double} keyword. Lastly, an expression for the value of the constant is given.

The loop iterator data type is declared when creating a loop and is also only accessible in the loop body. Its value corresponds to the current loop iteration, \eg for the first iteration zero and the last one $n - 1$, where $n$ is the number of iterations. Loop iterators are usually not implemented as a special data type but by incrementing, \eg, an integer; however, Luie does not allow mutable, classical data type and, therefore, cannot increment an integer to iterate over a loop. In turn, it requires a special data type whose value is constant for each iteration.

\subsection{Basic Operations}
\label{sec:concept_basicOperations}
Any language needs some basic operations to manipulate its data types. While these operations may be addition or multiplication on classical computers, quantum computers manipulate their data with gates. Luie provides multiple predefined gates with which the qubits and registers can be manipulated. The syntax for applying gates to qubits is similar to other quantum languages; the name of the gate is given, and all parameters are listed next, separated with commas.

The first kind of gate set natively available consists of basic single-qubit gates; these include the $X$, $Y$, and $Z$ gates as well as the Hadamard gate $H$. The logical extension of single-qubit gates is the second gate set, the multi-qubit gates; they include the controlled-not gate $CX$ and the Toffoli gate $CCX$. These gates are not required for the overall gate set to be universal because they can easily be simulated with the control flow statements available in the language. However, they are included because most programmers are used to them; internally, they are translated to controlled $X$ gates.
Finally, the parameterized gates are the last gate set. In contrast to the basic single- and multi-qubit gates, they are not constant in their behavior but depend on one or multiple parameters. These parameters are given as expressions and are evaluated at compile time; they are given in parentheses after the gate and separated by commas. Currently, the only parameterized gate available is the phase gate $P(\lambda)$.
The respective behavior and mathematical definition of all predefined gates in Luie are discussed in Sec.~\ref{sec:background_quantumGates}. Since the language provides both the Toffoli and Hadamard gates, the overall gate set is universal. Furthermore, any gates that are not predefined can easily be implemented as composite gates; they are described in Sec.~\ref{sec:concept_compositeGates}.

However, one feature not provided for gate applications is implicit iteration. Implicit iteration allows gates that operate on a single qubit to be applied to an entire register instead by giving the register as an argument. Since Luie provides loop statements with which a gate can explicitly be applied to an entire register with additional control over the application, implicit iteration is not implemented in the language.

\subsection{Measurements}
\label{sec:concept_measurement}
For a quantum circuit to be of any use, measurements are required. However, there are many options for implementing measurements. Our language only implements implicit measurements and does not allow the explicit measurement of a qubit at an arbitrary point in the circuit. While explicit measurement may allow for more flexibility and manual optimization of qubit usage, both approaches can implement the same set of algorithms. Furthermore, our language focuses mainly on control flow statements, so we decided on a simple approach for the measurement of qubits. Additionally, explicit measurement statements in the language would require further classical data types to store the result of the measurement; this would increase the overall complexity of the language and require the user to add additional declarations for each explicit measurement.

In our language, the measurements are implicitly added at the end of each qubit wire. Thus, the compiler declares both a bit for each qubit and a classical register for each quantum register. Next, each qubit and quantum register is measured, and the result of the measurement is saved to the corresponding bit or classical register.

\subsection{Control Flow}
\label{sec:concept_controlFlow}
The main focus of the proposed language is quantum control flow. Therefore, Luie provides two different control flow statements: a loop statement with a fixed number of iterations and an if-statement operating on data in superposition. 

The loop statement can be used to iterate over a code block a fixed number of times. While the number of iterations needs to be known at compile time, it can depend on constants known at compile time, such as the size of a register. However, it can, \eg, not depend on the measurement of a qubit or the value of a qubit in superposition. This relates back to the principle of synchronization, as described in Sec.~\ref{sec:background_controlflow_synchronization}. Additionally, the body of the loop can depend on the iteration index, \ie the current value of the iterator. Therefore, the loop statement can, \eg, iterate over a register or part of it and apply a series of gates to each qubit. Furthermore, the loop body can also include all statements and declarations available for each code block. In turn, the loop statement can contain nested loops, or parts of its body can be guarded by an if-statement. However, the target language of the compiler and quantum computers in general do not have a concept of local variables. Therefore, while declarations can be performed, a declaration in a loop body is translated to a ``global'' declaration of a qubit for each iteration. This can significantly increase the qubit count of the program. To optimize the current implementation, a single qubit could be declared that is reset or measured at the end of each iteration, thereby reducing the required number of qubits. However, this optimization is currently not implemented.

In contrast to loop statements, the if-statements do not depend on constant data known at compile time; their execution, however, depends on the value of a qubit in superposition. Further, the guard of the statement can only be a single qubit, not a register, and cannot include boolean expressions.
The body of the statement can contain any other statements, including further if- and loop statements, as well as declarations. When compiled, each gate application inside the body is guarded by the given qubit and translated to a controlled version of the gate where the guard is the control. However, the declaration is not dependent on the value of the guard. Therefore, similar to declarations inside loop statements, the declaration is always performed independent of the guard's value. In turn, the declaration of qubits or registers inside an if-statement can increase the qubit count. However, since it is only performed once and not for each iteration, the gate count is not affected to the same extent as the declaration in loop statements. Additionally, besides the main code block, the if-statement also offers an optional else case. In this case, the translated statements of the else-block are controlled by the negated guard. 

\subsection{Expressions}
\label{sec:concept_expressions}
The language allows for complex expressions that are evaluated at compile time. These expressions can be used to access specific indices of a register or define the range of an iterator used by a loop statement. Besides the typical operations like addition and multiplication, the language also implements different functions. However, the targeted quantum computers should not need to and, in some cases, cannot evaluate these expressions; their values need to be not only constant but also evaluated at compile time. Furthermore, they can neither depend on the measurement nor the value of a qubit.

The first kind of operations are the basic operations; they include addition and subtraction, multiplication and division, and the negation. The grammar of the language is designed such that the usual operator precedences are respected. Additionally, parentheses can be used to adjust this order. The operations operate either on an integer value or an identifier. If an integer value is given, it can easily be parsed to an integer and used with the operations. However, the identifier can either be a constant or an iterator. For either, the compiler looks up the current value in the symbol table and evaluates the remaining expression. 

Besides basic operations, the language provides more complex functions that can be used for specific calculations. Firstly, the \texttt{sizeof} function takes a register as an input and returns the size of the given register. This can be useful for creating a loop statement that iterates over the size of a register. Furthermore, it can be used to create a register based on the size of another. Next, the \texttt{power} function takes two expressions as arguments, evaluates them, and raises the first argument's value to the value of the second argument. Lastly, similar to the \texttt{power} function, the \texttt{min} and \texttt{max} functions both take two expressions as arguments and return the minimum and maximum value of the evaluated expressions, respectively.

Luie provides two different expressions for creating ranges. Since they create ranges, or more specifically loop iterators for a given range, and the other operations only operate on basic numeric types, \eg integers and floating point values, they are not grouped together with the others but implemented separately. The first way to define a range is to give the start and end values separated by two dots, \eg $1\texttt{..}9$ for iterating from one to nine. In this case, both the start and end values are included in the iteration. However, this expression can only take integers for the start and end values, not expressions; this limitation is imposed such that the expression remains readable. The other operation for creating ranges is the \texttt{range} keyword followed by the start and end values in parenthesis and separated by a comma, \eg $\texttt{range}(1, 9)$. For this expression, the parameters can also be given as expressions. Furthermore, the \texttt{range} expression also provides a shorthand, where only the length $len$ of the range is given. In this case, it will always start from zero and iterate to $len - 1$. This is most useful for creating ranges to iterate over registers. 

Lastly, in contrast to many other languages, the access of a register is not implemented as an expression. A register access, \ie a qubit, is not a classical data type and cannot be used in any context other than as a gate parameter or as an if-statement guard. Therefore, it is implemented as a separate grammar rule and not as an expression.

\subsection{Composite Gates}
\label{sec:concept_compositeGates}
Similar to OpenQASM, Luie also allows for the declaration of composite gates. They can be used to declare a custom gate that applies the gate combination specified in the gate body. This can be particularly useful for gate combinations often used in the code to reduce redundancy and, furthermore, improve the readability of the code. For example, the swap gate can be defined on two qubits where three controlled-not gates are applied such that the values of both qubits are swapped. In this case, not only are three gates reduced to only one, but the code clearly indicated that the values are swapped without prior knowledge of the effect of the controlled-not gates.

While the concept of a composite gate is also available in OpenQASM, Luie expands on the possibilities provided by OpenQASM's implementation. Firstly, the gate declaration does not only allow for simple gate application statements but also for the control flow statements provided by Luie. This includes the loop statements and if-statements in superposition; only qubit and register declarations are prohibited in the body of gate declarations. In contrast to composite gates in OpenQASM, not only qubits are allowed as arguments but also registers. Therefore, registers can be given as arguments, and the loop statement can iterate over them, resulting in gates that can depend on the length of a register and do not need to be reimplemented for each different size. However, in contrast to OpenQASM, the language currently does not allow for parameterized composite gates.

\begin{figure}[htp]
    \centering     
    \lstinputlisting[style=Luie]{../figures/code/concept/qft.luie}
    \caption{Luie gate definition for the Quantum Fourier Transform.}
    \label{fig:qft_example}
\end{figure}

Two examples of composite gates are depicted in Fig.~\ref{fig:qft_example}. The first gate is a simple definition for the commonly used, and often predefined, \texttt{swap} gate. Below, a gate definition for the quantum Fourier transform is given. It not only references the previous swap gate but also uses both if- and loop statements. Firstly, the gate iterates over all qubits in the given register $reg$ with size $n$. In each iteration, the gate first applies the Hadamard gate to the qubits and, again, iterates over all remaining qubits in the register and applies the phase gate depending on the value of the qubit in superposition. Additionally, the parameter for the phase gate is calculated based on the offset of both indices. After iterating through the entire register and applying the corresponding gate, the gate swaps each qubit at index $i$ with the qubit at index $n - i$ for $0 \leq i \leq \floor*{\frac{n}{2}}$.

\section{Syntax}
\label{sec:concept_abstractGrammar}
The syntax of our programming language $Luie$ is defined by the context-free grammar $CFG_{Luie}$. It consists of a set of non-terminals $V_{Luie}$, terminals $\Sigma_{Luie}$, and grammar rules $R_{Luie}$ as well as the start symbol $prg_{Luie}$. The terminal symbol $n$ represents a natural number, while $id$ represents an identifier. The grammar rules $R_{Luie}$ are discussed in the following.
\begin{align*}
    CFG_{Luie} = \ & \{V_{Luie}, \Sigma_{Luie}, R_{Luie}, prg \}\\ 
    V_{Luie} = \ & \{ exp, rExp, gate, cGate, qArg stm,\\ 
            & \ \  dcl, gDcl, blk, t, prg\}\\ 
    \Sigma_{Luie} = \ & \{n, id, \texttt{..}, \texttt{range}, \texttt{(}, \texttt{)}, \texttt{+}, \texttt{-}, \\
               & \ \ \texttt{x}, \texttt{y}, \texttt{z}, \texttt{cx}, \texttt{ccx}, \texttt{[}, \texttt{]}, \dots \} \quad \quad \texttt{where } n \in \mathbb{N}, id \in Identifier
\end{align*}

The grammar contains two different kinds of expression rules, the arithmetic expression rule $exp$ and the range expression rule $rExp$.
\begin{align*}
    Expression: \ & exp ::= n \mid id \mid exp_1 \texttt{ + } exp_2 \mid exp_1 \texttt{ - } exp_2 \mid exp_1 \texttt{ * } exp_2 \mid \dots\\
    RangeExpression: \ & rExp ::= n_1 .. n_2 \mid \texttt{range(} exp \texttt{)} \mid \texttt{range(} exp_1, exp_2 \texttt{)}\\
\end{align*}

Three grammar rules that are used in the context of gate applications are the gate, constant gate, and qubit argument rules, $gate$, $cGate$, and $qArg$ respectively. The gate rule allows for the application of either a composite gate, given by an identifier, or a constant gate, given explicitly. Similarly, the qubit argument rule allows for either a qubit, specified by an identifier, or a register access, specified by an identifier and expression. Lastly, the constant gate rule simply allows for all predefined gates. 
\begin{align*}
    Gate: \ & gate ::= id \mid cGate\\
    ConstGate: \ & cGate ::= \texttt{x} \mid \texttt{y} \mid \texttt{z} \mid \texttt{cx} \mid \texttt{ccx}\\
    QubitArgument: \ & qArg ::= id \mid id[exp]\\
\end{align*}

Next, the declaration rules $dcl$ and $gDcl$ are used to declare qubits, register, or constant variables and gate declaration respectively. Each requires an identifier given by the $id$ rule. In the case of the constant and register declaration, the value and size are given by an expression. Additionally, the gate declaration requires a list of identifiers for the arguments and the code block of the composite gate.
\begin{align*}
    Declaration: \ & dcl ::= \texttt{const } id \texttt{ = } exp \texttt{;} \mid \\
                 & \quad \quad \quad \quad \texttt{qubit } id \texttt{;} \mid \\
                 & \quad \quad \quad \quad \texttt{qubit[} exp \texttt{] } id \texttt{;}\\
    GateDeclaration: \ & gDcl::= \texttt{gate } id \texttt{ (}id_1, \dots, id_n\texttt{) do } blk \texttt{ end}\\
\end{align*}

The statement rule $stm$ can either be an if-statement, with an optional else-block, a loop statement, a gate application statement, or a skip statement. The qubit argument rule is used for both the if-statement, to specify the control qubit, and the gate application statement, to specify the arguments. Additionally, the range expression is used to specify the range of the loop-statement. The block rule is used for the body of the if-statement, the optional else-block, and the loop statement.
\begin{align*}
    Statement: \ & stm ::= \texttt{qif } qArg \texttt{ do }  blk \texttt{ end} \mid\\
                 & \quad \quad \quad \quad \texttt{qif } qArg \texttt{ do }  blk_1 \texttt{ else } blk_2 \texttt{ end} \mid\\
                 & \quad \quad \quad \quad \texttt{for } id \texttt{ in } rExp \texttt{ do } blk \texttt{ end} \mid \\
                 & \quad \quad \quad \quad gate \ qArg_1, \dots, qArg_n \texttt{;} \mid \\
                 & \quad \quad \quad \quad \texttt{skip;}\\
\end{align*}

The last grammar rules are the block rule $blk$, which specifies a possibly empty list of translatables, the translatable rule $t$, which can either be a statement or declaration, and the program rule $prg$, which consists of an optional list of gate declarations and a code block. 
\begin{align*}
    Block: \ & blk::= t_1 \dots t_n \mid \epsilon\\
    Translatable : \ & t::= stm \mid dcl\\
    Program: \ & prg ::= gDcl_1 \dots gDcl_n \ blk \mid blk 
\end{align*}

Since we define some auxiliary functions that operate on OpenQASM code, a reduced grammar for the OpenQASM programming language is given by $CFG_{QASM}$. It consists of a set of non-terminals $V_{QASM}$, terminals $\Sigma_{QASM}$, and grammar rules $R_{QASM}$ as well as the start symbol $prg_{QASM}$. Some of the terminals and non-terminals in the grammar are the same as in the source code grammar $CFG_{Luie}$. In these cases, the definition of the grammar rules also the same; all other grammar rules are discussed in the following.
\begin{align*}
    CFG_{QASM} = \ & \{V_{QASM}, \Sigma, R_{QASM}, p \}\\ 
    V_{QASM} = \ & \{ cGate, qArg, gateApp, qubitDcl, qStm,\\ 
            & \ \  qasm, p \}\\ 
    \Sigma_{QASM} = \ & \{n, id, \texttt{ctrl}, \texttt{negctrl}, \texttt{(}, \texttt{)}, \\
               & \ \ \texttt{x}, \texttt{y}, \texttt{z}, \texttt{cx}, \texttt{ccx}, \texttt{[}, \texttt{]}, \dots \} \quad \quad \texttt{where } n \in \mathbb{N}, id \in Identifier
\end{align*}

\dots
\begin{align*}
    QubitDeclaration: \ & qubitDcl::= \texttt{qubit } id \texttt{;} \mid\\
    & \quad \quad \quad \quad \quad \texttt{qubit[}n\texttt{] } id \texttt{;}\\
\end{align*}

\begin{align*}
GateApplication: \ & gateApp::= cGate \ a_1, ..., a_n \texttt{;} \mid\\
& \quad \quad \quad \texttt{ctrl(}n\texttt{)} \texttt{ @ } qGate \ a_1, ..., a_n \texttt{;} \mid\\
& \quad \quad \quad \texttt{negctrl(}n\texttt{)} \texttt{ @ } qGate \ a_1, ..., a_n \texttt{;} \mid\\
    & \quad \quad \quad \texttt{ctrl(}n_1\texttt{)} \texttt{ @ } \texttt{negctrl(}n_2\texttt{)} \texttt{ @ } qGate \ a_1, ..., a_n \texttt{;}\\
\end{align*}

\begin{align*}
    QASMStatement: \ & qStm::= gateApp \mid qubitDcl \mid \epsilon\\
    QASM: \ & qasm ::= qStm_1 \dots qStm_n\\
    QASMProgam: \ & p ::= \texttt{OPENQASM 3.0;}\\
            & \quad \quad \texttt{include "stdgates.inc";}\\
            & \quad \quad qasm
\end{align*}

\section{Translation}
\label{sec:concept_translation}
In the following, we discuss a formal translation of the source code, given in the form of a Luie program $prg$, to the target code, in the form of a OpenQASM program $p$. 

An important part of the translation is the symbol table. It is used to propagate symbol information throughout the translation. A symbol tables is a function that maps either an identifier to the information of the corresponding symbol or the \texttt{uid} keyword to a number, the unique identifier counter, that is used to create unique identifiers for the target code variables.
\begin{align*}
    SymbolTable := \ & \{st \mid st : Identifier \dashrightarrow & (\{\texttt{const}\} \times \mathbb{R})\\
    & \cup& (\{\texttt{qubit}\} \times \mathbb{N} \times Identifier)\\
    & \cup& (\{\texttt{arg}\} \times QubitArgument)\\
    & \cup& (\{\texttt{gate}\} \times Block \times Identifier^+)
    \} \\
   \cup \ & \{st \mid st : \{\texttt{uid}\} \dashrightarrow & (\mathbb{N})\}
\end{align*}
The symbol information, contained in the symbol tables, varies depending on the four different kinds of symbols. The fist is the constant symbol which is identified by the \texttt{const} keyword, together with the value of the constant symbol given by a real number. Next, the \texttt{qubit} keyword identifies the symbol information of a register; it contains the size of the register as a natural number, where a size of one indicates a qubit, and the unique identifier for the register. Next, the argument symbol information saves the qubit argument to which a composite gate argument maps; it is identified by the \texttt{arg} keyword. Lastly, identified by the \texttt{gate} keyword, the gate symbol information saves the code block of a composite gates as well as list of argument identifiers with at least one element. 

At the root of the translation is the translation function $trans$. It maps a source code program $prg$ to the corresponding OpenQASM program $p$. Firstly, the header information for the program is added. Next, the initial symbol tables $st_\epsilon$ is updated with the gate declaration information. Lastly, the code block is translated. 
\begin{align*}
    trans : \ & Program \dashrightarrow QASMProgam\\
    trans(gDcl_1 \dots gDcl_n \ blk) = \ & \texttt{OPENQASM 3.0;}\\
                & \texttt{include "stdgates.inc";}\\
                & bt(blk, update(update(update(st_\epsilon, gDcl_1), ...), gDcl_n))
\end{align*}  
In this case, the initial symbol table $st_\epsilon$ is an empty symbol tables with only an initialization of the unique identifier counter. 
\begin{equation*}
    st_\epsilon = [ \texttt{uid} \mapsto 0 ]
\end{equation*}

To translate code blocks, the block translation function $bt$ is used. It maps a block $blk$ and a symbol table $st$ to the OpenQASM translation of the code block. Since a block consists of a list of statements and declarations, they are translated individually. However, the translation of a declaration may adjust the symbol table. Therefore, the translation of a translatable, \ie either a statement of declaration, returns not just the translation but also a, potentially updated, symbol table. This symbol table is used for the next translation. Additionally, if the block is empty, the function simply returns an empty result. 
\begin{align*}
    bt : \ & Block \times SymbolTable \dashrightarrow QASM\\
    bt(t_1 \dots t_n, st_1) = \ &  tr_1 \quad \text{where } (tr_1, st_2) = tt(t_1, st_1)\\
    & tr_n \quad \text{where } (tr_2, st_3) = tt(t_2, st_2)\\
    & \dots\\
    & tr_n \quad \text{where } (tr_{n - 1}, st_n) = tt(t_{n - 1}, st_{n - 1})\\
    & tr_n \quad \text{where } (tr_n, \_) = tt(t_n, st_n)\\
    bt(\epsilon, st) = \ &  \epsilon 
\end{align*}

The translatable translation function $tt$ can translate both declarations and statements. Since declarations may adjust the symbol table, it return not just the translation but also a symbol table. To translate the statements and declarations, it calls the corresponding translation functions $ct$ and $dt$ respectively.
\begin{align*}
    tt : \ & Translatable \times SymbolTable \dashrightarrow QASM \times SymbolTable\\
    tt(t, st) = \ & \begin{cases}
        dt(t, st)  \quad &\text{if } t \in Declarations\\
        (ct(t, st), st) &\text{otherwise }
    \end{cases}  
\end{align*}

The declaration translation $dt$ returns a possible translation of the given declaration and an updated symbol table. In the case of constant declarations, the symbols are only used at compile time and, therefore, only the symbol table is updated and any empty result is returned. In contrast, qubit and register declarations need to be specified in the target program. In turn, they are translated. While the syntax is quite similar in both languages, the translation needs to ensure the uniqueness of identifiers and evaluated the expression that gives the size of the register. A unique identifier is generated by the symbol table when it is updated with a qubit or register identifier.
\begin{align*}
    dt : \ & Declaration \times SymbolTable \dashrightarrow QASM \times SymbolTable\\
    dt(\underbrace{\texttt{qubit } id \text{;}}_{decl}, st) = \ & (\texttt{qubit } uid\texttt{;}, st')\\
                                                                & \text{where } st' = update(decl, st) \text{ and } st'[id] = (\texttt{qubit}, 1, uid)\\
    dt(\underbrace{\texttt{qubit[} exp \texttt{] q;}}_{decl}, st) = \ & (\texttt{qubit[} at(exp, st) \texttt{] } uid\texttt{;}, st')\\
                                                                & \text{where } st' = update(decl, st) \text{ and } st'[id] = (\texttt{qubit}, n, uid)\\
    dt(\underbrace{\texttt{const } id \texttt{ = } exp \texttt{;}}_{decl}, st) = \ & (\epsilon, update(decl, st))
\end{align*}

The update function $update$ is used to update the symbol table with symbol information from a declaration. In turn, it maps a declaration and symbol table to a symbol table.
\begin{equation*}
    update :  Declaration \times SymbolTable \dashrightarrow SymbolTable
\end{equation*}
In the case of a constant declaration, its identifier maps to the \texttt{const} keyword and the evaluation of expression given in the declaration. For the gate declaration, the symbol table maps the identifier to the \texttt{gate} keyword, the block of the gate declaration, and a list of identifiers that represent the arguments to the gate.  
\begin{align*}
    update(\texttt{const } id \texttt{ = } exp \texttt{;}, st) = \ & st[id \mapsto (\texttt{const}, at(exp, st))]\\
    update(\texttt{gate } id \texttt{(}id_1, \dots, id_n\texttt{)} \texttt{ do } blk \texttt{ end}, st) = \ & st[id \mapsto (\texttt{gate}, blk, id_1, \dots, id_n)]
\end{align*}
In the case of the qubit and register declaration, the update function is more complex. Since our language allows variables in independent scopes to have the same identifiers, we need to ensure the uniqueness of identifiers in the target code. For this, the symbol table contains a unique identifier counter which is used to create an unique identifier and incremented. The rest of the update is similar to the previous cases and symbol information, such as the size and unique identifier of a qubit of register, is saved to the symbol table.   
\begin{align*}
    update(\texttt{qubit } id\texttt{;}, st) = \ & st[id \mapsto (\texttt{qubit}, 1, \texttt{id\_}uid), \texttt{uid} \mapsto uid + 1]\\
                                                 & \text{where } uid = st[\texttt{uid}]\\
    update(\texttt{qubit[} exp \texttt{] } id{;}, st) = \ & st[id \mapsto (\texttt{qubit}, at(exp, st), \texttt{id\_}uid), \texttt{uid} \mapsto uid + 1]\\
                                                 & \text{where } uid = st[\texttt{uid}]
\end{align*}

The arithmetic translation $at$ is used to evaluated expressions in the source code to constant real values. While a constant value just evaluates to itself and an identifier to its constant value, the operations evaluate to the operation applied to the evaluation of the subexpressions. 
\begin{align*}
    at : \ & Expression \times SymbolTable \dashrightarrow \mathbb{R}\\
    at(n, st) = \ & n\\
    at(id, st) = \ & val \quad \text{if } st[id] = (\texttt{const}, val)\\
    at(exp_1 + exp_2, st) = \ & at(exp_1, st) + at(exp_2, st)\\
    at(exp_1 - exp_2, st) = \ & at(exp_1, st) - at(exp_2, st)\\
    at(exp_1 * exp_2, st) = \ & at(exp_1, st) * at(exp_2, st)\\
    at(exp_1 / exp_2, st) = \ & at(exp_1, st) / at(exp_2, st)\\
    at((exp), st) = \ & at(exp, st)\\
    at(-exp, st) = \ & -at(exp, st)\\
    at(\texttt{sizeof(} id \texttt{)}) = \ & n \hspace{8em} \text{if } st[id] = (\texttt{qubit}, n, uid)\\
    at(\texttt{sizeof(} id \texttt{)}) = \ & \texttt{sizeof(} qArg \texttt{)} \texttt \quad \quad \text{if } st[id] = (\texttt{arg}, qArg)
\end{align*}

The statements are translated with the statement translation function $ct$. It maps a statement and symbol table to the corresponding translation. Firstly, the skip statement is translated to an empty result.
\begin{align*}
    ct : \ & Statement \times SymbolTable \dashrightarrow QASM\\
    ct(\texttt{skip;}, st) = \ & \epsilon
\end{align*}

Next, the translation of a gate application is divided into the application of a constant and composite gate. In the case of a constant gate application, the same gate can be used in the translation and only the qubit arguments need to be translated to their translated counterparts. For this, the qubit translation function $qt$ is used. 
\begin{align*}
    ct(gate \ qArg_1, \dots, qArg_n\texttt{;}, st) = \ & gate \ qt(qArg_1, st), \dots, qt(qArg_2, st); \\
                                                       & \text{if } gate \in ConstGates\\
\end{align*}
To translate the application of a composite gate, the corresponding code block is translated. Additionally, the symbol table, used in the translation, maps the argument identifiers of the gate symbol to the qubit translation of the given qubit arguments.
\begin{align*}
    ct(gate \ qArg_1, \dots, qArg_n\texttt{;}, st) = \ & bt(blk, st[id_1 \mapsto (\texttt{arg}, q_1), \dots, id_n \mapsto q(\texttt{arg}, q_n)]) \\
        &\text{if } gate \not\in ConstGates\\
        &\text{where } q_i = qt(qArg_i, st), i \in [1, ...\, n]\\
        &\text{and } st[gate] = (\texttt{gate}, blk, id_1, \dots, id_n)
\end{align*}
The qubit translation function $qt$ is used to translate the identifier of a gate argument or control qubit, in the case of if-statements, to the corresponding unique identifier regardless of wether the qubit argument is a qubit, register access, or argument identifier of a composite gate. 
\begin{align*}
    qt :\ & \displaystyle QubitArgument \times SymbolTable \dashrightarrow QubitArgument\\
    qt(id, st) = \ & uid \quad\quad\quad\quad\quad\quad \text{if } st[id] = (\texttt{qubit}, 1, uid)\\
    qt(id[exp], st) = \ & uid\texttt{[}at(exp, st)\texttt{]} \quad \text{if } st[id] = (\texttt{qubit}, m, uid) \text{ and } m > n\\
    qt(id, st) = \ & qArg \quad\quad\quad\quad\quad\quad \text{if } st[id] = (\texttt{arg}, qArg)\\
    qt(id[exp], st) = \ & qArg\texttt{[}at(exp, st)\texttt{]} \quad \text{if } st[id] = (\texttt{arg}, qArg)
\end{align*}

The next statement translation case if the loop statement. It is translated by evaluating the given range expression with the range expression evaluation function $rt$ to get the range $(start, end)$. Next, the loop body is translated $end - start$ times and each time the symbol table is updated such that the loop iterator identifier maps the the current iteration value.
\begin{align*}
    ct(\texttt{for } id \texttt{ in } rExp \texttt{ do } blk \texttt{ end}, st) = \ 
        & bt(blk, st[id \mapsto (\texttt{const}, start)])\\
        & bt(blk, st[id \mapsto (\texttt{const}, start + 1)])\\
        & \dots\\
        & bt(blk, st[id \mapsto (\texttt{const}, end - 1)])\\
        & bt(blk, st[id \mapsto (\texttt{const}, end)])\\
        & \text{where } (start, end) = rt(rExp, st)
\end{align*}
The range expression evaluation function simply differentiates between the three different possibilities of defining a range and returns the range as a tuple. 
\begin{align*}
    rt : \ & rExp \times SymbolTable \to \mathbb{Z} \times \mathbb{Z}\\
    rt(z_1 .. z_2, st)  = \ & (z_1, z_2)\\
    rt(\texttt{range(} exp \texttt{)}, st) = \ & (0, at(exp, st) - 1)\\
    rt(\texttt{range(} exp_1, exp_2 \texttt{)}, st) = \ & (\floor{at(exp_1, st)}, \floor{at(exp_2, st)})
\end{align*}

The last translation is the translation of if-statements. The if-statement is translated by adding the given qubit argument as a control qubit to all gate applications in the translated code. To achieve this, the code block is translated and the $control$ function, together with the translated qubit argument, is applied to the translation. In the case of the optional else-block, the $nControl$ function is used to add the qubit argument as a negative control to the translated gate applications.
\begin{align*}
        ct(\texttt{qif } qArg \texttt{ do } blk \texttt{ end}, st) = \ 
            &  control(qt(qArg, st), kt(blk, st)) \\
        ct(\texttt{qif } qArg \texttt{ do } blk_1 \texttt{ else } blk_2 \texttt{ end}, st) = \ 
            &  control(qt(qArg, st), kt(blk_1, st)) \\
            &  nControl(qt(qArg, st), kt(blk_2, st))
\end{align*}
The $control$ function maps a qubit argument $qArg$ together with QASM code to a controlled version given code where $qArg$ if a control qubit for all gate applications in the program. In the case of a list of quantum statements, the function is applied to each statement separably. If the statement is a qubit declaration, the declaration is simply returned without changes.
\begin{align*}
    control : \ & QubitArgument \times QASM \to QASM\\
    control(qArg, qStm_1 \dots qStm_n) = \ & control(qArg, qStm_1)\\
        & ...\\
        & control(qArg, qStm_n)\\
    control(qArg, qubitDcl) = \ & qubitDcl
\end{align*}
If a gate application is given, the qubit argument is prepended to the gate arguments and a control modifier is either added of adjusted to the new number of control qubits.
\begin{align*}
    &control(qArg, cGate \ qArg_1, \dots, qArg_n ) =  \texttt{ctrl(1) @ } cGate \ qArg, qArg_1, \dots, qArg_n\texttt{;}\\
    &control(qArg, cGate \ \texttt{negctrl(}n\texttt{) @ } qArg_1, \dots, qArg_n ) = \\
    & \quad \quad \quad \quad \texttt{ctrl(1) @ } \texttt{negctrl(}n\texttt{) }cGate \ qArg, qArg_1, \dots, qArg_n\texttt{;}\\
    &control(qArg, \texttt{ctrl(}i \texttt{) @ } cGate \ qArg_1, \dots, qArg_n ) = \\
    & \quad \quad \quad \quad \texttt{ctrl(}i+1 \texttt{) @ } cGate \ qArg, qArg_1, \dots, qArg_n\texttt{;}\\
    &control(qArg, \texttt{ctrl(}i \texttt{) @ } \texttt{negctrl(}n\texttt{) @ } cGate \ qArg_1, \dots, qArg_n ) = \\
    & \quad \quad \quad \quad \texttt{ctrl(}i+1 \texttt{) @ } \texttt{negctrl(}n\texttt{) @ } cGate \ qArg, qArg_1, \dots, qArg_n\texttt{;}
\end{align*}
Correspondingly, the $nControl$ function exhibits the same behavior but the qubit argument $gArg$ represent an negated control qubit; its definition can be found in Appendix~\ref{appendix:translation}. Since place the negative control as the second modifier, the $nControl$ function cannot simply prepend the qubit argument $qArg$. However, it is inserted after the first $i$ arguments where $i$ is the number of positive control qubits given by the \texttt{ctrl} modifier.

% Only in implementation
%\section{Error Handling}
Generally, an important part of a program is error handling; useful and precise error messages are essential for comfortable interactions with the program. This is especially the case for compilers where the user should not only easily understand what the issue is but also where in the source code the error occurred.  

Our compiler has two types of errors with different severities. The first type is the \emph{warning}. A warning from the compiler can indicate issues in the source code that may cause unintended behavior. However, the issue itself does not prevent the compilation of the program and is simply an indication that there may be something wrong. In contrast, the \emph{critical error} is caused by a flaw in the source program that prevents the correct compilation and will result in the abortion of the compilation. In the following, we will discuss the different warnings and critical errors the compiler may raise and their corresponding causes. Furthermore, we discuss why they are either a warning or critical error. 

\subsection{Warnings}
The compiler can throw two different kinds of warnings. The first is the invalid range warning and the second is the unused symbol warning.

An invalid range warning can occur in the context of loop statements. They iterate over a range that is defined by the user. It can be given as either a size $n$ and iterate from $0$ to $n-1$ or a start and end index, $i_{Start}$ and $i_{End}$ respectively, and iterate from the start to the end. However, the range iterator is designed to only increase. Therefore, a range where $i_{Start} \geq i_{End}$ is invalid. Since the for loop is unrolled at compile time, a range with a size less than or equal to zero can just be ignored. However, the user may not indent this behavior. Therefore, the compiler warns the user that the range is invalid.

The unused symbol warning is raised when a symbol, \eg a register of composite gate, is defined in the source code but never used. The unused symbol does not have any negative effect on the compilation and the optimization step can easily remove, \eg, an unused register. Therefore, this is only a warning and the program can be compiled. However, an unused symbol may indicate that the wrong symbol was used somewhere else or part of the program is no longer used. Hence, the user is warned of the unused symbol and unintended behavior may be prevented.

\subsection{Critical Errors}
The list of critical errors, often just referred to as errors, is more extensive than the list of warnings. It includes the invalid access, number of arguments for gates and functions, and size errors. Furthermore, there are errors for the attempted declaration of a variable that is already declared, an undeclared and type error, and, lastly, an error for the invalid use of a qubit in a guarded code block.

The first error is the invalid access error. It occurs when a register is accessed at an invalid index $i$, \ie $i$ is either smaller than zero or larger than $size - 1$, where $size$ is the size of the register. While this error could easily be ignored and would cause no issue when compiling the program, the resulting code would be an invalid circuit description. 

Secondly, the invalid number of arguments error for gates and functions is caused when the number of arguments given to either a gate or function does not correspond to the number of required arguments. For example, the Hadamard gate always expects one argument while the controlled-not gate requires two. Similarly, the \texttt{sizeof}-function operates on only one argument. The compiler cannot proceed when given too few arguments; for the opposite case, while dropping any leftover arguments is possible, it would result in unexpected behavior. % Additionally, too many arguments indicate an issue in the program. 
Therefore, the compiler reports an error for both cases and aborts the compilation.

Another error is the invalid size error. It occurs then a register is declared with an invalid size. A size is invalid if it is less or equal to $0$. A register with non entries cannot be used for anything and, likely, indicates an issue in the program while a register with a negative amount of entries is impossible. Therefore the compilation is aborted and the error is thrown.

The next two errors are concerned with the declaration of variables in a given context; they are the undeclared and already-declared errors. An undeclared error is raised when a variable is used in a context where it is not defined. In this case, the symbol table does not have a symbol stored for the given identifier and the compiler cannot continue. In contrast, the already-declared error occurs when a variable is declared in a context where the same variable identifier has already been assigned to a different symbol. While the compiler could overwrite the previous declaration, this can easily lead to unexpected behavior and, in turn, we do not allow a declaration in the same scope to be overwritten. 

The type error is thrown when a variable is used in a function or gate but is not the required type. Languages with loose typing may be able to convert some types to the required type by, for example, parsing the integer value of a string. However, this can not only result in unexpected behavior and hard to debug errors in the code but, in the case of a quantum language, it may also require the conversion between classical and quantum data which is not easily achievable. 

Finally, the last error is the use-of-guard error; it occurs when a qubit is referenced in a context that is guarded by itself. While the compiler can easily translate any such occurrence, they result in a invalid circuit description. As described in Sec.~\ref{sec:background_branching}, a gate that operates on and is controlled by the same qubit cannot be reversible. Therefore, the compiler prevents the generation of an invalid circuit and aborts the compilation. 

\section{Optimization}
\label{sec:concept_optimization}
Our compiler does not only translate the source language Luie to the target language OpenQASM but can also apply optimizations to the program. Since quantum programs must be deterministic and, therefore, most language constructs are translated at compile time, many classical optimizations are inherently applied. The following, inherent optimizations are discussed generally in Sec.~\ref{sec:background_codeOptimization}.

The first kind of inherent optimizations are constant propagation and constant folding. 
While the target language does allow for expressions, they need to be constant. Furthermore, some language features that depend on expressions have no equivalent in the target language. In turn, they, and their corresponding expressions, need to be evaluated at compile time. This is the case for, \eg, the control flow statements. Therefore, any variable is always constant and its value is propagated to be used in the evaluation of expressions. Further, each expression is evaluated at compile time such that constant propagation and constant folding are inherently applied.

Secondly, loop unrolling is always applied to all loop statements. OpenQASM does not allow for loop statements or any other method for the iteration of statements, besides the restrictive implicit iteration. Furthermore, quantum computers in general cannot provide the ability to iterate over gates since they operate on static circuits. Therefore, to allow for loop statements in our language, the loop is unrolled entirely at compile time. Additionally, for each iteration the current value of the loop iterator is propagated as a constant through the loop body.   

The last inherent optimization is function inlining. Our language provides the ability to define custom gates that consist of an arbitrary combination of gate applications and, possibly, control flow statements. While OpenQASM has a similar functionality, the composite gates OpenQASM provides are more restrictive; for example, they do not allow for registers as arguments to the gate. Furthermore, in contrast to classical computers, quantum computers do not natively support function calls or related concepts. Therefore, each time a composite gate is used, the compiler inlines the gate body at the location where the gate is applied.   

\subsection{Optimization Rules}
\label{sec:concept_optimizationRules}
After the code is translated, the compiler can perform additional peephole optimizations; these are not inherent to the translation of the program and, therefore, are optional. 
The peephole optimizations can be applied to the internal representation of the source language and most hardware-independent optimizations are usually applied to the intermediate representation instead of the target language. However, the peephole optimization rules only operate on sequences of gate applications such that control flow statements or composite gates may only hinder, not aid, the effectiveness of the optimization. The overall performance of the optimization may be increased by optimizing composite gates before inlining their code so that a gate that is called ten times only needs to be optimized once; nevertheless, inlining the gates before applying the rules may enable more optimizations and, thereby, increase the effectiveness at the cost of performance. Therefore, we apply the optimization rules after the code is translated.
% \todo{Why do we need to apply to translated code and not intermediate? most optimizations typically on intermediate} 
Additionally, the user can not only specify whether optimizations are applied but also which to apply when using the compiler. In the following section, we discuss the different peephole optimization rules. 
Additionally, while some optimizations are referred to by the terms commonly used in literature, as described in Sec.~\ref{sec:background_circuitOptimization}, the others without any naming conventions are given descriptive names.
% Additionally, the theoretical background for the rules is discuss in Sec.~\ref{sec:background_circuitOptimization}.

The first kind of optimization rules are the \emph{null gate} optimizations; they describe sequences of gate applications such that the resulting behavior is equivalent to applying the identity gate. In the case of classical computers, an example is the sequential execution of two negation operations. In contrast, a quantum example is the application of two successive Hadamard gates. While these optimizations can easily be performed by the programmer themselves for a simple list of instructions, the manual optimizations increase in complexity when using composite gates and control flow statements. Moreover, the programmer cannot remove two null gates by hand that are contained in two different successive composite gates. Therefore, the optimization rules can not only help to reduce the workload of the programmer but apply optimization rules that cannot be implemented without major changes to the program.

Next, the \emph{peeping control} optimization rule also belongs to the null gate rules. However, its implementation requires some additional evaluations and, therefore, we separate it from the other null gate rules. The peeping control rule can remove a controlled gate from the circuit if the value of the control wire is $\ket{0}$ at the position of the gate. To estimate the value of the control wire, the implementation needs to iterate over all previous gates on the wire. Therefore, while it is still a null gate, its implementation differs greatly from the other null gates. Furthermore, we implement an additional optimization that removes the control from the gate if the value of the control wire is known to be $\ket{1}$. While the rule does not reduce the gate count of the circuit, it may enable further optimizations on the control wire. For example, two $X$ gates on the control wire can be separated by a controlled-not gate. If the value of the control is $\ket{1}$ when the controlled-not gate is applied, the control can be removed and the now successive $X$ gates can be removed with a null gate optimization. Since this optimization does not remove the gate from the circuit, it is not a null gate optimization.

Another optimization rule that the compiler implements is the \emph{Hadamard reduction} rule. It implements the matrix equivalences for both the $X$ and $Z$ gates being surrounded by Hadamard gates, $HXH = Z$ and $HZH = X$. Thereby, the rule reduces the gate count of the circuit if the optimization rule is applied. However, the optimization rule does not remove the gate combination but replaces it with another; therefore, the rule is not a null gate optimization. 
%\change{Dont like the description, cannot come up with better right now}
Similar to the null gate optimizations, the rules themselves are not hard to apply by hand. However, in combination with the more complex statements available in the language, the application of the rule is not trivial and the compiler can optimize parts of the circuit that would have required major changes to optimize them manually.

Lastly, the \emph{control reversal} optimization rule optimizes a controlled-not gate that is surrounded by Hadamard gates on both the control and target wires. When applied, the four Hadamard gates are removed and the control and target qubits of the gate are switched. Therefore, the gate count is reduced by four gates. As described in Sec.~\ref{sec:background_circuitOptimization}, the optimization is based on two Hadamard reductions and the control reversal of the controlled-$Z$ gate. However, the reversal of the controlled-$Z$ gate does not have any direct gain when applied; it can only enable other optimizations. Furthermore, the application of the controlled-$Z$ reversal may also disabled other optimizations. Therefore, an optimization algorithm using the rule would need to either test both possibilities or estimate the value of the application and, in turn, increase the complexity of the optimization algorithm significantly. Because of this, our optimizations do not include the controlled-$Z$ reversal but implement only the special use case of the rule, the control reversal optimization rule.

\subsection{Circuit Graph}
\label{sec:concept_circuitGraph}
While it is possible to directly apply optimizations to the program code or internal representation of the code, this approach can be tedious and error prone. For example, the easiest approach would be to iterate over the code and search for code sequences with more efficient but equivalent alternatives, similar to the peephole optimization patterns on classical computers presented in Sec.~\ref{sec:background_codeOptimization}. However, two consecutive gates operating on a single wire may be separated by multiple gate applications on different wires in the programmatic description. 
Therefore, many simple optimization rules may not be applied when using a simplistic algorithm. A more complex approach would be to subdivide the program into lists of gate applications where the wires, being operated on by each list, are disjunct. While this approach can result in the application of more optimization rules, it will also miss possible applications and already requires a complex implementation. Furthermore, improving on this method only increases its complexity and, in turn, makes it more prone to errors and generally tedious to work with and debug. Therefore, the language does not directly apply the optimizations to the program but uses a circuit graph description, based on the graph described by Kreppel et al.~\cite{KMO*23}, to apply the optimizations.

The circuit graph $C$ is a graphical description of a quantum circuit; it is an acyclic and directed graph. It is an extension of the classical graph definition. Therefore, besides the set of nodes $V$ and edges $E$, it includes a set of qubits $Q$ associated with graph and relations between input, output nodes and qubit $Q_V$ as well as the edges and qubits $Q_E$. The set of nodes $V$ consists of the set of input nodes $I$, output nodes $O$, and gate nodes $G$.
\begin{align*}
    C &= (V, E, Q, Q_E, Q_V)\\
    V &= \underbrace{I}_{\text{Input Nodes}} \cup \underbrace{O}_{\text{Output Nodes}} \cup \underbrace{G}_{\text{Gate Nodes}}\\
    E &\subseteq \{ (x, y) \mid x,y \in V \land x \neq y \}\\
    Q_E &\subseteq \{ (e, q) \mid e \in E \land q \in Q \}\\
    Q_V &\subseteq \{ (v, q) \mid v \in I \cup O \land q \in Q \}
\end{align*}
For each qubit in the circuit, there exists both an input node and an output node and each input-output node pair is assigned exactly one qubit.
Furthermore, each input node has exactly one outgoing edge while each output node has exactly one incoming edge; the qubit assigned to the outgoing or incoming edges are the same as the qubits assigned to the corresponding input or output nodes.  
\begin{align*}
    \forall q \in Q :\ & (\exists_{=1} i \in I \text{ such that } (i, q) \in Q_V) \land\\
                       & (\exists_{=1} o \in O \text{ such that } (o, q) \in Q_V) \\
    \forall i \in I :\ & (\exists_{=1} v \in (O \cup G) \text{ such that } (i, v) \in E  \ \land\\
                       & (\exists_{=1} q \in Q \text{ such that } (i, q) \in Q_V \ \land \\
                       &  ((i, v), q) \in Q_E))\\
    \forall o \in O :\ & (\exists_{=1} v \in (I \cup G) \text{ such that } (v, o) \in E \ \land\\
                       & (\exists_{=1} q \in Q \text{ such that } (o, q) \in Q_V \ \land \\
                       &  ((v, o), q) \in Q_E))
\end{align*}

Besides the input and output nodes, all other nodes represent gates in the circuit. For all gate nodes, the number of incoming edges is equivalent to the number of outgoing edges. Additionally, the number of incoming, or outgoing, edges is the same as the number of arguments for the gate. 
Importantly, in this case, any qubits controlling the application of the gate, \eg the first qubit in a controlled-not gate, also count to the number of arguments. 
\begin{align*}
    \forall g \in G :\ & |g| = 2n \quad \text{where } n \text{ Number of arguments for gate of } g
\end{align*}
Each edge is assigned exactly one qubit. Furthermore, for all gate nodes, each incoming edge has a corresponding outgoing edge with the same assigned qubit.
\begin{align*}
    \forall e \in E :\ & (\exists_{=1} q \in Q \text{ such that } (e, q) \in Q_E)\\
    \forall v \in G :\ & (\exists v': (v', v) \in E) \implies\\
    & (\exists q, v'' : ((v', v), q) \in Q_E \land (v, v'') \in E \land ((v, v''), q) \in Q_E)
\end{align*}
Therefore, for each qubit, there exists one path from its input node to its output node such that all gates that are applied to it are visited in order of application.

\subsubsection{Graph Construction}
When using the circuit graph to optimize a quantum program, the first step is to systematically construct the graph from the program. The creation of the graph starts with the input and output nodes for each qubit in the circuit. For each declaration of a qubit, an input and output node pair is created. If a register is declared, a pair is created separately for each qubit in the register, \ie for a register with size $n$, $n$ pairs are created in total. Next, the gate applications in the program can be iterated. For each gate application, a corresponding gate node is created. To insert this node into the graph, each qubit argument requires an incoming edge to this node and a corresponding outgoing edge from the node. Additionally, the incoming node must come from the gate that was previously applied to the qubit or, if no gate was applied beforehand, the input node. Similarly, the outgoing edge must lead to either the next applied gate or the corresponding output. Therefore, for each qubit, the edge coming into the output node can be diverted to the gate node and an outgoing edge, from the gate to the output node, can be created. Repeating this step for all gate applications in the program results in the corresponding circuit graph.

An example of a simple, unoptimized circuit graph is depicted in Fig.~\ref{fig:circuit_graph_unoptimized}. For simplicity, all applied gates are depicted inside of the corresponding node in the circuit graph. The circuit consists of three qubits, $q_0$, $q_1$, and $q_2$. Their input and output nodes are depicted on the left and right of the graph respectively and are labeled with the corresponding qubit. Firstly, an $X$ gate is applied to the first qubit $q_0$. Then, a controlled-not gate is applied to the first two qubits $q_0$ and $q_1$, where $q_0$ is the control qubit. Simultaneously, two Hadamard gates $H$ are applied to the third qubit $q_2$. Lastly, another $X$ gate is applied to the first qubit $q_0$ while another controlled-not gate is applied to the second and third qubits $q_1$ and $q_2$. In this case, the second qubit $q_1$ is the control qubit.

\begin{figure}[htp]
    \centering     
    \includegraphics[width=.9\textwidth]{../figures/drawio/circuit_graph_unoptimized.pdf}
    \caption{An example of a simple, unoptimized circuit graph.}
    \label{fig:circuit_graph_unoptimized}
\end{figure}

\subsubsection{Graph Optimization}
The next step in the optimization process is the application of optimization rules. In this case, these are peephole optimizations. To optimize the graph, subgraphs are systematically iterated for each qubit $q \in Q$ by walking along the wire path of $q$. 
A wire path is a path that starts at the input node of a qubit $q$, ends in the corresponding output node, and follows the edges corresponding to the qubit $q$. 
For each node $star$ in the wire path, subpath $p$ starting at $start$ and following the wire path up to a maximum length $max$ are iterated.
This $max$ length depends on the maximum number of nodes that are affected by an optimization rule.
Each subgraph $p$ is checked for an optimized alternative based on a list of optimization rules $R$. If one is found, the subgraph is replaced with it. 
However, a single iteration over the entire graph may not find all optimizations as the application of optimizations may enable further optimizations. 
For example, after removing a gate combination, two previously separated Hadamard gates may now represent a null gate combination which can, in turn, also be removed. 
Therefore, the process needs to be repeated until no more optimizations can be applied. This is the case, if the process is repeated without applying any optimizations.
The optimization algorithm, depicted in Alg.~\ref{alg:concept_optimizationAlgorithm}, shows how the subpaths are iterated.

% Pseudo code for algorithm
\SetKwComment{Comment}{\# }{}
\begin{algorithm}
    \caption{The algorithm used to optimize a circuit graph.}
    \label{alg:concept_optimizationAlgorithm}
    \KwData{Circuit Graph $C = (V, E, Q, Q_E, Q_V)$, List of optimization rules $R$, Maximal rule length $max$}
    $repeat \gets true$\;
    \While{$repeat = true$}{
        $repeat \gets false$\;
        \ForEach(\Comment*[f]{Iterate through all qubits}){$q \in Q$}
        {
            $start \gets i \quad \text{where } (i, q) \in Q_V \land i \in I$\Comment*[r]{$I$ input nodes}
            \While(\Comment*[f]{$O$ output nodes}){$start \not\in O$}{
                $last \gets start$\;
                $p \gets [start]$ \Comment*[r]{Create path that starts with $start$}
                \For(\Comment*[f]{Iterate paths up to $max$ length}){$i \gets 0$ \KwTo $max$}{
                    \ForEach(\Comment*[f]{Check all rules for applicability}){$r \in R$}{ 
                        \If{$r$ can be applied to $p$}{
                            apply $r$ to $p$\;
                            $repeat \gets true$\;
                            $break$\;
                        }
                    }
                    \If(\Comment*[f]{Existence of next node}){$\nexists \ v : ((last, v), q) \in Q_E$}{
                        break\;
                    }
                    $last \gets v \quad \text{where } ((last, v), q) \in Q_E$\; 
                    append $last$ to $p$\;
                }
                
                $start \gets v \quad \text{where } ((start, v), q) \in Q_E$\Comment*[r]{Next node on wire path}
            }
        }
    }
\end{algorithm}

An example optimization process of a circuit graph depicted in Fig.~\ref{fig:circuit_graph_unoptimized}. Firstly, subpaths for the wire path of the first qubit $q_0$ are iterated. Here, the subpath of note is the path with the first two gate nodes $X$ and $CX$. Since the $X$ node is the child of the input node, we know that the value of the qubit will be $\ket{1}$ after the application. Therefore, we know that the controlled-not gate will always apply the $X$ gate to the second qubit $q_1$ and we can replace the $CX$ gate node with a simple $X$ node that is only visited by the second wire path. For simplicity, we assume that the optimizations are only applied after all subpaths are iterated. In turn, there is no further optimization that can be applied to the second qubit in this round of the optimizations. However, in our implementation, the optimizations are applied while iterating over the circuit. Lastly, on the third qubit wire, there are two consecutive Hadamard gates that can be removed. Overall, the first round of optimizations results in the circuit graph depicted in Fig.~\ref{fig:circuit_graph_first_optimized}. 
 
\begin{figure}[htp]
    \centering     
    \begin{minipage}{.6\textwidth}
        \centering     
        \includegraphics[width=\textwidth]{../figures/drawio/circuit_graph_optimized_firststep.pdf}
        \caption{Circuit graph after the first optimization.}
        \label{fig:circuit_graph_first_optimized}
    \end{minipage}
    \hfill
    \begin{minipage}{.35\textwidth}
        \centering     
        \includegraphics[width=\textwidth]{../figures/drawio/circuit_graph_optimized_complete.pdf}
        \caption{Completely optimized graph.}
        \label{fig:circuit_graph_optimized_complete}
    \end{minipage}
\end{figure}

In the next optimization round, the two $X$ gates on the first qubit wire are applied consecutively. Therefore, they can both be removed from the circuit. Finally, the combination of an $X$ gate child of an input node followed by a controlled-not gate can, again, be optimized such that the $CX$ gate is replaced with a simple $X$ gate on the third qubit wire. The result is a circuit where the first qubit remains unchanged and only an $X$ gate is applied to the second and third qubit. This result is also depicted in Fig~\ref{fig:circuit_graph_optimized_complete}. 



\subsubsection{Graph Translation}
After all possible optimizations rules were applied and no others were enabled in turn, the only remaining step is to translate the circuit graph back to a programmatic description. The qubits for the circuit are easily declared by iterating over all input or output nodes. Additionally, the qubit count can be reduced by leaving out all unused qubits, \ie no gates are applied to them. This is the case if the wire path only consists of the input and output nodes. In the case of the optimized circuit, depicted in Fig.~\ref{fig:circuit_graph_optimized_complete}, the first qubit $q_0$ can be skipped when adding the declarations to the translated program. 

For the gate applications, the only requirement is that, if a gate node is a descendent of another, its translated gate application statement must come after the statement of the ancestor node. Therefore, for most circuit graphs there are multiple possible programs describing the circuit correctly. For example, when translating the optimized graph described above, the order of the $X$ gate applications to both the second and third qubit is irrelevant such that both possibilities, \ie either applying the gate to $q_1$ or $q_2$ first, are valid translations of the program. Furthermore, while iterating over all input or output nodes and adding the corresponding qubit and register declarations to the beginning of the program is the easiest approach, the order and placement of the declarations is also arbitrary as long as they are declared before their first use. 



\section{Command Line Interface}
The command line interface of the compiler enables the programmer to interact with the compiler and specify specific behavior. There are currently four different parameters that can be specified, the input file, output file, optimizations to be applied, and the verbosity of the compilation. 

The first, and most important, parameter is the input file. Since without a specified input file the compiler has no program to compile, the parameter is mandatory. It can be given by either ``\texttt{-i}'' or ``\texttt{-{}-input}'' followed by the path to the input file. Similarly, the output file is specified with another parameter; it can be specified with ``\texttt{-o}'' or ``\texttt{-{}-output}'' followed by the path to the output file.  However, this parameter is not required and the default behavior is to create a file with the name ``output.qasm'' in the current directory.

Next, the optimization parameter can be used to specify which optimizations are applied to the code. Here, the possibilities are either none, the null gate, peeping control, Hadamard reduction, and control reversal optimization rules. Each optimization has a specific keyword that can be passed as a parameter to indicate the optimization. Additionally, multiple different optimizations can be applied by listing them separated by a plus sign.
\info{currently, spaces are NOT allowed (should be implemented for ease of use)}
For example, to apply both the null gate and peeping control optimizations, the correct parameter would be ``\texttt{nullgate+peepingcontrol}''. The optimization parameter is specified by either passing ``\texttt{-O}'', in this case an uppercase ``o'' to differentiate it from the output, or ``\texttt{-{}-optimization}''. When the optimization is not explicitly specified, the default behavior is no optimization. 

Another parameter that can be based on the compiler is the verbose parameter; it toggles the verbose mode of the compiler on. When this mode is active, the compiler prints info, warning, and error messages out to the user. In comparison, only specific error messages and warnings are displayed to the programmer when the compiler is executed without the verbose mode. In contrast to the other parameters, the verbose mode does not require any other arguments and can be activated by passing either ``\texttt{-v}'' or ``\texttt{-{}-verbose}'' to the compiler. If the parameter is not passed, the compiler is executed normally. 

Besides the parameters that can be passed to the compiler to modify its behavior, it also provides a help test. The help text can be accessed by passing the help parameter to the compiler with either ``\texttt{-h}'' or ``\texttt{-{}-help}''. Additionally, it is displayed when an invalid or unknown parameter was given by the programmer together with a message that indicates that the given parameter is invalid. The help text itself gives a short description of the compiler and lists all parameters that can be passed. Furthermore, each parameter is listed with a short description of its behavior.

An example for the usage of the CLI is depicted in Fig.~\ref{fig:concept_cli_example}. The first parameter is the input parameter; it indicates that the compiler should compile the Luie program called ``program.luie'' in the current directory. Next, the output file is specified. After the compilation is complete, the compiled code should be located in the build directory of the current directory and be called ``program.qasm''. Following the specification for the input and output files, the optimizations parameter indicates which optimizations are to be applied to the program. In this case, the compiler applies both the null gate and peeping control optimization gates. Lastly, since the verbose parameter was not specified, the compiler will only print essential errors and warnings and no other informational messages.
\begin{figure}[htp]
    \centering
    % , label=fig:concept_cli_example, caption={A command line interface example.}]
    \begin{lstlisting}[language=bash, style=bashstyle]
./LUIECompiler --input "./program.luie" 
               --output "./build/program.qasm" 
               --optimization nullgate+peepingcontrol
    \end{lstlisting}
    \caption{A command line interface example.}
    \label{fig:concept_cli_example}
\end{figure}