\usepackage{ae}
\usepackage{array}
\usepackage{threeparttable}
\usepackage[T1]{fontenc}
\usepackage[utf8]{inputenc}
\usepackage{graphicx}
\usepackage{multicol}
\usepackage{rotating}
\usepackage{float}
\usepackage{listings}
\usepackage{lastpage}
\usepackage{dsfont}
\usepackage{amsmath, amssymb, amsfonts, amstext, xspace}
\usepackage{color}
\usepackage{xcolor}
\usepackage{transparent}
\usepackage{colortbl}
\usepackage{multirow}
\usepackage{caption}
\usepackage{epstopdf}
\usepackage{svg}
\usepackage{subfig}
\usepackage{fp}
\usepackage{mathtools}
\usepackage{longtable}
\usepackage[toc]{appendix}
\usepackage{tabularx}
\usepackage{anyfontsize}
\usepackage{pdfpages}
\usepackage{xargs}   
\usepackage{tikz}

\usepackage{caption}
\usepackage{subcaption}

%\usepackage{minted}
% adapt title and author
\usepackage{titling} % https://tex.stackexchange.com/questions/27710/how-can-i-use-author-date-and-title-after-maketitle

\usepackage[pdftitle={},pdfsubject={},pdfauthor={}]{hyperref}

% comment in this line if you are writing your bachelor thesis
% Adjust your information
\title{Compilation of Quantum Programs with Control Flow Primitives in Superposition}
\subtitle{Master Thesis \ifdefined\PROPOSAL{} Proposal \fi}
\date{January 17, 2025}

\newcommand{\firstname}{Sascha}
\newcommand{\lastname}{Thiemann}
\newcommand{\matrNo}{406187}
\newcommand{\email}{sascha.thiemann@rwth-aachen.de}
\newcommand{\studyProgram}{Computer Science M.Sc.}

% Adjust first supervisor information
\newcommand{\firstsupervisor}{apl.\ Prof.\ Dr.\ Thomas Noll}
\newcommand{\firstsupervisorchair}{Chair for Software Modeling and Verification}
\newcommand{\firstsupervisoruniversity}{RWTH Aachen University}

% Adjust second supervisor information
\newcommand{\secondsupervisor}{Prof.\ Dr.\ rer.\ nat.\ Dominique Unruh}
\newcommand{\secondsupervisorchair}{Chair for Quantum Information Systems }
\newcommand{\secondsupervisoruniversity}{RWTH Aachen University}


\addtolength{\evensidemargin}{-7mm}\addtolength{\oddsidemargin}{7mm}
\newcolumntype{M}[1]{>{\centering\arraybackslash}m{#1}}
\captionsetup[subfloat]{font=scriptsize, labelformat=parens, labelsep=space, listofformat=subparens}

\newcolumntype{L}[1]{>{\raggedright\let\newline\\\arraybackslash\hspace{0pt}}m{#1}}

\definecolor{grey}{rgb}{0.8,0.8,0.8}
\definecolor{red}{rgb}{1,0,0}
\definecolor{green}{rgb}{0,1,0}
\definecolor{yellow}{rgb}{1,1,0}
\definecolor{lightblue}{rgb}{0,0,0.5}
\definecolor{lightgray}{rgb}{.95,.95,.95}
\definecolor{darkgray}{rgb}{.4,.4,.4}
\definecolor{purple}{rgb}{0.65, 0.12, 0.82}

\newcommand{\blankpage}{
	% \ifdefined\PROPOSAL{}
	% \else
		\newpage
		\thispagestyle{empty}
		\mbox{}
		\newpage
	% \fi
}
\def\checkmark{\tikz\fill[scale=0.4](0,.35) -- (.25,0) -- (1,.7) -- (.25,.15) -- cycle;}

\graphicspath{{../figures/}}


\lstloadlanguages{C,C++,csh,Java}
\lstset{
	language=csh,
	basicstyle=\footnotesize\ttfamily,
	numbers=left,
	numberstyle=\tiny,
	numbersep=5pt,
	tabsize=2,
	extendedchars=true,
	breaklines=true,
	frame=tlrb,
	stringstyle=\color{blue}\ttfamily,
	showspaces=false,
	showtabs=false,
	framexleftmargin=17pt,
	framexrightmargin=5pt,
	framexbottommargin=4pt,
	xleftmargin=.2\textwidth,
	xrightmargin=.2\textwidth,
	commentstyle=\color{green},
	morecomment=[l]{//}, %use comment-line-style!
	morecomment=[s]{/*}{*/}, %for multiline comments
	showstringspaces=false,
	morekeywords={ abstract, event, new, struct,
		async, await,
		as, explicit, null, switch,
		base, extern, object, this,
		bool, false, operator, throw,
		break, finally, out, true,
		byte, fixed, override, try,
		case, float, params, typeof,
		catch, for, private, uint,
		char, foreach, protected, ulong,
		checked, goto, public, unchecked,
		class, if, readonly, unsafe,
		const, implicit, ref, ushort,
		continue, in, return, using,
		decimal, int, sbyte, virtual,
		default, interface, sealed, volatile,
		delegate, internal, short, void,
		do, is, sizeof, while,
		double, lock, stackalloc,
		else, long, static,
		enum, namespace, string},
	keywordstyle=\color{cyan},
}

\lstdefinestyle{ANTLR}{
    % basicstyle=\small\ttfamily\color{magenta},%
    breaklines=true,%                                      allow line breaks
    moredelim=[s][\color{green!50!black}\ttfamily]{'}{'},% single quotes in green
    moredelim=*[s][\color{black}\ttfamily]{options}{\}},%  options in black (until trailing })
    commentstyle={\color{gray}\itshape},%                  gray italics for comments
    morecomment=[l]{//},%                                  define // comment
    emph={%
        STRING%                                            literal strings listed here
        },emphstyle={\color{blue}\ttfamily},%              and formatted in blue
    alsoletter={:,|,;},%
    morekeywords={:,|,;},%                                 define the special characters
    keywordstyle={\color{black}},%                         and format them in black
}

\lstdefinestyle{QCM}{
    breaklines=true,%                                      allow line breaks
    moredelim=[s][\color{green!50!black}\ttfamily]{'}{'},% single quotes in green
    moredelim=*[s][\color{black}\ttfamily]{options}{\}},%  options in black (until trailing })
    commentstyle={\color{gray}\itshape},%                  gray italics for comments
    morecomment=[l]{;},%                                  define // comment
    emph={%
		l1, l2, l3, l4, l5, l6, l7, l8%                                         literal strings listed here
        },emphstyle={\color{red}\ttfamily},%              and formatted in blue
    alsoletter={|,;},%
    morekeywords={add, rjne, jz, jg, mul, jmp, rjmp, nop, rjle, radd},%                                 define the special characters
    keywordstyle={\color{blue}},%                         and format them in black
}


\usepackage[colorinlistoftodos,prependcaption,textsize=small]{todonotes}
\newcommandx{\unsure}[2][1=]{\todo[linecolor=red,backgroundcolor=red!25,bordercolor=red,#1]{#2}}
\newcommandx{\change}[2][1=]{\todo[linecolor=blue,backgroundcolor=blue!25,bordercolor=blue,#1]{#2}}
\newcommandx{\info}[2][1=]{\todo[linecolor=green,backgroundcolor=green!25,bordercolor=green,#1]{#2}}
\newcommandx{\improvement}[2][1=]{\todo[linecolor=violet,backgroundcolor=violet!25,bordercolor=violet,#1]{#2}}
\newcommandx{\thiswillnotshow}[2][1=]{\todo[disable,#1]{#2}}
% commands: \unsure, \info, \change, \improvement, \thiswillnotshow; can have attribute [inline]