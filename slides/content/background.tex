\section{Background}
\subsection{Quantum Control Flow}
\begin{frame}{Quantum Control Flow}
    \begin{itemize}
        \item First used by \cite{AlGr05} to define functional programming language.
        \item For example, used to define Hadamard gate as function $had$:  
    \end{itemize}
    \begin{align*}
        had : Q& \to Q\\
        had : x \mapsto& \texttt{ if}^\circ x\\
                       & \texttt{ then } \{false \mid -true\}\\
                       & \texttt{ else } \{false \mid true\}
    \end{align*}
    \begin{itemize}
        \item Later, formally defined by \cite{YYF12}.
        \item Allows for the execution of functions based on values in superposition.
        \item Results in superposition of the results of individual executions.
    \end{itemize}
\end{frame}

\begin{frame}{Limitations --- Reversibility}
    \begin{itemize}
        \item Mainly limited by two principles: \emph{reversibility} and \emph{synchronization}.
        \item Instructions required to be reversible, as they are unitary transformations.
        \item[$\Rightarrow$] Control flow, as implemented in classical computers, not possible.
        % \item For example, any classical jump instruction is inherently irreversible.
        % \item Two jumps can lead to the same location such that a reversed program cannot know from where the jump came
        \item Landauer Embedding~\cite{Land61} seems to offer solution.
        % \item The embedding can turn any non-reversible function into a reversible one by not only returning the output but also the input of the function.
        \item Can make non-reversible function $f : D \to D'$ reversible by outputting input value.
        \item Result is $g : D \to D' \times D$ with $g(x) = (f(x), x)$.
        \item However, output includes program history and the result depends on the history, they become entangled.
        \item Leads to disruptive entanglement, invalid results~\cite{YVC24}.
    \end{itemize}
\end{frame}

\begin{frame}{Limitations --- Synchronization}
    \begin{itemize}
        \item Program counter may become entangled with data.
        \item[$\Rightarrow$] Disruptive entanglement.
        \item The principle of synchronization: Control flow must become independent from the data.
        \item Program may include padding operations to ensure synchronization.
        \item[$\Rightarrow$] Loops cannot depend solely on value in superposition.
        \item[$\Rightarrow$] Instead, a loop must be bounded by classical value~\cite{YVC24}. 
        % \item An example of this is the \emph{Tortoise and hare} problem.
        % \item When a program is run with a loop based on a superposition of two different values, the loop is executed a different number of times and one will finish first.
        % \item The control flow must become independent of the data, but the program cannot wait.
    \end{itemize}
\end{frame}

\subsection{Quantum Control Machine}
\begin{frame}{Quantum Control Machine}
    \begin{itemize}
        \item Quantum Control Machine (QCM), proposed by \cite{YVC24}
        \item Focused on quantum control flow.
        \item Syntax and logic similar to classical assembly language
        % for example, utilizing (conditional) jump instructions.
        \item Employs a branch control register $bcr$ to enable reversible jumps.
        \item Instead of increasing IP by $1$ after statement, increased by value of $bcr$.
        \item The $bcr$ can then be reversibly modified.
        \begin{itemize}
            \item To jump by $5$, the $bcr$ increased by $5$
            \item At its destination, decreased by $5$ again.
        \end{itemize}
    \end{itemize}
\end{frame}


\begin{frame}{Instructions}
    \begin{itemize}
        \item Here, some instructions of the QCM are listed.
        \item For every instruction there exists a reversed instruction, \eg, \texttt{radd} is the subtraction operation.
    \end{itemize}
    \begin{table}[htp]
        \centering
        \begin{threeparttable}[b]
            \begin{tabular}{llp{.5\textwidth}}
                \multicolumn{1}{l|}{Operation}                          & \multicolumn{1}{l|}{Syntax}                                      & Semantics\tnote{1}                                                              \\ \hline
                
                \multicolumn{1}{l|}{No-op}                              & \multicolumn{1}{l|}{\texttt{nop}}               & Only increases instruction pointer by the $bcr$.     \\ \hline
                
                \multicolumn{1}{l|}{Addition}                           & \multicolumn{1}{l|}{\texttt{add} $ra$ $rb$}     & Adds register $rb$ to $ra$.                                            \\
                \multicolumn{1}{l|}{Multiplication}                     & \multicolumn{1}{l|}{\texttt{mul} $ra$ $rb$}     & Multiplies register $ra$ by $rb$.                                      \\ \hline
                
                \multicolumn{1}{l|}{Jump}                               & \multicolumn{1}{l|}{\texttt{jmp} $p$}           & Increases $bcr$ by $p$.                              \\ \hline
                \multicolumn{1}{l|}{\multirow{2}{*}{Conditional Jumps}} & \multicolumn{1}{l|}{\texttt{jz} $p$ $ra$}       & Increases $bcr$ by $p$ if $ra$ is $0$.               \\
                \multicolumn{1}{l|}{}                                   & \multicolumn{1}{l|}{\texttt{jne} $p$ $ra$ $rb$} & Increases $bcr$ by $p$ if $ra$ is not equal to $rb$. 
                % \\ \hline \multicolumn{3}{p{.9\textwidth}}{\small * }                                                                  
            \end{tabular}
            \begin{tablenotes}
                \item [1] After all operations, the instruction pointer is increased by the value of the $bcr$.
            \end{tablenotes}
        \end{threeparttable}
        \caption{An excerpt of the QCM instruction set with instructions used in later examples.}
        \label{tab:qcm_instructionset}
    \end{table}
\end{frame}

\begin{frame}{(Non-) Reversible Example}
    \begin{itemize}
        \item Example of a classical program and the reversible equivalent.
        \item Both programs calculate $x^y$. 
        \item Neither are synchronized because loop is not bound by classical value.
    \end{itemize}
    \begin{figure}[htp]
        \centering     
        \begin{minipage}{.40\textwidth}
            \lstinputlisting[style=QCM]{../figures/code/background/exp_non_reversible.qcm}
            \caption{A non-reversible exponentiation algorithm.}
            \label{fig:qcm_not_reverse}
        \end{minipage}
        \hfill
        \begin{minipage}{.50\textwidth}
            \lstinputlisting[style=QCM]{../figures/code/background/exp_non_sync.qcm}
            \caption{A reversible exponentiation algorithm.}    
            \label{fig:qcm_reverse}
        \end{minipage}
    \end{figure}
\end{frame}

\begin{frame}{Reversible Synchronized Example}
    \begin{itemize}
        \item Synchronized implementation, calculating $x^{\min{y, max}}$.
        \item $max$ is a classical bound for loop.
        \item Line $9$ includes padding instruction.
    \end{itemize}
    \begin{figure}[htp]
        \centering     
        \lstinputlisting[style=QCM, basicstyle=\ttfamily\Large]{../figures/code/background/exp_sync.qcm}
        \caption{A synchronized, reversible exponentiation algorithm.}
        \label{fig:qcm_sync}
    \end{figure}
\end{frame}