\section{Evaluation}
\begin{frame}{Evaluation}
    \begin{itemize}
        \item The evaluation consist of two aspects:
        \begin{enumerate}
            \item the optimizations performed by the compiler and
            \item the execution time of the compilation stages.
        \end{enumerate}
        \item As an example program, we use the quantum ripple-carry adder proposed by \cite{CDKM04}.
        \item It takes two registers $a$ and $b$ as well as two qubits $cin$ and $cout$.
        \item The adder adds the $a$ register to the $b$ register.
        \item The $cin$ and $cout$ qubits are be used as input and output carry bits.
        \item In our implementation, it consists of $CX$ and $CCX$ gates.
    \end{itemize}
\end{frame}

\begin{frame}{Optimization Evaluation}
    \begin{itemize}
        \item For the optimization evaluation, we used both inputs with classical values and values in superposition.
        \item The first inputs were $a = \ket{1}$ and $b = \ket{15}$.
        \begin{itemize}
            \item Since the inputs are classical values, peeping control rules and null gate rules can be applied.
            \item The circuit can be optimized such that the resulting one only contains gates that initialize the result.
            \item Only two X gates remain. 
            \item While the first gate flips the first qubit of the a register, initializing it to $\ket{1}$, the second flips the carry output qubit, indicating a result of $\ket{16}$.
        \end{itemize}
        \item The second inputs were $a = \frac{1}{\sqrt{2}} (\ket{0} + \ket{3})$ and $b = \ket{4}$.
        \begin{itemize}
            \item Since the inputs are now values in superposition, peeping control rules and null gate rules can only partially be applied.
            \item In this case, only twelve of 25 gates can be optimized.
            \item For other inputs in superposition, even fewer gates are optimized.
        \end{itemize}
    \end{itemize}
\end{frame}

\begin{frame}{Performance Evaluation}
    \begin{itemize}
        \item To evaluate the performance, we compiled the adder with an input of $a = \frac{1}{\sqrt{2}} (\ket{0} + \ket{3})$ and $b = 15$ for different register sizes $n$.
        \item Since the program size does not change, the execution times of the semantic analysis remain constant.
        \item The code generation stage shows a linear increase, as the compiled program increases linearly with the register size.
        \item The optimization has the worst performance with an approximate quadratic increase.
    \end{itemize}
    \begin{table}[htp]
        \centering     
        \begin{tabular}{c|ccc}
        \multirow{2}{*}{Register Size $n$} & \multicolumn{3}{c}{Execution Time of Stages in ms}                                                  \\ \cline{2-4} 
                                           & \multicolumn{1}{c|}{Semantic Analysis} & \multicolumn{1}{c|}{Code Generation} & Optimization \\ \hline
%        16                                 & \multicolumn{1}{c|}{28.3}                & \multicolumn{1}{c|}{45.4}              & 117          \\
%        32                                 & \multicolumn{1}{c|}{27.5}                & \multicolumn{1}{c|}{43.8}              & 215.4          \\
        64                                 & \multicolumn{1}{c|}{27.3}                & \multicolumn{1}{c|}{47.8}              & 711.6          \\
        128                                & \multicolumn{1}{c|}{26.3}                & \multicolumn{1}{c|}{50.4}              & 2292.4         \\
        256                                & \multicolumn{1}{c|}{26.2}                & \multicolumn{1}{c|}{59.7}              & 10755.7        \\
        512                                & \multicolumn{1}{c|}{25.8}                & \multicolumn{1}{c|}{74.9}              & 60204.7        \\
        1024                               & \multicolumn{1}{c|}{26.1}                & \multicolumn{1}{c|}{109.1}             & 405376.6      
    \end{tabular}
    \caption{The execution times compiling a quantum ripple-carry adder with different register sizes.}
    % \label{tab:eval_executionTime}
    \end{table}
    
\end{frame}