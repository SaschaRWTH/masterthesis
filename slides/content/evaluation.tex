\section{Evaluation}
\begin{frame}{Evaluation}
    \begin{itemize}
        \item Evaluation consists of two aspects:
        \begin{enumerate}
            \item Optimizations performed.
            \item Execution time stages.
        \end{enumerate}
        \item Example program: quantum ripple-carry adder proposed by \cite{CDKM04}.
        \begin{itemize}
            \item Takes two registers $a$ and $b$, two qubits $cin$ and $cout$.
            \item Adder adds $a$ register to $b$ register.
            \item $cin$ and $cout$ qubits are used as input, output carry bits.
        \end{itemize}
        \item Our implementation consists only of $CX$ and $CCX$.
    \end{itemize}
\end{frame}

\begin{frame}{Optimization Evaluation}
    \begin{itemize}
        \item For optimization evaluation, classical inputs and in superposition.
        \item First inputs: $a = \ket{1}$ and $b = \ket{15}$.
        \begin{itemize}
            \item Classical inputs $\to$ peeping control rules, null gate rules can be applied.
            \item Optimized circuit: only gates that initialize the result.
            \item Only two $X$ gates remain. 
            \item First $X$ flips $a[0]$ qubit $\to a = \ket{1}$, second flips $cout$, indicating result of $\ket{16}$.
        \end{itemize}
        \item Second inputs: $a = \frac{1}{\sqrt{2}} (\ket{0} + \ket{3})$ and $b = \ket{4}$.
        \begin{itemize}
            \item Inputs in superposition $\to$ peeping control rules, null gate rules only applied \emph{partially}.
            \item Only $\frac{12}{25}$ gates can be optimized.
            \item Other inputs in superposition: even fewer gates optimized.
        \end{itemize}
    \end{itemize}
\end{frame}

\begin{frame}{Performance Evaluation}
    \begin{itemize}
        \item Compiled adder with input of $a = \frac{1}{\sqrt{2}} (\ket{0} + \ket{3})$ and $b = \ket{15}$ for different register sizes $n$.
        \item Program size does not change $\to$ execution times of the semantic analysis remain constant.
        \item Code generation stage: linear increase.
        \begin{itemize}
            \item Unrolled loop $\to$ compiled program increases linearly.
        \end{itemize}
        \item Optimization worst performance, approximate quadratic increase.
    \end{itemize}
    \begin{table}[htp]
        \centering     
        \begin{tabular}{c|ccc}
        \multirow{2}{*}{Register Size $n$} & \multicolumn{3}{c}{Execution Time of Stages in ms}                                                  \\ \cline{2-4} 
                                           & \multicolumn{1}{c|}{Semantic Analysis} & \multicolumn{1}{c|}{Code Generation} & Optimization \\ \hline
%        16                                 & \multicolumn{1}{c|}{28.3}                & \multicolumn{1}{c|}{45.4}              & 117          \\
%        32                                 & \multicolumn{1}{c|}{27.5}                & \multicolumn{1}{c|}{43.8}              & 215.4          \\
        64                                 & \multicolumn{1}{c|}{27.3}                & \multicolumn{1}{c|}{47.8}              & 711.6          \\
        128                                & \multicolumn{1}{c|}{26.3}                & \multicolumn{1}{c|}{50.4}              & 2292.4         \\
        256                                & \multicolumn{1}{c|}{26.2}                & \multicolumn{1}{c|}{59.7}              & 10755.7        \\
        512                                & \multicolumn{1}{c|}{25.8}                & \multicolumn{1}{c|}{74.9}              & 60204.7        \\
        1024                               & \multicolumn{1}{c|}{26.1}                & \multicolumn{1}{c|}{109.1}             & 405376.6      
    \end{tabular}
    \caption{The execution times compiling a quantum ripple-carry adder with different register sizes.}
    % \label{tab:eval_executionTime}
    \end{table}
    
\end{frame}