\section{Language}
\subsection{Overview}
\begin{frame}{Language Overview}
    \begin{itemize}
        \item Idea for language: Provide high-level language with the capabilities of the QCM. 
        \item Want to remove low-level concepts, add high-level ones.
        \item Jump instructions in superposition are removed $\to$ need to add other control flow primitives.
        \item Introduce multiple high-level concepts and control flow statements:
        \begin{itemize}
            \item blocks and scopes,
            \item different data types,
            \item composite gates,
            \item loop statements, unrolled at compile time, and
            \item quantum if- and else-statements.
        \end{itemize}
    \end{itemize}
\end{frame}

\subsection{Syntax}
\begin{frame}[fragile]{Syntax}
    \begin{minipage}{.45\textwidth}
        \begin{itemize}
            \item Define $CFG_{Luie}$ for our language.
            \item Start symbol is the program, consisting of gate declarations and a block.
            \item A block is a list of translatables, either statements or declarations.
        \end{itemize}
        \vfill
        \Large
        \begin{align*}
            CFG_{Luie} = \ & (V_{Luie}, \Sigma_{Luie}, R_{Luie}, prg_{Luie} )\\ 
            V_{Luie} = \ & \{ exp, rExp, gate, prg_{Luie}, \dots \}\\ 
            \Sigma_{Luie} = \ & \{\texttt{..}, \texttt{range}, \texttt{(}, \texttt{)}, \dots \} 
         %   \quad \text{where } n \in \mathbb{N}_0, id \in Identifier
        \end{align*}
        \vspace{\alignmargin}
        \begin{align*}
            Program: \ & prg_{Luie} &::=& gDcl_1 \dots gDcl_n \ blk \mid \\
                       &            &   & blk \\
            Block: \ & blk &::=&  t_1 \dots t_n \mid \\
                       &       &&  \epsilon
            % Trans : \ & t::= stm \mid dcl\\
        % \end{align*}
        % \begin{align*}
            % Decl: \ & dcl ::= \texttt{const } id \texttt{ = } exp \texttt{;} \mid \\
            %             & \quad \quad \quad \texttt{qubit } id \texttt{;}\\
            % GateDecl: \ & gDcl::= \texttt{gate } id \texttt{ (}id_1, \dots, id_n\texttt{) do } blk \texttt{ end}
        \end{align*}
    \end{minipage}
    \begin{minipage}{.50\textwidth}
        \begin{figure}[htp]
            \centering     
            \begin{lstlisting}[style=Luie, basicstyle=\ttfamily\large]
/* Gate Declaration */
gate c_h_reg(control, reg) do
   /* Block Start */

   qif control do
     for i in range(sizeof(reg)) do
       h reg[i];  
     end
   end

   /* Block End */
end

/* Block Start */
const regSize : int = 3; // Decl.
qubit c;                 // Decl.
qubit[regSize] a;        // Decl.
c_h_reg c, a;           
/* Block End */
            \end{lstlisting}
            \caption{Luie program with structure highlighted.}
            % \label{fig:qft_example}
        \end{figure}
    \end{minipage}
\end{frame}


\newcommand{\adown}{\rotatebox[origin=c]{90}{$\Lsh$}}
\newcommand{\aup}{\rotatebox[origin=c]{90}{$\Rsh$}}
\begin{frame}[fragile]{Syntax}
    % \vfill
    \begin{minipage}{.45\textwidth}
        \begin{itemize}
            \item Three different statements: quantum if-statement, a loop statement, gate application.
            \item To combine qubit or register access: qubit argument.
            \item Expressions for numeric values or ranges.
            \item Fix set of defined gates $\to$ differentiate translations.
        \end{itemize}
        \Large
        \begin{align*}
            Statement: \ & stm ::= \texttt{qif } qArg \texttt{ do }  blk \texttt{ end} \mid\\
            & \quad \quad \quad \quad \texttt{for } id \texttt{ in } rExp \texttt{ do } blk \texttt{ end} \mid \\
            & \quad \quad \quad \quad id \ qArg_1, \dots, qArg_n \texttt{;}\\
            QubitArg: \ & qArg ::= id \mid id[exp]\\
            % Expression: \ & exp ::= n \mid id \mid exp_1 \texttt{ + } exp_2 \mid exp_1 \texttt{ - } exp_2 \mid \dots\\
            % RangeExpression: \ & rExp ::= n_1 .. n_2 \mid \texttt{range(} exp \texttt{)} \mid \texttt{range(} exp_1, exp_2 \texttt{)}
    \end{align*}
    \end{minipage}
    \begin{minipage}{.50\textwidth}
        \begin{figure}[htp]
            \centering     
            \begin{lstlisting}[style=Luie, mathescape=true,basicstyle=\ttfamily\large, literate={↓}{$\adown$}{1}, literate={↑}{$\aup{}$}{1}] 
gate c_h_reg(control, reg) do
   /* If-Statement */
   qif control do
      /* Loop Stat. | $\adown{}$ Expression */
      for i in range(sizeof(reg)) do
             /* ↑ Range Exp. */
         h reg[i];
         /* ↑ Qubit Argument */
      end
   end
end

const regSize : int = 3; 
qubit c;
qubit[regSize] a;
/* Composite Gate Application */
c_h_reg c, a;  
      /* ↑  ↑ Qubit Argument */                 
            \end{lstlisting}
            \caption{Luie program with statements and arguments highlighted.}
            % \label{fig:qft_example}
        \end{figure}
    \end{minipage}
\end{frame}

% \begin{frame}{Example Program}
%     \begin{itemize}
%         \item In the example program, a composite gate is defined that applies the $H$ gate to a register with a control qubit.
%         \item This composite gate is then applied to a register.
%     \end{itemize}
%     \begin{figure}[htp]
%         \centering     
%         \lstinputlisting[style=Luie, basicstyle=\ttfamily\Large]{../figures/code/slides/luie_example.luie}
%         \caption{An example Luie program.}
%         \label{fig:qft_example}
%     \end{figure}
% \end{frame}


\subsection{Translation}
\begin{frame}{Symbol Table}
    \begin{itemize}
        \item Saves symbol information relevant for translation.
        \item Contains four types of symbols:
        \begin{enumerate}
            \item named constants,
            \item quantum registers and qubits,
            \item qubit arguments, and
            \item composite gates. 
        \end{enumerate} 
    \end{itemize}
    \begin{align*}
        SymbolTable := \ \{st \mid st : Identifier \dashrightarrow & (\{\texttt{const}\} \times \mathbb{Q})\\
        \cup& (\{\texttt{qubit}\} \times \mathbb{N} \times Identifier)\\
        \cup& (\{\texttt{arg}\} \times QubitArgument)\\
        \cup& (\{\texttt{gate}\} \times Block \times Identifier^+)
        \}
    \end{align*}
\end{frame}

\begin{frame}{Translation Function and Block Translation}
    \begin{itemize}
        \item $trans$ function translates program.
        \item Symbol table updated with gate declaration by $up$ function.
        \item Block translation $bt$ translates all translatables.
        \item Updates symbol table used in next translation.
    \end{itemize}
    \Large
    \begin{align*}
        trans(\texttt{gate c\_h\_reg} \dots) =& \ \texttt{OPENQASM 3.0;}\\
                                              & \ \texttt{include "stdgates.inc";}\\
                                              & \ bt(\texttt{const regSize = 3;} \dots, up(st_\epsilon, \texttt{gate c\_h\_reg} \dots))\\
        st_1 = up(st_\epsilon, \texttt{gate c\_h\_reg} \dots) =& \ 
        [ \texttt{c\_h\_reg} \mapsto (\texttt{gate}, \texttt{qif control do ... end}, \\
        & \quad\quad\quad\quad\quad\quad\quad \texttt{control}, \texttt{reg})  ]\\
        &\\
        bt(\texttt{const regSize = 3;} \dots, st_1) =& \ tr_1 \quad \quad \text{ where } (tr_1, st_2) = tt(\texttt{const regSize = 3;}, st_1)\\ 
                                                & \ tr_2 \quad \quad \text{ where } (tr_2, st_3) = tt(\texttt{qubit c;}, st_2)\\ 
                                                & \ tr_3 \quad \quad \text{ where } (tr_3, st_4) = tt(\texttt{qubit[regSize] a;}, st_3)\\ 
                                                & \ tr_4 \quad \quad \text{ where } (tr_4, \_) = tt(\texttt{c\_h\_reg c, a;}, st_4)
    \end{align*}
\end{frame}

% \begin{frame}{Translatable and Declaration Translation}
%     \begin{itemize}
%         \item $tt$ translates each translatable, differentiates between declaration and command.
%         \item Declarations translated by $dt$. 
%         \item Update symbol table $\to$ function returns (updated) symbol table.
%         \item Language allows for different variable scopes $\to$ same identifier can be used multiple times.
%         \item[$\Rightarrow$] Unique identifier $uid$ is generated for translation. 
%     \end{itemize}
%     \LARGE
%     \begin{align*}
%         tt(\texttt{c\_h\_reg c, a;}, st) =& \ (ct(\texttt{c\_h\_reg c, a;}, st), st)\\
%         tt(\texttt{qubit c;}, st) =& \ dt(\texttt{qubit c;}, st)\\
%         &\\
%         dt(\texttt{qubit c;}, st) =& \ (\texttt{qubit } uid_c \texttt{;}, up(st, \texttt{qubit c;}))\\
%         &\\
%         up(st, \texttt{qubit c;}) =& \ st[\texttt{c} \mapsto (\texttt{qubit}, 1, uid_c)]
%     \end{align*}
% \end{frame}

% \begin{frame}{Command Translation}
%     \begin{itemize}
%         \item Commands are translated with $ct$.
%         \item Argument translation $qt$ used to differentiate between qubits and register accesses, get $uid$.
%         \item $control$ adds the translated $qArg$ as guard to all gate applications in translation.  
%     \end{itemize}
%     \begin{align*}
%         st =& \ [ \texttt{c} \mapsto (\texttt{qubit}, 1, uid_c),\\
%             & \ \ \ \texttt{a} \mapsto (\texttt{qubit}, 3, uid_a)]\\
%             &\\
%         ct(\texttt{if c do x a[0]; end}, st) =& \ control(qt(c, st), bt(\texttt{x a[0];}, st))\\
%                                              =& \ control(uid_c, \texttt{x } uid_a[0] \texttt{;})\\
%         qt(c, st) =& \ st[c]_3 = uid_c\\
%         bt(\texttt{x a[0];}, st) =& \ \texttt{x } qt(a[0]) \texttt{;} = \texttt{x } uid_a[0] \texttt{;}\\
%             &\\
%         control(uid_c, \texttt{x } uid_a[0] \texttt{;}) =& \ \texttt{ctrl(1) @ x } uid_c, uid_a \texttt{;} 
%     \end{align*}
    % \begin{align*}
    %     ct : \ & Statement \times SymbolTable \dashrightarrow QASM \\        
    %     ct(\texttt{qif } qArg \texttt{ do } blk \texttt{ end}, st) = \ &  control(qt(qArg, st), bt(blk, st)) \\
    % \end{align*}
    % \vspace{\alignmargin}
    % \begin{align*}
    %     control(qArg, &&&  id \ qArg_1, \dots, qArg_{n'} \texttt{;}) = \\  
    %     & \ \texttt{ctrl(1) @ } && id \ qArg, qArg_1, \dots, qArg_{n'}\texttt{;}\\
    %     %
    %     control(qArg, & \ \texttt{ctrl(}n \texttt{) @ } && id \ qArg_1, \dots, qArg_{n'} \texttt{;}) = \\
    %     & \ \texttt{ctrl(}n+1 \texttt{) @ } && id \ qArg, qArg_1, \dots, qArg_{n'}\texttt{;}
    % \end{align*}
% \end{frame}

% \begin{frame}{Command Translation}
%     \begin{itemize}
%         \item The commands are translated with the $ct$ function.
%         \item An example of a composite gate translation is depicted below.
%         \item The qubit argument translation $qt$ is used to differentiate between qubits and register accesses.
%         \item From the symbol table, the information about the composite gate is retrieved.
%     \end{itemize}
%     \begin{align*}
%         ct : \ & Statement \times SymbolTable \dashrightarrow QASM\\
%         ct(id \ qArg_1, \dots, qArg_n\texttt{;}, st) = \ & bt(blk, st_\epsilon[id_1 \mapsto (\texttt{arg}, q_1), \dots, id_n \mapsto (\texttt{arg}, q_n)]) \\
%             &\text{if } id \not\in ConstGates\\
%             &\text{where } q_i = qt(qArg_i, st), i \in [1, ...\, n]\\
%             &\text{and } st[id] = (\texttt{gate}, blk, id_1, \dots, id_n)
%     \end{align*}
% \end{frame}

% \begin{frame}{Qubit Argument Translation}
%     \begin{itemize}
%         \item ...
%     \end{itemize}
%     \begin{align*}
%         qt :\ & \rlap{$QubitArgument \times SymbolTable \dashrightarrow QubitArgument$} &&\\
%         qt(id, st) = \ & uid  && \text{if } st[id] = (\texttt{qubit}, 1, uid)\\
%         qt(id[exp], st) = \ & uid\texttt{[}\underbrace{at(exp, st)}_{n}\texttt{]} && \text{if } st[id] = (\texttt{qubit}, m, uid) \text{ and } m > n\\
%     \end{align*}
% \end{frame}

