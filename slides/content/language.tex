\section{Language}
\subsection{Overview}
\begin{frame}{Language Overview}
    \begin{itemize}
        \item The idea for our language is to provide a high-level language with the capabilities of the QCM. 
        \item We want to remove low-level concepts and add high-level ones.
        \item Additionally, since jump instructions in superposition are removed, we need to add other control flow statements so that the
        language is as expressive as the QCM.
        \item For this, we introduce multiple high-level concepts and two basic control flow statements:
        \begin{itemize}
            \item Blocks and scopes
            \item Different data types
            \item Composite gates
            \item Loop statements, unrolled at compile time
            \item Quantum if- and else-statements
        \end{itemize}
    \end{itemize}
\end{frame}

\subsection{Syntax}
\begin{frame}{Syntax}
    \begin{itemize}
        \item We define a CFG $CFG_{Luie}$ for our language.
        \item The start symbol is the program, consisting of arbitrarily many gate declarations and a block.
        \item A block is a list of translatables, either statements of declarations.
    \end{itemize}
    \begin{align*}
        CFG_{Luie} = \ & (V_{Luie}, \Sigma_{Luie}, R_{Luie}, prg_{Luie} )\\ 
        V_{Luie} = \ & \{ exp, rExp, gate,  qArg, stm, prg_{Luie}, \dots \}\\ 
        \Sigma_{Luie} = \ & \{\texttt{..}, \texttt{range}, \texttt{(}, \texttt{)}, \dots \} 
        \quad \texttt{where } n \in \mathbb{N}_0, id \in Identifier
    \end{align*}
    \vspace{\alignmargin}
    \begin{align*}
        Program: \ & prg_{Luie} ::= gDcl_1 \dots gDcl_n \ blk \mid blk \\
        Block: \ & blk::= t_1 \dots t_n \mid \epsilon\\
        Translatable : \ & t::= stm \mid dcl\\
    % \end{align*}
    % \begin{align*}
        Declaration: \ & dcl ::= \texttt{const } id \texttt{ = } exp \texttt{;} \mid \\
                     & \quad \quad \quad \texttt{qubit } id \texttt{;}\\
        GateDeclaration: \ & gDcl::= \texttt{gate } id \texttt{ (}id_1, \dots, id_n\texttt{) do } blk \texttt{ end}
    \end{align*}
\end{frame}

\begin{frame}{Syntax}
    \begin{itemize}
        \item There are three different statements: quantum if-statement, loop statement, and application of predefined or composite gate.
        \item The qubit argument differentiates between qubit or register access.
        \item For the register access or constant declarations, expressions can be used.
        \item Additionally, we defined a set of defined gates to differentiate the corresponding translations.
    \end{itemize}
    \begin{align*}
        Statement: \ & stm ::= \texttt{qif } qArg \texttt{ do }  blk \texttt{ end} \mid\\
                     & \quad \quad \quad \quad \texttt{for } id \texttt{ in } rExp \texttt{ do } blk \texttt{ end} \mid \\
                     & \quad \quad \quad \quad id \ qArg_1, \dots, qArg_n \texttt{;}\\
        QubitArgument: \ & qArg ::= id \mid id[exp]\\
        Expression: \ & exp ::= n \mid id \mid exp_1 \texttt{ + } exp_2 \mid exp_1 \texttt{ - } exp_2 \mid \dots\\
        RangeExpression: \ & rExp ::= n_1 .. n_2 \mid \texttt{range(} exp \texttt{)} \mid \texttt{range(} exp_1, exp_2 \texttt{)}
    \end{align*}
    \vspace{\alignmargin}
    \begin{equation*}
        ConstGates = \{\texttt{h}, \texttt{x}, \texttt{y}, \texttt{z}, \texttt{cx}, \texttt{ccx}\}
    \end{equation*}
\end{frame}

\begin{frame}{Example Program}
    \begin{figure}[htp]
        \centering     
        \lstinputlisting[style=Luie]{../figures/code/slides/luie_example.luie}
        \caption{An example Luie program.}
        \label{fig:qft_example}
    \end{figure}
\end{frame}


\subsection{Translation}
\begin{frame}{Symbol Table}
    \begin{itemize}
        \item The symbol tables saves the symbol information relevant for the translation.
        \item It contains four different types of symbols:
        \begin{enumerate}
            \item Named constants
            \item Quantum registers and qubits
            \item Qubit arguments
            \item Composite gates 
        \end{enumerate} 
    \end{itemize}
    \begin{align*}
        SymbolTable := \ \{st \mid st : Identifier \dashrightarrow & (\{\texttt{const}\} \times \mathbb{Q})\\
        \cup& (\{\texttt{qubit}\} \times \mathbb{N} \times Identifier)\\
        \cup& (\{\texttt{arg}\} \times QubitArgument)\\
        \cup& (\{\texttt{gate}\} \times Block \times Identifier^+)
        \}
    \end{align*}
\end{frame}

\begin{frame}{Translation Function and Block Translation}
    \begin{itemize}
        \item The $trans$ function translate the Luie program to OpenQASM.
        \item The initial symbol table $st_\epsilon$ contains no mappings.
        \item Next, the block translation function $bt$ translates all translatables, \ie, statements and declarations.
    \end{itemize}
    \begin{align*}
        trans : \ & Program \dashrightarrow QASMProgam\\
        trans(gDcl_1 \dots gDcl_n \ blk) = \ & bt(blk, update(update(update(st_\epsilon, gDcl_1), ...), gDcl_n))\\
        &\\
        bt : \ & Block \times SymbolTable \dashrightarrow QASM\\
        bt(t_1 \dots t_n, st_1) = \ &  tr_1 \quad \text{where } (tr_1, st_2) = tt(t_1, st_1)\\
        & \dots\\
        & tr_n \quad \text{where } (tr_n, \_) = tt(t_n, st_n)
    \end{align*}
\end{frame}

\begin{frame}{Translatable and Declaration Translation}
    \begin{itemize}
        \item The translatable function $tt$ translates each translatable.
        \item Since declarations update the symbol table, the function returns the updated symbol table.
        \item The language allows for different variable scopes and, in turn, an identifier can be used multiple times.
        \item Therefore, a unique identifier $uid$ is generated for the translation of identifiers. 
    \end{itemize}
    \begin{align*}
        tt : \ & Translatable \times SymbolTable \dashrightarrow QASM \times SymbolTable\\
        tt(t, st) = \ & \begin{cases}
            dt(t, st)  \quad &\text{if } t \in Declarations\\
            (ct(t, st), st) &\text{otherwise }
        \end{cases}\\
        &\\
        dt : \ & Declaration \times SymbolTable \dashrightarrow QASM \times SymbolTable\\
        dt(\underbrace{\texttt{qubit } id \text{;}}_{decl}, st) = \ & (\texttt{qubit } uid\texttt{;}, st')\\
                                                                    & \text{where } st' = update(decl, st) \text{ and } st'[id] = (\texttt{qubit}, 1, uid)\\
    \end{align*}
\end{frame}

\begin{frame}{Command Translation}
    \begin{itemize}
        \item The commands are translated with the $ct$ function.
        \item We take a look at an example translation of a quantum if-statement.
        \item The qubit argument translation $qt$ is used to differentiate between qubits and register accesses and looks up the $uid$.
        \item The $control$ function adds the translated $qArg$ as a guard to all gate applications in the block translation.  
    \end{itemize}
    \begin{align*}
        ct : \ & Statement \times SymbolTable \dashrightarrow QASM \\        
        ct(\texttt{qif } qArg \texttt{ do } blk \texttt{ end}, st) = \ &  control(qt(qArg, st), bt(blk, st)) \\
    \end{align*}
    \vspace{\alignmargin}
    \begin{align*}
        control(qArg, &&&  id \ qArg_1, \dots, qArg_{n'} \texttt{;}) = \\  
        & \ \texttt{ctrl(1) @ } && id \ qArg, qArg_1, \dots, qArg_{n'}\texttt{;}\\
        %
        control(qArg, & \ \texttt{ctrl(}n \texttt{) @ } && id \ qArg_1, \dots, qArg_{n'} \texttt{;}) = \\
        & \ \texttt{ctrl(}n+1 \texttt{) @ } && id \ qArg, qArg_1, \dots, qArg_{n'}\texttt{;}
    \end{align*}
\end{frame}

% \begin{frame}{Command Translation}
%     \begin{itemize}
%         \item The commands are translated with the $ct$ function.
%         \item An example of a composite gate translation is depicted below.
%         \item The qubit argument translation $qt$ is used to differentiate between qubits and register accesses.
%         \item From the symbol table, the information about the composite gate is retrieved.
%     \end{itemize}
%     \begin{align*}
%         ct : \ & Statement \times SymbolTable \dashrightarrow QASM\\
%         ct(id \ qArg_1, \dots, qArg_n\texttt{;}, st) = \ & bt(blk, st_\epsilon[id_1 \mapsto (\texttt{arg}, q_1), \dots, id_n \mapsto (\texttt{arg}, q_n)]) \\
%             &\text{if } id \not\in ConstGates\\
%             &\text{where } q_i = qt(qArg_i, st), i \in [1, ...\, n]\\
%             &\text{and } st[id] = (\texttt{gate}, blk, id_1, \dots, id_n)
%     \end{align*}
% \end{frame}

% \begin{frame}{Qubit Argument Translation}
%     \begin{itemize}
%         \item ...
%     \end{itemize}
%     \begin{align*}
%         qt :\ & \rlap{$QubitArgument \times SymbolTable \dashrightarrow QubitArgument$} &&\\
%         qt(id, st) = \ & uid  && \text{if } st[id] = (\texttt{qubit}, 1, uid)\\
%         qt(id[exp], st) = \ & uid\texttt{[}\underbrace{at(exp, st)}_{n}\texttt{]} && \text{if } st[id] = (\texttt{qubit}, m, uid) \text{ and } m > n\\
%     \end{align*}
% \end{frame}

